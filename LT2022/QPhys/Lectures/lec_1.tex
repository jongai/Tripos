\lecture{1}{21 Jan. 12:00}{Introduction}
\section*{Introduction}
At the end of the 19th century, people classified things as either particles or waves (strings or electromagnetic waves). It is thought that we can have infinite precision if we have enough computing power and data.

In the early 20th century, there are experimental challenges.
\begin{enumerate}
    \item Photoelectric effect
    \item Black body radiation
    \item Electron diffraction
    \item Atomic structure etc.
\end{enumerate}

\subsection{Photoelectric Effect}
When you shine a light to induce a photocurrent, the stopping voltage satisfies
\[
    \frac{1}{2}m v^2 = eV_0= E - W.
\]
By looking at the comparison between photocurrent and retarding voltage, we see that increasing light intensity gives the same stopping voltage, while increasing light frequency increases the stopping voltage. It would no make sense with waves, but considering the photons as packages makes sense. That is
\[
    eV_0 = h\nu - W,\quad V_0 = \frac{h\nu}{e} - \frac{W}{e}
\]
where \(\nu\) is a discrete amount. So we have
\[
    E = h\nu = \hbar \omega.
\]

\subsection{Black body}
We have Energy Density \(=\) Density per mode \(\times\) Energy per mode,
\[
    \rho(\lambda,T)d\lambda = \frac{8\pi}{\lambda^4}k_B Td\lambda.
\]
It means that when the wavelength is short, the black body energy blows up, which does not match up with experimental data where the energy decreases as the wave length gets further smaller. It can be explained by Plank's radiation formula where the packages are considered instead.
\[
    \rho_E = \frac{8\pi\rho \gamma^3}{c^3} \frac{1}{e^{\nicefrac{hv}{k_{B}T}} - 1}.
\]