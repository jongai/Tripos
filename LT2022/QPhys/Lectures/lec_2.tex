\lecture{2}{24 Jan. 12:07}{}
\subsection{Atoms}
If the electrons rotate around the proton, it would create radiation. Thus, it would emit radiation and lose energy. So it cannot be the correct model. Bohr suggested that if the waves behave like particles, maybe particles behave like waves as well.

\[
    p = \frac{h}{\lambda}.
\]

If there are stable orbits, \(2\pi r_n = \mu \lambda\).

And we end up with
\[
    E_n = -\frac{hcR}{n^2}
\]
where \(R\) is called \textit{Rydberg constant}.

In sum, the angular momentum of an orbiting electron is quantized to fix the problem.

\subsection{Electron Diffraction}
By considering electrons as waves, it exhibits diffraction pattern as well.
\[
    \lambda = \frac{h}{p} = \frac{t}{\sqrt{2eVm} }.
\]
The first maximum occurs at \(\lambda = d \sin \theta\) theoretically, and it agrees well with experimental results.

\subsection{Quantum Interference}
Even when one photon is sent through the slit at a time, interference patterns are described. The photon seems to be interfering with itself.

\section{Wave Functions}
\subsection{Introduction}
\leavevmode
\begin{definition}{}{}
    We define a \textit{wavefunction} as
    \begin{align*}
        \varphi(x, t) &= A e^{i(kx - \omega t)},\quad (k = \frac{p}{t}, \omega = \frac{E}{t})\\
        P(x,t)dx &= \abs{\varphi(x,t)} ^2 dx \quad (\int P dx = 1).
    \end{align*}
\end{definition}
For a free particle, \(E = \frac{1}{2} mv^2 = \frac{p^2}{2m}\). To describe a localized wave function, we need to wave packet.