\lecture{7}{3 Feb. 2022}{}
We continue from the Simpson's Paradox example. Let \(A = \{\text{change color}\}\), \(B = \{ \text{blue}\}\), \(B^c = \{\text{green}\}\), \(C = \{\text{Cambridge}\}\) and \(C^c = \{\text{Oxford}\}\). We have
\begin{align*}
    \probability{}{A\mid B \cap C} &> \probability{}{A \mid B^c \cap C}\\
    \probability{}{A\mid B \cap C^c} &> \probability{}{A \mid B^c \cap C^c}.
\end{align*}
    But it is not true that \(\probability{}{A \mid B} > \probability{}{A \mid B^c} \). LTP for conditional probabilities is the following. Suppose \(C_1, C_2, \ldots\) is a partition of \(B\), and we have
    \begin{align*}
        \probability{}{A\mid B} &= \frac{\probability{}{A \cap B}}{\probability{}{B}} = \frac{\probability{}{A \cap (\cup_n C_n)}}{\probability{}{B} }\\
        &= \frac{\probability{}{\cup_n (A \cap C_n)} }{\probability{}{B} } = \frac{\sum\limits_{n}\probability{}{A \cap C_n} }{\probability{}{B}}\\
        &= \frac{\sum_n \probability{}{A \mid C_n} \probability{}{C_n}}{\probability{}{B} } = \sum_n \probability{}{A\mid C_n} \frac{\probability{}{B \cap C_n}}{\probability{}{B}}
    \end{align*}
    So in conclusion, we have
    \[
        \probability{}{A\mid B} = \sum_n \probability{}{A\mid C_n} \probability{}{C_n\mid B} .
    \]
Special Case:
\begin{itemize}
\item If all \(\probability{}{C_n}\) are equal, then \(\probability{}{C_n \mid B} \) are all equal.
\item If \(\probability{}{A\mid C_n} \) are all equal. Note that \(\sum_n \probability{}{C_n \mid B}=1\). Then we have
\[
    \probability{}{A\mid B} = \probability{}{A\mid C_n}.
\]
\end{itemize}
\begin{example}
    Uniform permutation \((\sigma(1),\sigma(2), \ldots , \sigma(52)) \in \Sigma_{52}\) ("well-shuffled cards"). We call \(\{1,2,3,4\}\) the aces. We consider \(A= \{\sigma(1), \sigma(2)\text{ aces}\}\), and \(B = \{\sigma(1) \text{ ace}\} = \{\sigma(1) \leq 4\}\), \(C_i = \{\sigma(1) = i\}\).

    Note \(\probability{}{A \mid C_i} = \probability{}{\sigma(2) \in \{1,2,3,4\} \mid \sigma(1) = i} = \frac{3}{51}\)  for \(i \leq 4\) by previous example. And we have \(\probability{}{C_i}  = \frac{1}{52}\). So we have \(\probability{}{A\mid B} = \frac{3}{51}\). In total, we have
    \[
        \probability{}{A} = \probability{}{B} \times \probability{}{A\mid B} = \frac{4}{52}\times \frac{3}{51}.
    \]
\end{example}
\section{Discrete Random Variables}
Motivation: Roll two dices. \(\Omega=\{1, \ldots,6\}^2 = \{(i,j)\mid 1\leq i,j \leq 6\}\). If we restrict attention to first dice \(\{(i,j)\mid i = 3\}\); sum of dices \(\{(i,j) \mid i + j = 8\}\); max of dice \(\{(i,j)\mid i,j \leq 4, i\text{ or }j=4\}\).

Goal: "Random real-valued measurements".

\begin{definition}{}{}
    A \textit{discrete random variable} \(X\) (often denoted by RV) on a probability space \((\Omega, \mathcal{F}, \probability{}{})\) is a function \(X: \Omega\to \mathbb{R}\) such that
    \begin{enumerate}
        \item \(\{\omega \in \Omega\mid X(\omega) = x\}\in \mathcal{F} \).
        \item \(\ima(X)\) is finite or countable (subset of \(\mathbb{R}\)).
    \end{enumerate}
    We can write \(\{\omega \in \Omega\mid X(\omega) = x\}\) as \(\{X = x\}\). So \(\probability{}{X=x} \) is valid. And the image is often \(\mathbb{Z}\) or \(\{0,1\}\) for example, instead of \(\{\text{Heads}, \text{Tails}\}\).

    If \(\Omega\) is finite or countable, and \(\mathcal{F}=\mathcal{P} (\Omega)\), both requirements hold automatically.
\end{definition}
\begin{example}[Part II Applied Probability]
    If we consider the arrival problem, we have \(\Omega=\{\text{countable subsets } (a_1,a_2, \ldots) \text{ of }(0, \infty)\}\). Then,
    \begin{align*}
        N_t &= \text{number of arrivals by time t}\\
        &= \abs{\{a_i \mid a_i \leq t\}} \in \{0,1,2, \ldots\}
    \end{align*}
    is a discrete RV for each time \(t\).
\end{example}
\begin{definition}{}{}
    The \textit{probability mass function} (p.m.f.) of discrete RV \(X\) is the function \(p_X:\mathbb{R} \to [0,1]\) given by
    \[
        p_X(x) = \probability{}{X = x}\quad \forall x \in \mathbb{R}.
    \]
\end{definition}
\begin{note}
    \leavevmode
    \begin{itemize}
    \item If \(x \notin \ima(X)\) (that is, \(X(\omega)\) never takes value \(x\)), then
    \[p_X(x) = \probability{}{\omega \in \Omega\mid X(\omega) = x}=\probability{}{\varnothing} = 0.\]
    \item \(\begin{aligned}[t]\sum\limits_{x\in (X)}p_X(x) &= \sum\limits_{x\in \ima(x)}\probability{}{\{\omega \in \Omega\mid X(\omega) = x\}}\\
    &=\probability{}{\cup_{x\in \ima(X)}\{\omega\in \Omega\mid X(\omega)=x\}}=\probability{}{\Omega} = 1 \end{aligned}\).
    \end{itemize}
\end{note}
\begin{example}[Indicator Function]
    Event \(A\in \mathcal{F} \), define \(\mathbf{1}_A:\omega \to \mathbb{R}\) by
    \[
    \textbf{1}_A(\omega) = \begin{dcases}
        1, &\text{ if } \omega \in A\\
        0, &\text{ if }\omega \notin A \\
    \end{dcases}
    \]
    called the \textit{indicated function} of \(A\). \(\textbf{1}_{A}\) is a discrete RV with \(\ima(\textbf{1}) = \{0,1\}\). The probability mass function is
    \begin{align*}
        p_{\textbf{1}_A}(1) &= \probability{}{\textbf{1}_A = 1} = \probability{}{A}\\
        p_{\textbf{1}_A}(0) &= \probability{}{\textbf{1}_A = 0} = \probability{}{A^c} = 1 - \probability{}{A} \\
        p_{\textbf{1}_A}(x) = 0 \quad \forall x \notin \{0,1\}.
    \end{align*}
    It encodes "did A happen" as a real number.
\end{example}
\begin{remark}
    Given a probability mass function, we can always construct a probability space \((\Omega, \mathcal{F}, \mathbb{P})\) and a RV defined on it with this pmf.
    \begin{itemize}
    \item \(\Omega = \ima(X)\). That is, \(\{x \in \mathbb{R}\mid p_X(x) > 0\}\);
    \item \(\mathcal{F} = \mathcal{P}(\Omega)\);
    \item \(\probability{}{\{x\}} = p_X(x)\) and extend it to all \(A \in \mathcal{F}\).
    \end{itemize}
\end{remark}