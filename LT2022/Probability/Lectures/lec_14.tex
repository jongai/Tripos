\lecture{14}{19 Feb. 2022}{}
\subsubsection{Gambler's Ruin}
We have
\begin{align*}
    \pr_x(\text{hit }0) &= \lim_{a \to \infty} \pr_x(\text{hit }0\text{ before }a)\\
    &= \begin{dcases}
        1 - (\frac{q}{p})^x, &\text{ if } p > q \quad\text{\color{red}Is this true??}\\
        1, &\text{ if } p \leq q\\
    \end{dcases}
\end{align*}
When \(p = \frac{1}{2}\),
\[
    \ex[\text{time to hit }0] \geq \ex[\text{time to hit }0\text{ or }a] = x(a - x) \to \infty.
\]
So \(T_x\), the time to hit \(0\) from \(x\) is finite with probability \(1\) but infinite expectation.

\begin{remark}
    Alternative derivation that \(\ex[T_1] = \infty\).
    \begin{align*}
        \ex[T_1] &= \frac{1}{2}\times 1 + \frac{1}{2}(1 + \ex[T_2])\\
        &= 1 + \ex[T_1].
    \end{align*}
    Conclude that \(\ex[T_1] = \infty\).
\end{remark}
\subsection{Generating Functions}
\begin{definition}{}{}
    If \(X\) is a RV taking values in \(\{0,1,2,\dots\}\), the \textit{probability generating function} of \(X\) is
    \[
        G_X(z) = \ex[z^X] = \sum_{k \geq 0}z^k \pr(X = k).
    \]
    It can be seen as a function \((-1,1) \to \mathbb{R}\).
\end{definition}
\begin{remark}
    Probability generating functions encodes the distribution of \(X\) as a function with nice analytic properties.
\end{remark}
\begin{example}
    If \(X \sim \mathrm{Bern}(p)\), then
    \[
        G_X(z) = z^0 \pr(X=0) + z^1\pr(X = 1) = (1-p) + pz.
    \]
\end{example}
\begin{example}
    If \(X \sim \mathrm{Poisson}(\lambda)\), then
    \[
        G_X(z) = \sum_{k \geq 0}z^k e^{-\lambda}\frac{\lambda^k}{k!} = e^{-\lambda}\sum_{k\geq 0}\frac{(\lambda z)^k}{k!} = e^{\lambda(z-1)}.
    \]
\end{example}
One can recover PMF from PGF. Note that
\[
    G_X(0) = 0^0 \pr(X=0) = \pr(X=0).
\]