\lecture{1}{20 Jan. 11:00}{Probability Space}
\begin{example}
If we have a die with outcomes 1, 2, \ldots, 6. 
\begin{enumerate}
    \item $\mathbb{P}(2) = \frac{1}{6}$
    \item $\mathbb{P}(\text{multiple of }3) = \frac{2}{6} = \frac{1}{3}$
    \item \(\mathbb{P}(\text{pair or a multiple of }3) = \frac{4}{6} = \frac{2}{3}\) 
\end{enumerate}
\end{example}
\section{Formal Setup}
We try to define a probability space rigorously in this section.
\begin{definition}{Probability Space}{prospace}
    We have the following,
    \begin{enumerate}
        \item Sample space \(\Omega\), a set of outcomes.
        \item \(\mathcal{F}\), a collection of subsets of \(\Omega\) (called events).
        \item \(\mathcal{F}\) is a \(\sigma\)-algebra if
        \begin{enumerate}
            \item \textbf{F1}: \(\Omega\in\mathcal{F} \)
            \item \textbf{F2}: if \(A \in \mathcal{F} \) then \(A^c \in \mathcal{F} \)
            \item \textbf{F3}: For all countable collections \(\{A_n\}\) in \(\mathcal{F} \), \(\cup_n A_n \in \mathcal{F} \).
        \end{enumerate} 
    \end{enumerate}
    Given \(\sigma\)-algebra \(\mathcal{F}\) on \(\Omega\), function $\mathbb{P}: \mathcal{F} \to [0,1]$ is a probability measure if
    \begin{enumerate}
        \item \textbf{P1}: The probability function is nonnegative.
        \item \textbf{P2}: \(\mathbb{P}(\Omega) = 1\) 
        \item \textbf{P3}: For all countable collection \(\{A_n\}\) of disjoint events in \(\mathcal{F} \), we have
        \(
            \mathbb{P}(\cup_n A_n) = \sum\limits_{n=1}^{\infty} \mathbb{P}(A_n).
        \)
    \end{enumerate}
    Then \((\Omega, \mathcal{F}, \mathbb{P})\) is a probability space. 
\end{definition}

\begin{problem}
    Why \(\mathbb{P} : \mathcal{F} \to [0,1]\), not \(\mathbb{P}: \Omega \to [0,1]\)?
\end{problem}
We will justify the definition in the following examples.
\begin{example}
When \(\Omega\) is finite or countable,
\begin{enumerate}
    \item In general: \(\mathcal{F} = \mathcal{P} (\Omega)\).
    \item \(\mathbb{P}(2)\) is shorthand for \(\mathbb{P}(\{2\})\).
    \item \(\mathbb{P}\) is determined by  \(\mathbb{P}(\{w\}), \forall w \in \Omega\).
\end{enumerate}
\end{example}

\begin{remark}
When \(\Omega\) is uncountable, a probability space behaves differently, as shown in the following example.
\begin{example}
    If \(\Omega = [0,1]\), and we want to choose a real number, all equally likely.

    If \(\mathbb{P}\{0\} = \alpha > 0\), then \(\mathbb{P}(\{0,1,\frac{1}{2},\ldots ,\frac{1}{n}\} = n\alpha)\). This cannot happen if \(n\) large, because we would have \(\mathbb{P} > 1\). So \(\mathbb{P}(\{0\}) = 0\) or undefined. 
\end{example}
\end{remark}
\begin{example}
    When \(\Omega\) is infinitely countable (e.g., \(\Omega = \mathbb{N}\) or \(\Omega = \mathbb{Q}\cap [0,1]\)), however, it is not possible to choose uniformly. Suppose it is possible, there are two possibilities
    \begin{itemize}
        \item If \(\probability{}{\{\omega\}} = \alpha \quad \forall \omega \in \Omega\),

        then \(\probability{}{\Omega} = \sum\limits_{\omega \in \Omega} \probability{}{\{\omega\}} = \infty\). \contra
        \item If \(\probability{}{\{\omega\}} = 0 \quad \forall \omega \in \Omega\),

        then \(\probability{}{\Omega} = \sum\limits_{\omega \in \Omega} \probability{}{\{\omega\}} = 0\). \contra
    \end{itemize}
    So it is not possible to have one such uniform probability space. But that's fine as there exists many other interesting probability measures on a infinite countably set.
\end{example}
\begin{property}
From the axioms, we want to prove the following properties of a probability space.
    \begin{enumerate}
\item \(\mathbb{P}(A^c) = 1 - \mathbb{P}(A)\).
\begin{proof}
    \(A, A^c\) disjoint. \(A \cup A^c = \Omega\). So \(\mathbb{P}(A) + \mathbb{P}(A^c) = \mathbb{P}(\Omega) = 1\) 
\end{proof}
\item \(\mathbb{P}(\varnothing) = 0\)
\item If \(A \subseteq B\), then \(\mathbb{P}(A) \leq \mathbb{P}(B)\).
\item \(\mathbb{P}(A\cup B) = \mathbb{P}(A) + \mathbb{P}(B) - \mathbb{P}(A\cap B)\) 
\end{enumerate}
\end{property}

\subsection{Examples of Probability Spaces}
\begin{example}
    Here we list some concrete examples of probability spaces.
    \begin{enumerate}
\item \(\Omega\) finite, \(\Omega = \{w_1, \ldots , w_n \}\), \(\mathcal{F} = \text{all subsets under uniform choice}\).

\(\mathbb{P}:\mathcal{F} \to [0,1], \mathbb{P}(A) = \frac{\abs{A} }{\abs{\Omega} }\).
In particular: \(\mathbb{P}(\{w\}) = \frac{1}{\abs{\Omega}} \forall w \in \Omega\).

\item If we are choosing without replacement
\(n\) indistinguishable marbles that are labelled \(\{1, \ldots, n \}\). Pick \(k \leq n\) marbles uniformly at random.

Here we have \(\Omega=\{A \subseteq \{1, \ldots , n \}\), \(\abs{A} = k\), \(\abs{\Omega} = \binom{n}{k}\).

\item If we have a well-shuffled deck of cards, and we uniformly chose permutation of 52 cards.

\(\Omega = \{\text{all permutations of 52 cards}\}\). \(\abs{\Omega} = 52!\).

Then we have
\[
    \mathbb{P}(\text{first three cards have the same suit}) = \frac{52 \cdot 12 \cdot 11 \cdot 49!}{52!} = \frac{22}{425}.
\]
    \end{enumerate}
\end{example}