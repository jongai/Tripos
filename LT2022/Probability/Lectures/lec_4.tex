\lecture{3}{27 Jan. 2022}{Inclusion-Exclusion Principle}
\begin{proof}
    We used induction. The \(n = 2\) case is proved in the example sheet.
    \begin{align*}
        \probability{}{\bigcup_{i=1}^n A_i} &= \probability{}{(\bigcup_{i=1}^{n-1}A_i)\bigcup A_n}\\
        &= \probability{}{\bigcup_{i=1}^{n-1}A_i} + \probability{}{A_n} -\probability{}{(\bigcup_{i=1}^{n-1}A_i)\bigcap A_n}.
    \end{align*}
    Note that for \(J \subseteq \{1, \ldots ,n-1 \}\),
    \[
        \bigcap_{i\in J}(A_i\cap A_{n})=\bigcap_{i\in J\cup \{n\}}A_i.
    \]
    The inductive statement tells us
    \begin{align*}
        \probability{}{\bigcup_{i=1}^n A_i} =& \sum\limits_{\substack{J\subseteq \{1, \ldots ,n-1 \}\\ J \neq \varnothing}} (-1)^{\abs{J} +1} \mathbb{\MakeUppercase{p}}(\bigcap_{i\in J}A_i) + \probability{}{A_n}\\
        &- \sum\limits_{\substack{J\subseteq \{1, \ldots , n - 1 \}\\ J \neq \varnothing }}(-1)^{\abs{J} +1}\probability{}{\bigcap_{i\in J \cup \{n\}}A_i}\\
        =& \sum\limits_{\substack{I\subseteq \{1, \ldots ,n-1 \}\\ I \neq \varnothing}} (-1)^{\abs{I} +1} \mathbb{\MakeUppercase{p}}(\bigcap_{i\in I}A_i) + \probability{}{A_n}\\
        &+ \sum\limits_{\substack{I\subseteq \{1, \ldots , n - 1 \}\\ n\in I, \abs{I} \geq 2}}(-1)^{\abs{I} +1}\probability{}{\bigcap_{i\in I}A_i}\\
        =& \sum\limits_{\substack{I \subseteq \{1, \ldots ,n \}\\ I \neq \varnothing}} (-1)^{\abs{I} +1} \probability{}{\bigcap_{i\in I}A_i}.
    \end{align*}
\end{proof}
\subsection{Bonferroni Inequalities}
\begin{problem}
    What if you truncate Inclusion-Exclusion Principle?
\end{problem}
Recall countable subadditivity states that \(\probability{}{\cup A_i}\leq \sum \probability{}{A_i} \), also known as union bound. We have the following inequalities.
\begin{itemize}
    \item \(\probability{}{\cup_{i=1}^n A_i} \leq \sum\limits_{k=1}^{r} (-1)^{k+1}\sum\limits_{i_1 < \ldots < i_k}\probability{}{A_{i_1}\cap \ldots \cap A_{i_k} } \) when \(r\) is odd;
    \item \(\probability{}{\cup_{i=1}^n A_i} \geq \sum\limits_{k=1}^{r} (-1)^{k+1}\sum\limits_{i_1 < \ldots < i_k}\probability{}{A_{i_1}\cap \ldots \cap A_{i_k} } \) when \(r\) is even.
\end{itemize}
\begin{problem}
    When is it good to truncate at, for example, \(r = 2\)?
\end{problem}
\begin{proof}
    We induct on \(r\) and \(n\). When \(r\) is odd
     \begin{align*}
         \probability{}{\bigcup_{i=1}^n A_i} =& \probability{}{\bigcup_{i = 1} A_i} + \probability{}{A_n} - \probability{}{\bigcup_{i=1}^{n-1}(A_i \cap A_n)}\\
         \leq& \sum\limits_{\substack{J\subseteq \{1, \ldots , n - 1\}\\1 \leq  \abs{J} \leq r }} (-1)^{\abs{J} +1}\probability{}{\bigcap_{i\in J} A_i}+\probability{}{A_n} \\
         &- \sum\limits_{\substack{J\subseteq \{1, \ldots , n - 1\}\\1 \leq  \abs{J} \leq r -1}} (-1)^{\abs{J} +1}\probability{}{\bigcap_{i\in J \cup \{n\}} A_i}\\
         \leq& \sum\limits_{\substack{I\subseteq \{1, \ldots , n\}\\1 \leq  \abs{I} \leq r }} (-1)^{\abs{I} +1}\probability{}{\bigcap_{i\in I} A_i}.
     \end{align*}
     And a similar argument follows when \(r\) is even.
\end{proof}
\subsection{Counting with IEP}
Inclusion Exclusion Principle gives up a route to solve questions that do not have a closed form answer.

When we have a uniform probability measure on \(\Omega\) with \(\abs{\Omega} < \infty\),
\[
    \probability{}{A} =\frac{\abs{A}}{\abs{\Omega} }~\forall A \subseteq \Omega.
\]
Then \(\forall A_1, \ldots, A_n \subseteq \Omega \),
\[
    \abs{A_1 \cup \ldots \cup A_n} = \sum\limits_{k=1}^{n} (-1)^{n+1}\sum\limits_{i_1< \ldots < i_k }\abs{A_{i_1}\cap \ldots \cap A_{i_k} },
\]
and similarly for Bonferroni inequalities.

\begin{example}
    We count the number of surjections \(f: \{1, \ldots, n\}\to \{1, \ldots ,m \}\) with \(n \geq m\).

    We have the probability space and event
    \begin{align*}
        \Omega &= \{f:\{1, \ldots ,n\}\to \{1, \ldots , m \}\},\\
        A &= \{f: \ima(f) = \{1, \ldots , m \}\}.
    \end{align*}

    For all \(i \in \{1, \ldots ,m \}\), let \(B_i = \{f\in \Omega\mid i \notin \ima(f)\}\). We have the following key observations:
    \begin{itemize}
        \item \(A = B_1^c \cap \ldots B_m^c = (B_1 \cup \ldots \cup B_m )^c \).
        \item \(\abs{B_{i_1}\cap \ldots \cap B_{i_k}} \) is nice to calculate, and we have
        \[
            \abs{B_{i_1}\cap \ldots \cap B_{i_k} } = \abs{\{f\in \Omega\mid i_1, \ldots, i_k \notin \ima(f) \}} = (m-k)^n.
        \]
        So by IEP, we have 
        \begin{align*}
            \abs{B_1 \cup \ldots \cup B_m} &= \sum\limits_{k=1}^{m} (-1)^{k+1}\sum\limits_{i_1<\ldots<i_{k}} \abs{B_{i_1}\cap \ldots \cap B_{i_k}}\\
            &= \sum\limits_{k=1}^{m} (-1)^{k+1} \binom{m}{k}(m-k)^n.
        \end{align*}
        So \(\abs{A} = m^n - \sum\limits_{k=1}^{m} (-1)^{k+1}\binom{m}{k}(m-k)^n = \sum\limits_{k=0}^{m} (-1)^k \binom{m}{k}(m-k)^n\).
    \end{itemize}
\end{example}
