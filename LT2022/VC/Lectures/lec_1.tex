\lecture{1}{21 Jan. 10:00}{Introduction}
\begin{problem}
Why do we do Vector Calculus?
\end{problem}
\begin{enumerate}
    \item Calculus is important, and we want to apply it to a wider range of functions.
    \item It is a tool that is needed throughout quantitative sciences.
\end{enumerate}
Lecture notes are online.

We will learn to differentiate and integrate function (or maps) of the form
\[
    f: \mathbb{R}^m \to \mathbb{R}^n.
\]
An element of \(\mathbb{R}^m\) or \(\mathbb{R}^n\) is a vector, so the subject is called vector calculus.

We present some examples of multivariable functions. In general, for a physicist, there are two types of functions, ones where the domain represents a physical space and the ones where the codomain represents a physical space.
\begin{enumerate}
    \item A function \(f: \mathbb{R} \to \mathbb{R}^n\) defines a \textit{curve} in \(\mathbb{R}^n\).

    In physics, we might think of \(\mathbb{R}\) as time and \(\mathbb{R}^n\) as space and write this as
    \[
        f: t \mapsto \mathbf{x} (t)~\text{with}~\mathbf{x} \in \mathbb{R}^n.
    \]
    (Obviously we should take \(n = 3\)).

    Generalizing, a map
    \[
        f: \mathbb{R}^2 \to \mathbb{R}^n
    \]
    defines a \textit{surface} in \(\mathbb{R}^n\), and so on.
    \item In other applications, the domain \(\mathbb{R}^m\) might be viewed as physical space. For example, in physics a \textit{scalar fielid} is a map
    \[
        f: \mathbb{R}^3 \to \mathbb{R}.
    \]
    \begin{eg}
        The temperature \(T(x)\) is a scalar field, as is the Higgs Field
    \end{eg}

    A \textit{vector field} is a map
    \[
        f: \mathbb{R}^3 \to \mathbb{R}^3
    \]
    where the domain is physical space and the codomain is something more abstract.
    \begin{eg}
        The electric field \(\mathbf{E} (\mathbf{x})\) and magnetic field \(\mathbf{B} (\mathbf{x} )\) are vector fields.
    \end{eg}
\end{enumerate}
\section{Curves}
We consider maps of the form
\[
    f:\mathbb{R}\to \mathbb{R}^n.
\]
We assign a coordinate \(t\) to \(\mathbb{R}\) and the Cartesian coordinates on \(\mathbb{R}^n\)
\[
    \mathbf{x} = (x^1, \ldots, x^n ) = x^i \mathbf{e}_i
\]
where \(\mathbf{e} _i\) is orthonormal basis such that \(\mathbf{e} _i \mathbf{e} _j = \delta_{ij}\). (For \(\mathbb{R}^3\)) we also use notation \(\{\mathbf{e} _i\} = \{\mathbf{x} ,\mathbf{y} ,\mathbf{z} \}\).

The image of the function \(f\) is a \textit{parameterised curve} \(\mathbf{x} (t)\), with \(t\) the parameter. We will call the curve \(C\).

\begin{eg}
    Here we give some familiar examples of parameterised curves.
    \begin{enumerate}
        \item Consider the map \(\mathbb{R}\to\mathbb{R}^3\) given by
        \[
            \mathbf{x} (t) = (at, bt^2, 0).
        \]
        The curve \(C\) is the parabola \(ay = bx^2\) in the plane \(z = 0\).
        \begin{note}
            When plotting the curve, we lose information about the parameter \(t\).
        \end{note}
        \item Consider \(\mathbf{x} (t) = (\cos t, \sin t, t)\).

        The curve \(C\) is a helix. The choice of parameterisation is not unique. For example, the map \(\mathbf{x} (t) = (\cos \lambda t,\sin \lambda t, \lambda t)\) gives exactly the same helix.

        Sometimes the choice of parameterisation matters.
        \begin{eg}
            If \(t\)  is time and \(\mathbf{x} (t)\) is position, the velocity is proportional to \(\lambda\).
        \end{eg}
        But we will see that some questions are independent of the choice of parameterisation.
    \end{enumerate}
    \subsection{Differentiating the Curve}
    A vector function \(\mathbf{x} (t)\) is \textit{differentiable} at \(t\) if as \(\delta t \to 0\), we have
    \[
        \mathbf{x} (t + \delta t) - \mathbf{x} (t) = \dot{\textbf{x} }(t)\delta t + O(\delta t^2).
    \]
    \begin{note}
        "Big-O" notation \(O(\delta t^2)\) means terms are proportional to \(\delta t^2\) or smaller.
    \end{note}

    In physics, the dot is usually used for time derivatives, and the prime is used for spacial derivatives. In math, these are used interchangeably.

    We write
    \[
        \delta \mathbf{x} (t) = \mathbf{x} (t + \delta t) - \mathbf{x} (t),
    \]
    and the derivative is then
    \[
        \dot{\mathbf{x}} = \frac{d \mathbf{x} }{dt}\coloneqq \lim\limits_{\delta t \to 0} \frac{\delta \mathbf{x}}{dt}.
    \]
\end{eg}