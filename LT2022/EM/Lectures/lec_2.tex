\lecture{2}{25 Jan. 9:00}{}
We view a non-test body as a collection of \(N\) particles, each viewed as test body, and add them to get the total force on the body.
\begin{align*}
    \mathbf{F} (t) &= \sum\limits_{i=1}^{N} q_i [\mathbf{E} (t,x_i(t)+\mathbf{x}_i \times \mathbf{B} (t,\mathbf{x}_i(t)))]\\
    &= \sum\limits_{i=1}^{N} \int \delta^{(3)}(\mathbf{x} - \mathbf{x}_i) q_i [\mathbf{E} (t,x_i(t)+\mathbf{x}_i \times \mathbf{B} (t,\mathbf{x}_i(t)))] dV\\
    &= \int_{V(t)} \rho(t,\mathbf{x} ) [\mathbf{E} (t,x_i(t)+\mathbf{x}_i \times \mathbf{B} (t,\mathbf{x}_i(t)))] dV.
\end{align*}
Similarly, we can find the torque on the body.
\[
    \tau = \int_{V(t)}\mathbf{x} \times \rho(t,\mathbf{x} ) [\mathbf{E} (t,x_i(t)+\mathbf{x}_i \times \mathbf{B} (t,\mathbf{x}_i(t)))] dV.
\]

\subsection{Maxwell's Equations}
\leavevmode
\begin{theorem}{Maxwell's Equations}{}
    \leavevmode
    \begin{itemize}
        \item \textbf{M1}: \(\nabla \cdot \mathbf{E}  = \frac{\rho}{\epsilon_0}\),
        \item \textbf{M2}: \(\nabla \cdot \mathbf{B}  = 0\),
        \item \textbf{M3}: \(\nabla \times \mathbf{E} = - \frac{\partial \mathbf{B}}{\partial t} \),
        \item \textbf{M4}: \(\nabla \times \mathbf{B} = \mu_0 \mathbf{J} + \frac{1}{c^2} \frac{\partial \mathbf{E} }{\partial t}\),
    \end{itemize}
    with \(\mu_0 = \num{1.2566E-6}\unit{NA^{-2}}\approx 4\pi\times 10^{-6}\unit{NA^{-2}}\).
\end{theorem}
\begin{remark}
    \textbf{M1} - \textbf{M4} are linear, so they obey the superposition principle.
\end{remark}
\begin{proposition}{}{}
    \leavevmode
    \[
        c^2 = \frac{1}{\mu_0\epsilon_0}.
    \]
\end{proposition}
\begin{proof}
    \[
        0 = \frac{\partial }{\partial t} (\nabla \mathbf{E} - \frac{\rho}{\epsilon_0}) = \nabla \cdot \frac{\partial \mathbf{E} }{\partial t} - \frac{1}{\epsilon_0}\frac{\partial \rho}{\partial t} = c^2 \nabla  \cdot (\nabla  \times \mathbf{B} )-c^2\mu_{0}\nabla \cdot \mathbf{J} -\frac{1}{\epsilon_0}\frac{\partial \rho}{\partial t} .
    \]
\end{proof}
Also, we note the conservation of charge equation from before. We must have \(c^2 = \frac{1}{\mu_0 \epsilon_0}\).
\begin{align*}
    \nabla \times (\nabla \times \mathbf{E} ) &= - \frac{\partial }{\partial t} (\nabla \times \mathbf{E} )\\
    \nabla (\nabla\times \mathbf{E}) - \nabla ^2 \mathbf{E}  &= - \mu_0 \frac{\partial \mathbf{J}}{\partial t} - \frac{1}{c^2}\frac{\partial^2 \mathbf{E} }{\partial t^2}  
\end{align*}
Using \textbf{M!}, we have
\[
    -\frac{1}{c^{2}}\frac{\partial^2\mathbf{E}}{\partial x^2} +\nabla ^2 \mathbf{E} =\frac{1}{\epsilon_0}\nabla \rho + \mu_0 \frac{\partial \mathbf{J}}{\partial t} .
\] 
Similarly, we have
\[
    -\frac{1}{c^2}\frac{\partial^2 \mathbf{B}}{\partial t^2} = - \mu_0 \nabla \times \mathbf{J} .
\]
\(\mathbf{E},\mathbf{B} \) obey inhomogeneous wave equations. In vacuum (\(\rho = \mathbf{J} = 0\)) obey homogeneous wave equations. \(c\) is speed parameter of light, so solutions describing waves of speed of light. Light is an electromagnetic wave.
\subsection{Averaging}
\textbf{M1} - \textbf{M4} are microscopic. Maxwell equations are valid even at short (subatomic) distance. Inside charged matter, we average to obtain smooth \(\rho, \mathbf{J} \). \(\mathbf{E} , \mathbf{B} \) also need averaging. So we get new fields \(\mathbf{D}, \textbf{H} \), and new macroscopic Maxwell equations.

The equations we learn are valid for
\begin{enumerate*}
    \item micro equations;
    \item vacuum where the two equations are the same;
    \item in a material that they are almost the same (e.g.,\ air)
\end{enumerate*}

\subsection{Coulomb's and Ohm's Laws}
\leavevmode
\begin{definition}{}{}
    A \textit{conductor} is a material containing "free charges" that can move in response to \(\mathbf{E}, \mathbf{B} \) fields. For example, a solid metal - some electrons can move through the lattice of ions.

    An \textit{insulator} is a material with no free charge, that is a poor conductor.
\end{definition}
\begin{theorem}{Ohm's Law}{}
    Conductor at rest in a field, it is an experimental fact that the current density obeys
    \[
        \mathbf{J} = \sigma \mathbf{E}
    \]
    where \(\sigma\) is called the conductivity of the material.
\end{theorem}
The conductivity is very large for a good conductor (e.g., metals), and very small for insulator (e.g., diamond).

There is a theoretical explanation that will be covered in Part II Electrodynamics.

Consider a conductor occupying region \(V\).
\[
    0 = \frac{\partial \rho}{\partial t} +\nabla \mathbf{J} = \frac{\partial \rho}{\partial t} + \sigma \nabla \cdot \mathbf{E} = \frac{\partial \rho}{\partial t} + \frac{\sigma}{\epsilon_0}\rho.
\]
We have a first order differential equation that we can solve, and we have
\[
    \rho(t,\mathbf{x} ) = \rho_0(\mathbf{x} )e^{-\nicefrac{t}{t_{\text{decay}}}}. \quad t_{\text{decay}} = \frac{\epsilon_0}{\sigma}
\]
So charge density always decrease rapidly in \(V\), the charge ends up on surface of the conductor.

Consider instead metal moving with velocity \(v\) (i.e. lattice ions have velocity \(v\)), then Ohm's law is
\[
    \mathbf{J}  = \sigma(\mathbf{E} + \mathbf{v} \times \mathbf{B}).
\]
\section{Electrostatics}
\subsection{Time independent}
We have
\begin{itemize}
    \item \textbf{1A}: \(\nabla \cdot \mathbf{E}  = \frac{\rho}{\epsilon_0}\),
    \item \textbf{1B}: \(\nabla \times \mathbf{E} = 0\),
    \item \textbf{2A}: \(\nabla \cdot \mathbf{B}  = 0\),
    \item \textbf{2B}: \(\nabla \times \mathbf{B} = \mu_0 \mathbf{J}\).
\end{itemize}
Equations \textbf{1A}, \textbf{1B} are equations of electrostatics and equations \textbf{2A}, \textbf{2B} are equations of magnetostatics.

Strictly, electrostatics means that
\begin{enumerate*}
    \item time-independent;
    \item charge at rest (\(\mathbf{J} = 0\));
    \item \(\mathbf{B} = 0\) ;
\end{enumerate*}
\subsection{Scalar Potential}
Equation \textbf{1B} means that the field is curl free, thus conservative. So
\[
    \mathbf{E} = -\nabla\Phi
\]
where \(\Phi(\mathbf{x} )\) is called the scalar potential.

Consider a test body in a time-independent \(\mathbf{E} \)-field. Equation of motion is just
\[
    m\ddot{\mathbf{x} } = \mathbf{F}  = q\mathbf{E} =-q \nabla  \Phi.
\]
And we have
\[
    0 = m\dot{\mathbf{x} }\cdot \ddot{\mathbf{x} } + q \dot{\mathbf{x} }\cdot \nabla \Phi = \frac{\mathrm{d}E}{\mathrm{d}t}
\]
where \(E = \) 