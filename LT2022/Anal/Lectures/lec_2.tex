\lecture{2}{24 Jan. 11:00}{Bolzano–Weierstrass theorem}
\leavevmode
\begin{theorem}{Bolzano-Weierstrass Theorem}
If \(x_n \in R\) and there exists \(K\) such that \(\abs{x_n} \leq  K\) for all \(n\), then we can find \(n_1<n_2<n_3<\dots\) and \(x\in \mathbb{R}\) such that \(x_{n_j} \to x\) as \(j \to\infty\).
In other words, every bounded sequence has a convergent subsequence.
\end{theorem}
\begin{remark}
    We say nothing about the uniqueness of the limit \(x\).

    For example, \(x_n = (-1)^n\) has two subsequences tending to \(-1\) and \(1\) respectively.
\end{remark}
\begin{proof}
    Set \([a_1,b_1] = [-K,K]\). Let \(c\) be the mid-point of \(a_1, b_1\), consider the following alternatives,
    \begin{enumerate}
        \item \(x_n \in [a_1,c]\) for infinitely many \(n\).
        \item \(x_n \in [c,a_2]\) for infinitely many \(n\).
    \end{enumerate}
    Note that \((1)\) and \((2)\) can hold at the same time. But if \((1)\) holds, we set \(a_2 = a_1\) and \(b_2 = c\). If \((1)\) fails, we have that \((2)\) must hold, and we set \(a_2 = c\) and \(b_2 = b_1\).

    We proceed as above to construct sequences \(a_n, b_n\) such that \(x_m \in [a_n, b_n]\) for infinitely many values of \(m\). They also satisfy
    \[
        a_{n-1}\leq a_n\leq b_n\leq b_{n-1},\quad b_n - a_n = \frac{b_{n-1}-a_{n-1}}{2}.
    \]
    \(a_n\) is an increasing sequence and bounded, and \(b_n\) is a decreasing sequence and bounded. By Fundamental Axiom,
    \(a_n \to a \in [a_1,b_1], b_n \to b \in [a_1,b_1]\). Using \cref{le:limprop}, \(b - a = \frac{b - a}{2}\implies a=b\).

    Since \(x_m \in [a_n, b_n]\) for infinitely many values of \(m\), having chosen \(n_j\) such that \(x_{n_j}\in [a_j,b_j]\), that is \(n_{j+1}>n_j\) such that \(x_{n_{j+1}} \in [a_{j+1},b_{j+1}]\). In other words, there is unlimited supply.

    Hence, \(a_j\leq x_{n_{j}}\leq b_j\), so \(x_{n_j}\to a\).
\end{proof}
\subsection{Cauchy Sequences}
\leavevmode
\begin{definition}{Cauchy Sequence}{}
    A sequence \(a_n \in \mathbb{R}\) is called a \textit{Cauchy sequence} if given \(\epsilon > 0~\exists N > 0\) such that \(\abs{a_n - a_m} <\epsilon ~\forall n,m > N\).
\end{definition}
\begin{note}
    \(N\) is dependent on \(\epsilon\).
\end{note}
A function is Cauchy if after you wait long enough, any two elements in the sequence would be close enough.
\begin{lemma}{}{}
    A convergent sequence is a Cauchy sequence.
\end{lemma}
\begin{proof}
    If \(a_n \to a\), given \(\epsilon > 0\), exists \(N\) such that for all \(n \geq  N\), \(\abs{a_n - a} < \epsilon\).

    Take \(m, n \geq N\),
    \[
        \abs{a_n - a_m} \leq \abs{a_n - a} + \abs{a_m - a} < 2\epsilon.
    \]
\end{proof}
\begin{lemma}{}{}
    Every Cauchy sequence is convergent.
\end{lemma}
\begin{proof}
    First we note that if \(a_n\) is Cauchy, then it is bounded.

    Take \(\epsilon = 1\), \(N = N(1)\) in the Cauchy property, then
    \[
        \abs{a_n - a_m} < 1, \quad n, m \geq N(1).
    \]
    We have
    \[
        \abs{a_m} \leq \abs{a_m - a_N} + \abs{a_N} < 1 + \abs{a_N} \quad \forall m \geq N.
    \]
    Let \(K = \mathop{\max}\{1 + \abs{a_N}, \abs{a_n}~n=1,2 \ldots ,N-1\}\).

    Then \(\abs{a_n} \leq K\) for all n. By the Bolzano–Weierstrass theorem, \(a_{n_j} \to a\). We must have \(a_n \to a\).

    Given \(\epsilon>0\), there exists \(j_0\) such that for all \(j \geq j_0\), \(\abs{a_{n_j}-a} < \epsilon\).

    Also, there exists \(N(\epsilon)\) such that \(\abs{a_m - a_n} < \epsilon\) for all \(m, n \geq N(\epsilon)\).

    Take \(j\) such that \(n_j\geq \max \{N(\epsilon),n_{j_0}\}\). Then if \(n \geq N(\epsilon)\),
    \[
        \abs{a_n - a} \leq \abs{a_n - a_{n_j}} + \abs{a_{n_j} - a} < 2\epsilon.
    \]
\end{proof}
Thus, on \(\mathbb{R}\), a sequence is convergent if and only if it is Cauchy.

The old fashion name of this is called the "general principle of convergence".

It is a useful property because we don't need what the limit actually is.
\section{Series}
\begin{definition}{}{}
    If \(a_n \in \mathbb{R},\mathbb{C}\) We say that \(\sum\limits_{j=1}^{\infty} a_j\) converges to \(s\) if the sequence of partial sums
    \[
        S_{N} = \sum\limits_{j=1}^{N} a_j \to S
    \]
    as \(N \to \infty\). We write \(\sum\limits_{j=1}^{\infty} a_j = s\). If \(S_N\) does not converge, we say that \(\sum\limits_{j=1}^{\infty} a_j\) \textit{diverges}.
\end{definition}
\begin{remark}
    Any problem on series is really a problem about the sequence of partial sums.
\end{remark}
\begin{lemma}{}{}
    \leavevmode
    \begin{enumerate}
        \item If \(\sum\limits_{j=1}^{\infty} a_j\) and \(\sum\limits_{j=1}^{\infty} a_j\) converges, then so does \(\sum\limits_{j=1}^{\infty} \lambda a_j + \mu b_j\), when \(\lambda,\mu \in \mathbb{C}\);
        \item Suppose there exists \(N\) such that \(a_i = b_i\) for all \(i \geq  N\). Then either \(\sum\limits_{i=1}^{\infty} a_i\) and \(\sum\limits_{i=1}^{\infty} b_i\) both converge or they both diverge. (initial terms do not matter for convergence)
    \end{enumerate}
\end{lemma}
\begin{proof}
    \begin{enumerate}
        \item Exercise.
        \item If we have \( n\geq N\),
        \begin{align*}
            S_n &=\sum\limits_{i=1}^{N-1} a_i + \sum\limits_{i=N}^{n} a_i\\
            d_n &=\sum\limits_{i=1}^{N-1} b_i + \sum\limits_{i=N}^{n} b_i
        \end{align*}
        So \(S_n - d_n = \sum\limits_{i=1}^{N-1} a_i - b_i\) which is a constant. So \(S_n\) converges if and only if \(d_n\) does.
    \end{enumerate}
\end{proof}