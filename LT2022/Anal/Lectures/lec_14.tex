\lecture{14}{21 Feb. 2022}{}
\section{Power Series}
We want to look at
\[
    \sum\limits_{n=0}^{\infty} a_n z^n,
\]
with \(z \in \mathbb{C}, a_n \in \mathbb{C}\). The case \(\sum\limits_{n=0}^{\infty} a_n (z - z_0)^n\), with \(z\) fixed can be reduced to power series around 0 by translation.
\begin{lemma}{}{smallerrad}
    If \(\sum\limits_{n=0}^{\infty} a_n z_1^n\) converges and \(\abs{z} < \abs{z_1}\), then \(\sum\limits_{n=0}^{\infty} a_n z^n\) converges absolutely.
\end{lemma}
\begin{proof}
    Since \(\sum\limits_{n=0}^{\infty} a_n z_1^n\) converges, \(a_n z_1^n \to 0\). Thus, there exists \( K > 0\) such that \(\abs{a_n z_1^n} \leq K\) for all \(n\).

    Then,
    \[
        \abs{a_n z^n} = \abs{a_n z^n} \frac{\abs{z_1^n}}{\abs{z_1^n}} \leq K \abs{\frac{z}{z_1}}^n.
    \]
    Since the geometric series \(\sum\limits_{n=0}^{\infty} \abs{\frac{z}{z_1}}^n \) converges, the lemma follows by comparison.
\end{proof}
Using this lemma, we will prove that every power series has a \textit{radius of convergence}.
\begin{theorem}{}{thradius}
    A power series either
    \begin{enumerate}
        \item converges absolutely for all \(z\), or
        \item converges absolutely for all \(z\) inside a circle \(\abs{z} = R\) and diverges for all \(z\) outsider it, or
        \item converges for \(z = 0\) only.
    \end{enumerate}
\end{theorem}
\begin{definition}{}{}
    The circle \(\abs{z} = R\) is called the \textit{circle of convergence} and \(R\) the \textit{radius of convergence}.

    In (1) of \cref{th:thradius}, we agree that \(R = \infty\), and in (3) \(R = 0\), so \(R \in [0, \infty]\).
\end{definition}
\begin{proof}
    Let \(S = \{x \in \mathbb{R}\mid x \geq 0, \sum a_n x^n\text{ converges}\}\). Clearly \(0 \in S\). By \eqref{le:smallerrad}, if \(x_1 \in S\), then \([0, x_1] \subseteq S\). If \(S = [0, \infty)\), we have case 1.

    Otherwise, there exists a finite supremum for \(S\). \(R = \sup S < \infty, R \geq 0\). If \(R > 0\), we'll prove that if \(\abs{z_1} < R\), then \(\sum a_n z_1^n\) converges absolutely. Pick \(R_0\) such that \(\abs{z_1} < R_0 < R\), then \(R_0 \in S\) and the series converges for \(z = R_0\). By \eqref{le:smallerrad}, \(\sum \abs{a_{n}z_1^n}\) converges.

    Finally, we show that if \(\abs{z_2} > R\), then the series does not converge for \(z_2\). Pick \(R < R_0 < \abs{z_2} \). If \(\sum a_n z_2^n\) converges, by \eqref{le:smallerrad}, \(\sum a_n R_0^n\) would be convergent, which contradicts that \(R = \sup S\).
\end{proof}
The following lemma is useful for computing \(R\).
\begin{lemma}{}{}
    If \(\abs{\frac{a_{n+1}}{a_n}}\to \ell \), as \(n \to \infty\), then \(R = \frac{1}{\ell}\).
\end{lemma}
\begin{proof}
    By the ratio test, we have absolute convergence if
    \[
        \lim\limits_{n \to \infty} \abs{\frac{a_{n+1}}{a_n}\frac{z^{n+1}}{z^n}} = \ell \abs{z} < 1.
    \]
    So if \(\abs{z} < \frac{1}{\ell}\), we have absolute convergence. If \(\abs{z} > \frac{1}{\ell}\), the series diverges, again by the ratio test.
\end{proof}
\begin{remark}
    One can also use the Root Test to get that if \(\abs{a_n}^{\frac{1}{n}} \to \ell\), then \(R = \frac{1}{\ell}\).
\end{remark}
\begin{example}
    \leavevmode
    \begin{enumerate}
        \item \(\sum\limits_{n=0}^{\infty} \frac{z^n}{n!}\).
        
        We have
        \[
            \abs{\frac{a_{n+1}}{a_n}} = \frac{n!}{(n+1)!} = \frac{1}{n + 1} \to 0 = \ell.
        \]
        So \(R = \infty \).
        \item Geometric Series, \(\sum\limits_{n=0}^{\infty} z^n\).
        
        By ratio test, we have \(R = 1\). Note that at \(\abs{z} = 1\), we have divergence.
        \item \(\sum\limits_{n=0}^{\infty} n!z^n\).

        We have
        \[
            \abs{\frac{a_{n+1}}{a_n}} = \frac{(n+1)!}{n!} = n + 1 \to \infty.
        \]
        So \(R = 0\).
        \item \(\sum\limits_{n=1}^{\infty} \frac{z^n}{n}\), and we have \(R = 1\). For \(z = 1\), it diverges. What happens for \(\abs{z} = 1\) and \(z \neq 1\)?
        
        Consider \(\sum\limits_{n=1}^{\infty} \frac{z^n}{n}(1-z)\), we have the partial sum
        \begin{align*}
            S_N &= \sum\limits_{n=1}^{N} (\frac{z^n - z^{n+1}}{n})\\
            &= \sum\limits_{n=1}^{N} \frac{z^n}{n} - \sum\limits_{n=1}^{\infty} \frac{z^{n+1}}{n}\\
            &= \sum\limits_{n=1}^{N} \frac{z^n}{n} - \sum\limits_{n=2}^{N+1} \frac{z^n}{n - 1}\\
            &= z - \frac{z^{N + 1}}{N} + \sum\limits_{n=2}^{N} z^n (\frac{1}{n} - \frac{1}{n - 1})\\
            &= z - \frac{z^{N + 1}}{N} + \sum\limits_{n=2}^{N} -\frac{z^n}{n(n-1)}.
        \end{align*}
        If \(\abs{z} = 1\), \(\frac{z^{N + 1}}{N} \to 0\) as \(N \to \infty \) and \(\sum \frac{1}{n(n - 1)}\) converges, so \(S_N\) converges.
        \item \(\sum\limits_{n=1}^{\infty} \frac{z^n}{n^2}\), and we have \(R = 1\). But it converges for all \(z\) with \(\abs{z} = 1\).
    \end{enumerate}
\end{example}
Conclusion is that, in principle, nothing can be said about \(\abs{z} = R\) and each case has to be discussed separately.

Within the radius of convergence, ``life is great''. Power series behave as if ``they were polynomials''.