\lecture{1}{21 Jan. 11:00}{Limits}
Books:
\begin{itemize}
    \item \textit{A First Course in Mathematical Analysis} -Burkill
    \item \textit{Calculus} -Spivak
    \item \textit{Analysis I} -Tao
\end{itemize}
\section{Limits and Convergence}
\subsection{Review from Numbers and Sets}
\begin{notation}
    We denote sequences by \(a_n\) or \((a_n)^\infty_{n=1}\), with \(a_n \in \mathbb{R}\).
\end{notation}
\begin{definition}{}{}
    We say that \(a_n \to a\) as \(n \to \infty\) if given \(\epsilon > 0\), there exists \(N\) such that \(\abs{a_{n}-a} < \epsilon \) for all \(n \geq N\).
\end{definition}
\begin{note}
    \(N = N(\epsilon)\) which is dependent on \(\epsilon\). That is, if you want to go closer to \(a\), sometimes you need to go higher in \(N\).
\end{note}
\begin{definition}{limit of a sequence}{limseq}
    We say that a sequence is a
    \begin{align*}
        \begin{rcases}
            \textit{increasing sequence if }a_n \leq a_{n+1},\\
            \textit{decreasing sequence if } a_n \geq a_{n+1},
        \end{rcases}&\textit{monotone sequence}\\
        \begin{rcases}
            \textit{strictly increasing sequence if }a_n \leq a_{n+1},\\
            \textit{strictly decreasing sequence if } a_n \geq a_{n+1}.
        \end{rcases}&\textit{strictly monotone sequence}
    \end{align*}
\end{definition}
We also have
\begin{theorem}{Fundamental Axiom of the Real Numbers}{lubp}
    If \(a_n \in \mathbb{R}\) and \(a_n\) is increasing and bounded above by \(A \in R\), then there exists \(a \in \mathbb{R}\) such that \(a_n \to n\) as \(n \to \infty \).

    That is, an increasing sequence of real numbers bounded above \textit{converges}.
\end{theorem}
\begin{remark}
    It is equivalent to the following,
    \begin{itemize}
        \item A decreasing sequence of real numbers bounded below converges.
        \item Every non-empty set of real numbers bounded above has a \textit{supremum} (Least Upper Bound Axiom).
    \end{itemize}
\end{remark}
\begin{definition}{supremum}{supremum}
    For \(S \subseteq \mathbb{R},S\neq \varnothing\). We say that \(\sup S = k\) if 
    \begin{enumerate}
        \item \(x \leq k, \quad \forall x \in S\),
        \item given \(\epsilon > 0\), there exists \(x \in S\) such that \(x > k - \epsilon\).
    \end{enumerate}
\end{definition}
\begin{note}
    Supremum is unique, and there is a similar notion of infimum.
\end{note}
\begin{lemma}{Properties of Limits}{limprop}
    \leavevmode
    \begin{enumerate}
        \item The limit is unique. That is, if \(a_n \to a\), and \(a_n \to b\), then \(a = b\).
        \item If \(a_n \to a\) as \(n \to \infty\) and \(n_1 < n_2 < n_3 \ldots \), then \(a_{n_j}\to a\) as \(j \to \infty\) (subsequences converge to the same limit).
        \item If \(a_n = c\) for all \(n\) then \(a_n \to c\) as \(n \to \infty\).
        \item If \(a_n \to a\) and \(b_n \to b\), then \(a_n + b_n \to  a+ b\).
        \item If \(a_n \to a\) and \(b_n \to b\), then \(a_n b_n \to  ab\).
        \item If \(a_n \to a\) , then \(\frac{1}{a_n} \to \frac{1}{a}\).
        \item If \(a_n < A\) for all \(n\) and \(a_n \to a\), then \(a \leq A\).
    \end{enumerate}
\end{lemma}
\begin{proof}
    \leavevmode
    \begin{enumerate}
        \item[1.] Given \(\epsilon > 0\), there exists \(N_1\) such that \(\abs{a_n - a} < \epsilon, \forall n \geq N_1\), and there exists \(N_2\) such that \(\abs{a_n - b} < \epsilon, \forall n \geq N_2\).

        Take \(N = \max\{n_1, n_2\}\), then if \(n \geq N\),
        \[
            \abs{a - b} \leq \abs{a_n - a} +\abs{a_n -b} < 2\epsilon.
        \]
        If \(a \neq b\), take \(\epsilon = \frac{\abs{a - b}}{3}\), we have
        \[
            \abs{a - b} < \frac{2}{3}\abs{a - b}. \contra
        \]
        \item[2.] Given \(\epsilon > 0\), there exists \(N\) such that \(\abs{a_n - a} < \epsilon, \forall n \geq N\), Since \(n_j \geq j\), we know

        \[
            \abs{a_{n_j} - a} < \epsilon, \forall j \geq  N.
        \]
        That is, \(a_{n_j} \to a\) as \(j \to \infty\).
        \item[5.] We have
        \begin{align*}
            \abs{a_n b_n - ab} &\leq \abs{a_n b_n - a_n b} + \abs{a_n b - ab}\\
            &= \abs{a_n} \abs{b_n - b} + \abs{b} \abs{a_n - a}.
        \end{align*}
        Given \(\epsilon > 0\), there exists \(N_1\) such that \(\abs{a_n - a} < \epsilon, \forall n \geq N_1\), and there exists \(N_2\) such that \(\abs{b_n - b} < \epsilon, \forall n \geq N_2\).

        If \(n \geq N_1(1)\), \(\abs{a_n - a} < 1 \), so \(\abs{a_n} \leq \abs{a} + 1\).

        We have
        \[
            \abs{a_n b_n - ab} \leq \epsilon(\abs{a} +1+\abs{b}), \forall n \geq N_3(\epsilon) = \max\{N_1(1),N_1(\epsilon),N_2(\epsilon)\}.
        \]
    \end{enumerate}
\end{proof}
\begin{lemma}{}{}
    \[
        \frac{1}{n} \to 0 \text{ as } n\to \infty.
    \]
\end{lemma}
\begin{proof}
    \(\frac{1}{n}\) is a decreasing sequence that is bounded below. By the Fundamental Axiom, it has a limit \(a\).
    
    We claim that \(a = 0\). We have
    \[
        \frac{1}{2n}=\frac{1}{2}\times \frac{1}{n}\to \frac{a}{2} \text{ by \cref{le:limprop}}.
    \]

    But \(\frac{1}{2n}\) is a subsequence, so by \cref{le:limprop} \(\frac{1}{2n}\to a\). By uniqueness of limits proved again in \cref{le:limprop}, we have \(a = \frac{a}{2} \implies a = 0\).
\end{proof}
\begin{remark}
    The definition of limit of a sequence makes perfect sense for \(a_n \in \mathbb{C}\) by replacing the absolute value with modulus.

    \begin{definition}{}{}
    We say that \(a_n \to a\) as \(n \to \infty\) if given \(\epsilon > 0\), there exists \(N\) such that \(\abs{a_{n}-a} < \epsilon \) for all \(n \geq N\).
    \end{definition}

    And the first six parts of \cref{le:limprop} are the same over \(\mathbb{C}\). The last one does not make sense over \(\mathbb{C}\) since it uses the order of \(\mathbb{R}\).
\end{remark}