\lecture{23}{14 Mar. 2022}{}
\subsection{Improper Integrals}
\begin{definition}[breakable]{}{}
    Suppose \(f:[a,\infty) \to \mathbb{R}\) is integrable (and bounded) on every interval \([a,R]\) and that as \(R \to \infty\),
    \[
        \int^R_a f(x) \dif x \to \ell.
    \]
    Then we say that \(\int^{\infty}_a f(x) \dif x\) \textit{exists} or \textit{converges} and that its value is \(\ell\). If \(\int^R_a f(x) \dif x\) does not tend to a limit, we say that \(\int^\infty_a f(x) \dif x\) \textit{diverges}.

    A similar definition applies to \(\int^a_{-\infty} f(x) \dif x\)

    If \(\int^\infty_a f =\ell_1\) and \(\int^a_{-\infty} f = \ell_2\), we write \(\int^\infty_{-\infty} f = \ell_1 + \ell_2\). (independent of the particular value of \(a\))
\end{definition}
\begin{remark}
    The last part is not the same as saying that \(\lim_{R \to \infty}\int^R_{-R}f(x) \dif x\) exists. It is stronger since \(\int^R_{-R} x \dif x = 0\).
\end{remark}
\begin{example}
    The integral \(\int^{\infty}_1 \frac{\dif x}{x^k}\) converges if and only if \(k > 1\).

    Indeed, if \(k\neq 1\),
    \[
        \int^R_1 \frac{\dif x}{x^k} = \eval{\frac{x^{1-k}}{1-k}}^R_1 = \frac{R^{1-k}-1}{1-k}
    \]
    and as \(R \to \infty\), this limit is finite if and only if \(k > 1\). If \(k = 1\),
    \[
        \int^R_1 \frac{\dif x}{x} = \log R \to \infty.
    \]
\end{example}
\begin{remark}
    \begin{enumerate}
        \item \(\nicefrac{1}{\sqrt{x}}\) is continuous on \([\delta, 1]\) for any \(\delta > 0\), and
        \[
            \int^1_\delta \frac{\dif x}{\sqrt{x}} = \eval{2\sqrt{x}}^1_\delta = 2 - 2\sqrt{\delta} \to 2
        \]
        as \(\delta \to 0\). The function is unbounded on \((0,1]\), but the limit of the integral exists.
        
        However,
        \[
            \int^1_0 \frac{\dif x}{x} = \lim_{\delta \to 0} \int^1_\delta \frac{\dif x}{x} = \lim_{\delta \to 0} \eval{\log x}^1_\delta = \lim_{\delta \to 0} (\log 1 - \log \delta)
        \]
        does not exist.
        \item If \(f \geq 0\) and \(g \geq 0\) for \(x \geq a\), and
        \[
            f(x) \leq Kg(x). \qquad \forall x\geq a
        \]
        Then \(\int^{\infty}_a g\) converges implies \(\int^\infty_a f\) converges and \(\int^\infty_a f \leq K \int^\infty_a g\).

        Just note that \(\int^R_a f \leq K \int^R_a g\). The function \(R \mapsto \int^R_a f\) is increasing because \(f \geq 0\), and bounded above because \(\int^\infty_a g\) converges. Take \(\ell = \sup_{R\geq a} \int^R_a f < \infty\). Now check that
        \[
            \lim_{R \to \infty} \int^R_a f = \ell.
        \]
        Given \(\epsilon>0\), there exists \(R_0\) such that
        \[
            \int^{R_o}_a f \geq \ell - \epsilon.
        \]
        Thus, if \(R \geq R_0\),
        \[
            \int^R_a f \geq \int^{R_0}_a f \geq \ell-\epsilon.
        \]
        And the limit follows.
        \begin{example}
            Consider the integral \(\int^\infty_0 e^{\nicefrac{-x^2}{2}} \dif x\). We have
            \[
                e^{-\nicefrac{x^2}{2}} \leq e^{-\nicefrac{x}{2}}. \qquad x \geq 1
            \]
            Since
            \[
                \int_1^R e^{-\nicefrac{x}{2}} \dif x = \frac{1}{2}(e^{-\nicefrac{1}{2}} - e^{-\nicefrac{R}{2}}) \to \frac{e^{-\nicefrac{1}{2}}}{2},
            \]
            the integral \(\int^\infty_0 e^{-\nicefrac{x^2}{2}}\) converges.
        \end{example}
        \item We know that if \(\sum a_n\) converges, then \(a_n \to 0\). However, \(\int^\infty_a f\) converges may not imply that \(f \to 0\).

        For example, consider a function of smaller and smaller bumps to the highest point with value \(1\).
    \end{enumerate}
\end{remark}
\begin{theorem}{Integral test}{inttest}
    Let \(f(x)\) be a positive decreasing function for \(x \geq 1\). Then
    \begin{enumerate}
        \item the integral \(\int^\infty_1 f(x) \dif x\) and the series \(\sum^\infty_1 f(n)\) both converge or diverge;
        \item As \(n \to \infty\), \(\sum_{r=1}^{n} f(r) - \int_{1}^{n} f(x) \dif x\) tends to a limit \(\ell\) such that \(0 \leq \ell \leq f(1)\).
    \end{enumerate}
\end{theorem}
\begin{proof}
    Note that \(f\) deceasing implies that \(f\) is integrable on every bounded subinterval by \cref{th:monoint}.

    If \(n  - 1 \leq x \leq n\), then \(f(n-1) \geq f(x) \geq f(n)\), so
    \[
        f(n - 1) \geq \int^n_{n-1}f(x) \dif x \geq f(n).
    \]
    Adding the terms together, we have
    \[
        \sum_1^{n-1} f(r) \geq \int^n_1 f(x) \dif x \geq \sum^n_2 f(r).
    \]
    From this, claim (1) is clear. For the proof of (2), set \(\phi(n) = \sum^n_1 f(r) - \int^n_1 f(x) \dif x\), then
    \[
        \phi(n) - \phi(n-1) = f(n) - \int^n_{n-1} f(x) \dif x\leq 0.
    \]
    So \(\phi(n)\) is decreasing. We also have \(0 < \phi(n) \leq f(1)\). Thus, \(\phi(n)\) tends to a limit \(\ell\) such that \(0 \leq \ell < f(1)\).
\end{proof}