\lecture{4}{28 Jan. 2022}{}
\begin{example}
    To determine the convergence of \(\sum\limits_{n=1}^{\infty} a_n = \frac{n}{2^n}\).

    By ratio test,
    \[
        \frac{n + 1}{2^n}\frac{2^n}{n} = \frac{n + 1}{2n}\to \frac{1}{2}<1.
    \]
    So we have convergence by ratio test.

    However, \(\sum\limits_{n=1}^{\infty} \frac{1}{n}\) diverges, and ratio test gives limit 1, and \(\sum\limits_{n=1}^{\infty} \frac{1}{n^2}\) converges, and ratio test gives limit 1. So ratio test is inconclusive if the limit is 1.

    Since \(n^{\frac{1}{n}} \to 1\) as \(n \to \infty\), so root test is also inconclusive when the limit is 1.

    To see this limit, write
    \[
        n^{\frac{1}{n}} = 1 + \delta_n, ~\delta_n > 0.
    \]
    So
    \[
        n = (1 + \delta_n)^n > \frac{n(n-1)}{2}\delta_n^2.
    \]
    And \(\delta_n^2 < \frac{2}{n - 1}\implies \delta_n \to 0\).
\end{example}
\begin{remark}
    Use the root test when there is a nth power in the series.
\end{remark}
\begin{theorem}{Cauchy's Condensation Test}{}
    Let \(a_n\) be a decreasing sequence of positive terms. Then \(\sum\limits_{n=1}^{\infty} a_n\) converges if and only if \(\sum\limits_{n=1}^{\infty} 2^n a_{2^n}\) converges.
\end{theorem}
\begin{proof}
    First we observe that if \(a_n\) is decreasing
    \[a_{2^k} \leq a_{2^{k-1} + i}\leq a_{2^{k-1}}\] 
    for all \(k \geq 1\) and \(1 \leq i\leq 2^{k-1}\).

    Assume that \(\sum\limits_{n=1}^{\infty} a_n\) converges with sum \(A\). Then
    \begin{align*}
    2^{n - 1}a_{2^n} &= \underbrace{a_{2^n} + \cdots a_{2^n}}_{2^{n-1} \text{ times}}\\
    &\leq a_{2^{n -1 } + 1} + \cdots + a_{2^n}\\
    &= \sum\limits_{m=2^{n-1}+1}^{2^n} a_m.
    \end{align*}
    Thus, \(\sum\limits_{n=1}^{N} 2^{n-1}a_{2^n}\leq \sum\limits_{n=1}^{N} \sum\limits_{m=2^{n-1}+1}^{2^n} a_m = \sum\limits_{m=2}^{2^N} a_m\). So
    \[
        \sum\limits_{n=1}^{N} 2^{n}a_{2^n} \leq 2 \sum\limits_{m=2}^{2^N} a_m \leq 2(A-a_1).
    \]
    Thus, \(\sum\limits_{n=1}^{N} 2^n a_{2^n}\) being increasing and bounded above, converges.

    Conversely, assume \(\sum\limits_{n=1}^{\infty} 2^n a_{2^n}\) converges to \(B\), then
    \begin{align*}
        \sum\limits_{m=2^{n - 1}+1}^{2^n} a_m &= a_{2^{n-1} + 1} + a_{2^{n-1}+2} + \cdots + a_{2^n}\\
        &\leq \underbrace{a_{2^{n-1}} + \cdots + a_{2^{n-1}}}_{2^{n-1} \text times} = 2^{n-1}a_{2^{n-1}}.
    \end{align*}
    Similarly, we have
    \[
        \sum\limits_{m=2}^{2^N} a_m = \sum\limits_{n=1}^{N} \sum\limits_{m=2^{n-1}+1}^{2^n} a_m \leq \sum\limits_{n=1}^{N} 2^{n-1}a_{2^{n-1}} \leq B.
    \]
    Therefore, \(\sum\limits_{m=1}^{N} a_m\) is a bounded increasing sequence and thus it converges.
\end{proof}
\begin{example}
    \(\sum\limits_{n=1}^{\infty} \frac{1}{n^k}\) for \(k > 0\) converges if and only if \(k > 1\). First we note that \(\frac{1}{n^k}\) is a decreasing sequence of positive terms.
    \[
        \frac{1}{(n + 1)k}<\frac{1}{n^k} \iff (\frac{n}{n+1})^k<1 \iff \frac{n}{n+1}<1.
    \]
    We use Cauchy condensation test, and we have
    \begin{align*}
        2^n a_{2^n} &= 2^{n}\left(\frac{1}{2^n}\right)^k\\
        &= 2^{n - nk} = (2^{1-k})^n.
    \end{align*}
    Which is a geometric series with the ratio \(2^{1-k}\). So \(\sum\limits_{n=1}^{\infty} \frac{1}{n^k}\) converges if and only if \(2^{1-k} < 1 \iff k > 1\).
\end{example}
\subsection{Alternating Series}
\leavevmode
\begin{theorem}{Alternating Series Test}{}
    If \(a_n\) decreases and tends to 0 as \(n \to \infty\), then the series \(\sum\limits_{n=1}^{\infty} (-1)^{n+1}a_n\) converges.
\end{theorem}
\begin{example}
    \(\sum\limits_{n=1}^{\infty} \frac{(-1)^{n+1}}{n}\) converges.
\end{example}
\begin{proof}
    The partial sum is
    \begin{align*}
        S_n &= a_1 - a_2 + \cdots + (-1)^{n+1}a_n\\
        S_{2n} &= (a_1 - a_2) + (a_3 - a_4) + \cdots + (a_{2n - 1} - a_{2n}) \geq S_{2n - 1}\\
        S_{2n} &= a_1 - (a_2 - a_3) - (a_4 - a_5) - \cdots - (a_{2n - 2} - a_{2n-1}) - a_{2n} \leq a_1
    \end{align*}
    So \(S_{2n}\) is increasing and bounded above, implying that \(S_{2n}\to S\). The odd terms satisfy
    \[
        S_{2n + 1} = S_{2n} + a_{2n + 1} \to S + 0 = S.
    \]
    This implies that \(S_n\) converges to \(S\) as well. Given \(\epsilon\), there exists \(N_1\) such that \(\forall n \geq N_1\), \(\abs{S_{2n}-S} < \epsilon\). We also know that there exists \(N_2\) such that \(\forall n \geq N_2\), \(\abs{S_{2n + 1} - S} <\epsilon\). Take \(N = 2 \mathop{\max} \{N_1, N_2\} + 1\), then if \(n \geq N\), \(\abs{S_k - S} < \epsilon\). So \(S_k \to S\).
\end{proof}