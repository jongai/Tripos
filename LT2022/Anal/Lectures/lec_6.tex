\lecture{6}{2 Feb. 2022}{}
\section{Functions}
\subsection{Continuity}
Suppose \(E \subseteq \mathbb{C}\) is a non-empty subset, and we have a function \(f: E \to \mathbb{C}\)and a point \(a \in E\). (this includes the case in which \(f\) is real-valued and \(E\) is a subset of \(\mathbb{R}\))
\begin{definition}{}{}
    \(f\) is \textit{continuous}  at \(a \in E\) if for every sequence \(z_n \in E\) with \(z_n \to a\), we have \(f(z_n) \to f(a)\).
\end{definition}
\begin{definition}{\(\epsilon\)-\(\delta\) Definition}{}
    \(f\) is \textit{continuous} at \(a \in E\), if given \(\epsilon > 0\), \(\exists \delta>0\) such that if \(\abs{z-a} <\delta, z \in E\), then \(\abs{f(z) - f(a)} < \epsilon\).
\end{definition}
We prove right away that the two definitions are equivalent.
\begin{theorem}{}{}
    The two definitions of continuity are equivalent.
\end{theorem}
\begin{proof}
    We first prove the second definition implies the first definition. We know that given \(\epsilon>0, \exists \delta>0\) such that \(\abs{z - a} < \delta, z \in E\), then \(\abs{f(z) - f(z)} < \epsilon\). Let \(z_n \to a\), then \(\exists n_0\) such that \(\forall n \geq n_0\), we have \(\abs{z_n - a} < \delta\). This implies, by the assumption, \(\abs{f(z_n) - f(a)} < \epsilon\). That is, \(f(z_n) \to f(a)\).

    Next, we prove the other direction. Assume \(f(z_n)\to f(a)\) whenever \(z_n \to a, z_n \in E\). Suppose \(f\) is not continuous at \(a\) according to Definition 2.

    \(\exists \epsilon>0\), s.t. \(\forall \delta>0\), there exists \(z\in E\) s.t. \(\abs{z-a} < \delta\) and \(\abs{f(z) -f(a)} \geq \epsilon\).

    Let \(\delta=\frac{1}{n}\) from non-continuity defined above, we get \(z_n\) such that \(\abs{z_n - a} <\frac{1}{n}\) and \(\abs{f(z_n) - f(a)} \geq \epsilon\). Clearly \(z_n \to a\), but \(f(z_n)\) does not tend to \(f(a)\) because \(\abs{f(z_n) - f(a)} \geq \epsilon\). Contradiction.
\end{proof}
\begin{proposition}{}{}
    \label{contprop}
    \(a \in E\), and \(g, f: E \to \mathbb{C}\) are both continuous at \(a\). So are the functions \(f(z) + g(z)\), \(f(z)g(z)\) and \(\lambda f(z)\) for any constant \(\lambda\). In addition, if \(f(z)\neq 0~\forall z \in E\), then \(\frac{1}{f(z)}\) is continuous at \(a\).
\end{proposition}
\begin{proof}
    Using Definition 1 of continuity, this is obvious, using the analogous results for sequences. (Lemma \eqref{le:limprop})

    For example,
    \[z_n \to a \implies f(z_n) \to f(a), g(z_n) \to g(a) \implies f(z_n) + g(z_n) \to f(a) + g(a).\]
\end{proof}
The function \(f(z) = z\) is continuous, so by using the proposition, we get that every polynomial is continuous at every point in \(\mathbb{C}\).

\begin{note}
    We say that \(f\) is \textit{continuous on \(E\)} if it is continuous at every \(a \in E\).
\end{note}
\begin{remark}
    Still it is instructive to prove Proposition \eqref{contprop} directly from the \(\epsilon\)-\(\delta\) definition.
\end{remark}
Next we look at compositions.
\begin{theorem}{}{}
    Let \(f: A \to \mathbb{C}\) and \(g: B \to \mathbb{C}\) be two functions such that \(f(A) \subseteq B\). Suppose \(f\) is continuous at \(a \in A\) and \(g\) is continuous at \(f(a)\), then \(g\circ f: A \to \mathbb{C}\) is continuous at \(a\).
\end{theorem}
\begin{proof}
    Take any sequence \(z_n \to a\), by assumption we know \(f(z_n) \to f(a)\). Set \(w_n = f(z_n) \in B\). By continuity of \(g\), we have \(g(w_n) \to g(f(a))\), and we are done.
\end{proof}
\begin{example}
    \leavevmode
    \begin{enumerate}
        \item Let \(f: \mathbb{R}\to \mathbb{R}\) be
        \[
            f(x) = \begin{dcases}
                \sin \left(\frac{1}{x}\right), &\text{ if } x\neq 0\\
                0, &\text{ if } x = 0\\
            \end{dcases},
        \]
        assuming that \(\sin x\) is continuous. (to be proved later) If \(x\neq 0\), propositions proved above imply that \(f(x)\) is continuous at any \(x \neq 0\).

        However, it is discontinuous at \(0\). Consider the sequence satisfying
        \[\frac{1}{x_n} = (2n + \frac{1}{2})\pi.\]
        We have \(f(x_n) \to 1, x_n \to 0\), but \(f(0)= 0\).
        \item Let \(f: \mathbb{R}\to \mathbb{R}\) be
        \[
            f(x) = \begin{dcases}
                x \sin\left(\frac{1}{x}\right), &\text{ if } x\neq 0\\
                0, &\text{ if } x = 0\\
            \end{dcases}.
        \]
        It's continuous at \(x\neq 0\) as above, and \(f\) is continuous at \(0\). Take \(x_n \to 0\), then \(\abs{f(x_n)} \leq \abs{x_n} \) because \(\sin \frac{1}{x} \leq 1\), so \(f(x_n) \to 0 = f(0)\).
        \item Let \(f: \mathbb{R}\to \mathbb{R}\) be
        \[
            f(x) = \begin{dcases}
                1, &\text{ if } x \in \mathbb{Q}\\
                0, &\text{ if } x \notin \mathbb{Q}\\
            \end{dcases}.
        \]
        It is discontinuous at every point. If \(x \in \mathbb{Q}\), take a sequence \(x_n \to x\) with \(x_n \notin \mathbb{Q}\), then \(f(x_n) = 0 \not \to f(x) = 1\). Similarly, if \(x \notin \mathbb{Q}\), take \(x_n \to x\) with \(x_n \in \mathbb{Q}\), we have \(f(x_n) = 1 \not \to f(x) = 0\).
    \end{enumerate}
\end{example}