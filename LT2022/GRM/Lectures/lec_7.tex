\lecture{7}{3 Feb. 2022}{}
\leavevmode
\begin{lemma}{}{}
    If \(p\) is an odd prime
    \[
        \abs{PSL_2(\nicefrac{\mathbb{\MakeUppercase{Z}}}{p\mathbb{\MakeUppercase{Z}}})} = \frac{p(p-1)(p+1)}{2}.
    \]
\end{lemma}
\begin{proof}
    From above, \(\abs{GL_2(\nicefrac{\mathbb{\MakeUppercase{Z}} }{p\mathbb{\MakeUppercase{Z}} })} = p(p^2 - 1)(p-1)\). First we note the group homomorphism \(GL_2(\nicefrac{\mathbb{\MakeUppercase{Z}} }{p\mathbb{\MakeUppercase{Z}}})\to (\nicefrac{\mathbb{\MakeUppercase{Z}}}{p\mathbb{\MakeUppercase{Z}}})^{\times}\) by taking the determinant is surjective. Thus, \(\abs{SL_2(\nicefrac{\mathbb{\MakeUppercase{Z}} }{p\mathbb{\MakeUppercase{Z}} })} = \frac{\abs{GL_2(\nicefrac{\mathbb{\MakeUppercase{Z}}}{p\mathbb{\MakeUppercase{Z}} })}}{p - 1} = p(p - 1)(p + 1)\).
    
    And if a scalar matrix \(\begin{pmatrix}
        \lambda &  0 \\
        0 &  \lambda \\
    \end{pmatrix} \in SL_2(\nicefrac{\mathbb{\MakeUppercase{Z}} }{p \mathbb{\MakeUppercase{Z}} })\), then \(\lambda^2 \equiv 1 \mod p\), so
    \[p \mid (\lambda-1)(\lambda + 1)\implies \lambda\equiv \pm 1 \mod p.\]
    Thus, \(Z \cap SL_2(\nicefrac{\mathbb{\MakeUppercase{Z}} }{p\mathbb{\MakeUppercase{Z}} }) = \{\pm I\}\), and they are distinct since \(p > 2\). So we have
    \[\abs{PSL_2(\nicefrac{\mathbb{\MakeUppercase{Z}} }{p\mathbb{\MakeUppercase{Z}} })} = \frac{\abs{SL_2(\nicefrac{\mathbb{\MakeUppercase{Z}}}{p\mathbb{\MakeUppercase{Z}} })}}{2} = \frac{p(p - 1)(p + 1)}{2}.\]
\end{proof}
\begin{example}
    Let \(G = PSL_2(\nicefrac{\mathbb{\MakeUppercase{Z}} }{5\mathbb{\MakeUppercase{Z}} })\), then \(\abs{G} = \frac{4\cdot 5 \cdot 6}{2} = 60\).

    Let \(G\) act on \(\nicefrac{\mathbb{\MakeUppercase{Z}} }{5\mathbb{\MakeUppercase{Z}}}\cup \{\infty\}\) via the Möbius-like transformation. By Lemma \eqref{pglin}, the permutation representation \(\phi: G \to \mathrm{Sym}(\{0,1,2,3,4,\infty\})\) is injective. We claim that \(\ima(\phi)\leq A_6\); that is, \(\psi:G \to S_6\to \{\pm 1\}\) is trivial.

    Let \(h \in G\) have order \(2^n m\), \(m\) odd. If \(\psi(h^m) = 1\), then \[\psi(h)^m = 1 \implies \psi(h) = 1.\] Noting that \(h^m\) has order \(2^n\), it suffices to show that \(\psi(g) = 1\) for all \(g \in G\) or order a power of 2. By Lemma \eqref{sylowp}, every such \(g\) belongs to a Sylow 2-subgroup.

    It suffices to show that \(\psi(H) = 1\) for \(H\) a particular Sylow 2-subgroup. (since \(\ker(\psi)\) is normal and all Sylow subgroups are conjugate)

    Take \(H = \left\langle \begin{pmatrix}
        2 &  0 \\
        0 &  3 \\
    \end{pmatrix},\begin{pmatrix}
        0 &  1 \\
        -1 &  0 \\
    \end{pmatrix}\right\rangle\leq G\). We compute
    \(\phi\left(\begin{pmatrix}
        2 &  0 \\
        0 &  3 \\
    \end{pmatrix}\right)=(1~4)(2~3)\)
     and 
    \(\phi\left(\begin{pmatrix}
        0 &  1 \\
        -1 &  0 \\
    \end{pmatrix}\right)=(0~\infty)(1~4)\). Both of which are even permutations. Thus, \(\psi(H) = 1\), and this proves the claim.

    Lastly, by example sheet Q14 tells you if \(G \leq A_6\) and \(\abs{G} = 60\) then \(G \cong A_5\).
\end{example}
\begin{property}[not proved in the course]
    \leavevmode
    \begin{enumerate}
        \item \(PSL_n(\nicefrac{\mathbb{\MakeUppercase{Z}} }{p\mathbb{\MakeUppercase{Z}} })\) is a simple group for all \(n \geq 2\) and \(p\) a prime except when \((n,p) = (2,2),(2,3)\). (finite groups of Lie type)
        \item The smallest non-Abelian simple groups are \(A_5 \cong PSL_2(\nicefrac{\mathbb{\MakeUppercase{Z}}}{5\mathbb{\MakeUppercase{Z}}})\) of order 60 and \(PSL_2(\nicefrac{\mathbb{\MakeUppercase{Z}} }{7\mathbb{\MakeUppercase{Z}}})\cong GL_3(\nicefrac{\mathbb{\MakeUppercase{Z}}}{2\mathbb{\MakeUppercase{Z}} })\) of order 168.
    \end{enumerate}
\end{property}
\subsection{Finite Abelian groups}
Later in the course we will prove the following result.
\begin{theorem}{}{}
    Every finite Abelian group is isomorphic to a product of cyclic groups.
    \label{finab}
\end{theorem}
\begin{note}
    Such an isomorphism is not unique.
\end{note}
\begin{lemma}{}{}
    If \(m, n \in \mathbb{\MakeUppercase{Z}}_{\geq 1}\) coprime, then \(C_m \times C_n \cong C_{mn}\).
    \label{cyclicmn}
\end{lemma}
\begin{proof}
    Let \(g\) and \(h\) be generators of \(C_n\) and \(C_m\). Then \((g, h) \in C_m \times C_n\) and \((g, h)^r = (g^r, h^r)\). Hence, \((g, h)^r = 1 \iff m \mid r \land n \mid r \iff mn \mid r\). Thus, \((g,h)\) has order \(mn = \abs{C_m \times C_n} \). And we have \( C_m \times C_n \cong C_{mn} = \left\langle(g,h)\right\rangle \).
\end{proof}
\begin{corollary}{}{}
    Let \(G\) be a finite Abelian group, then \(G \cong C_{n_1} \times C_{n_2} \times \cdots C_{n_k}\) where \(n_i\) is a prime power.
    \label{primeab}
\end{corollary}
\begin{proof}
    If \(n = p_1^{a_1}\cdots p_r^{a_r}\) (\(p_1,\ldots,p_r\) distinct primes), then Lemma \eqref{cyclicab} shows \(C_n \cong C_{p_1^{a_1}}\times \cdots C_{p_r^{a_r}}\). Writing each of the cyclic groups in Theorem \eqref{finab} in this way gives the result.
\end{proof}
We will prove a refinement of Theorem \eqref{finab}.
\begin{theorem}{}{}
    \label{divab}
    Let \(G\) be a finite Abelian group. Then
    \[G \cong C_{d_1}\times C_{d_2} \times \cdots \times C_{d_t}\]
    for some \(d_1 \mid d_2 \mid \dots \mid d_t\).
\end{theorem}
\begin{remark}
    The integers \(n_1, \ldots, n_k\) in Corollary \eqref{primeab} (up to order) and \(d_1, \ldots, d_t\) in Theorem \eqref{divab} (assuming \(d_1 > 1\)) are uniquely determined by \(G\). (Proof omit)
\end{remark}
\begin{example}
    \leavevmode
    \begin{enumerate}
        \item The Abelian groups of order 8 are \(C_8\), \(C_2 \times C_4\), \(C_2 \times C_2 \times C_2\).
        \item The Abelian groups of order 12 are \(C_2 \times C_2 \times C_3\) and \(C_4 \times C_3\) by Theorem \eqref{primeab}, and \(C_2 \times C_6\) and \(C_{12}\) by Theorem \eqref{divab}.
    \end{enumerate}
\end{example}
\begin{definition}{}{}
    The \textit{exponent} of a group \(G\) is the least integer \(n \geq 1\) such that \(g^n = 1 ~ \forall g\in G\). That is, the \(\operatorname{lcm}\) of all the orders of the elements of \(G\).
\end{definition}
\begin{example}
    \(A_4\) has exponent 6. The exponent here is greater than the biggest order of an element of the group.
\end{example}
\begin{corollary}{}{}
    Every finite Abelian group contains an element whose order is the exponent of the group.
\end{corollary}
\begin{proof}
    If \(G \cong C_{d_1} \times \cdots \times C_{d_t}\) with \(d_1 \mid d_2 \mid \cdots \mid d_t\), then every \(g \in G\) has order dividing \(d_t\), and if \(h \in C_{d_t}\) is a generator, then \((1, \ldots, h)\in G\) has order \(d_t\). Thus, \(G\) has exponent \(d_t\).
\end{proof}