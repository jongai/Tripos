\lecture{18}{1 Mar. 2022}{}
\begin{definition}{}{}
    A ring satisfying ACC is called \textit{Noetherian}.
\end{definition}
\begin{theorem}{Hilbert's Basis Theorem}{hilbertbasis}
    If \(R\) is a Noetherian ring, then \(R[X]\) is Noetherian.
\end{theorem}
\begin{proof}
    Assume otherwise that \(J \nsub R[X]\) is not finitely generated. Choose \(f_1 \in J\) of minimal degree. Then \((f_1) \neq J\), and we can choose \(f_2 \in J \setminus (f_1)\) of minimal degree. Then \((f_1, f_2) \neq J\), and we can choose \(f_3 \in J \setminus (f_1, f_2)\) of minimal degree, and so on. We obtain a sequence \(f_1, f_2, f_3, \dots \in R[X]\) with \(\deg f_i \leq \deg f_{i+1}\). Otherwise, we could have just picked \(f_{i+1}\) instead of \(f_i\). Let \(a_i\) be the leading coefficient of \(f_i\). We obtain
    \[
        (a_1) \subseteq (a_1, a_2) \subseteq (a_1, a_2, a_3) \subseteq \dots,
    \]
    a chain of ideals. Since \(R\) is Noetherian, \(\exists m \in \mathbb{N}\) such that \(a_{m+1} \in (a_1, \dots, a_m)\). Let \(a_{m+1} = \sum_{i=1}^m \lambda_i a_i\), and set \(g = \sum_{i=1}^m \lambda_i X^{\deg f_{m+1} - \deg f_i} f_i\). Then \(\deg f_{m+1} - \deg g\), and they have the same leading coefficient \(a_{m+1}\). Then \(f_{m+1} - g \in J\), and \(\deg (f_{m+1} - g) < \deg f_{m+1}\). So \(f_{m+1} - g \in (f_1,\dots, f_m)\) for minimality of the degree of \(f_{m+1}\), so \(f_{m+1} \in (f_1,\dots,f_m)\), contradiction. Thus, \(J\) is finitely generated, and \(R[X]\) is Noetherian by \cref{le:noefinite}.
\end{proof}
\begin{corollary}{}{}
    \(\mathbb{Z}[X_1, \dots, X_n]\) is Noetherian. \(F[X_1, \dots, X_n]\) is Noetherian.
\end{corollary}
\begin{example}
    Let \(R = \mathbb{C}[X_1, \dots, X_n]\), and \(V \subseteq \mathbb{C}^n\) be a subset of the form
    \[
        \set{(a_1, \dots, a_n) \in \mathbb{C}^n | f(a_1, \dots, a_n) = 0, \forall f \in \mathcal{F}}
    \]
    where \(\mathcal{F} \subseteq R\) is a possibly infinite set of polynomials.

    Let \(I = \set{\sum_{i=1}^m \lambda_i f_i | m \in \mathbb{N}, \lambda \in R, f_i \in \mathcal{F}}\), then \(I \nsub R\). Because \(R\) is Noetherian, by \cref{th:hilbertbasis},
    \[
        I = (g_1, \dots, g_r). \qquad g_i \in I
    \]
    Thus, \(V = \set{(a_1,\dots, a_n)\in \mathbb{C}^n| g_i(a_1, \dots, a_n) = 0, 1\leq i \leq r}\).
\end{example}
\begin{lemma}{}{}
    Let \(R\) be a Noetherian ring and \(I \nsub R\). Then \(\nicefrac{R}{I}\) is Noetherian.
\end{lemma}
\begin{proof}
    Let \(J_1' \subseteq J_2' \subseteq \dots\) be a chain of ideals in \(\nicefrac{R}{I}\).

    By ideal correspondence, we have \(J_i' = \nicefrac{J_i}{I}\) for some \(J_1 \subseteq J_2 \subseteq \dots\) a chain of ideals in \(R\) containing \(I\). Because \(R\) is Noetherian, there exists \(N \in \mathbb{N}\) such that \(J_n = J_{n+1}\) for all \(n \geq  N\). So \(J_n' = J_{n+1}'\) for all \(n \geq N\). Thus, \(\nicefrac{R}{I}\) is Noetherian.
\end{proof}
\begin{example}
    \begin{enumerate}
        \item \(\mathbb{Z}[i] = \nicefrac{\mathbb{Z}[X]}{(X^2 + 1)}\) is Noetherian.
        \item \(R[X]\) is Noetherian, then \(R = \nicefrac{R[X]}{X}\) is Noetherian.
    \end{enumerate}
\end{example}
\begin{example}[Examples of non-Noetherian rings]
    \leavevmode
    \begin{enumerate}
        \item \(R = \mathbb{Z}[X_1, X_2, \dots] = \cup_{n \geq 1} \mathbb{Z}[X_1, \dots, X_n]\); that is, the polynomials in countably many variables. The sequence of ideals
        \[
            (X_1) \subsetneq (X_1, X_2) \subsetneq (X_1, X_2, X_3) \subsetneq \dots
        \]
        is an infinite ascending chain.
        \item In \(R = \set{f \in \mathbb{Q}[X] | f(0) \in \mathbb{Z}} \leq \mathbb{Q}[X]\), the sequence of ideals
        \[
            (X) \subsetneq (\frac{1}{2} X) \subsetneq (\frac{1}{4} X) \subsetneq \dots
        \]
        is an infinite ascending chain since \(2 \in R\) is not a unit.
    \end{enumerate}
\end{example}
\section{Modules}
\subsection{Definitions and Examples}
\begin{definition}{}{}
    Let \(R\) be a ring. A \textit{module} over \(R\) is a triple \((M, + , \cdot)\) consisting of a set \(M\) and two operations \(+: M \times M \to M\) and \(\cdot : R \times M \to M\) such that
    \begin{enumerate}
        \item \((M, +)\) is an Abelian group, say with identity \(0\)(\(=0_M\)).
        \item The operation \(\cdot\) satisfies
        \begin{align*}
            (r_1 + r_2) \cdot m &= r_1 \cdot m + r_2 \cdot m &&\forall r_1, r_2 \in R, m \in M\\
            r \cdot (m_1 + m_2) &= r\cdot m_1 + r \cdot m_2 &&\forall r \in R, m_1, m_2 \in M\\
            r\cdot (r_2 \cdot m) &= (r_1 r_2) \cdot m &&\forall r_1, r_2 \in R, m \in M\\
            1_R \cdot m &= m &&\forall m \in M.
        \end{align*}
    \end{enumerate}
\end{definition}
\begin{remark}
    Don't forget closure when checking \(+, \cdot\) are well-defined.
\end{remark}
\begin{example}
    \begin{enumerate}
        \item Let \(R = F\) be a field. Then an \(F\)-module is precisely the same as a vector space over \(F\).
        \item \(R = \mathbb{Z}\), a \(\mathbb{Z}\)-module is precisely the same as an Abelian group, where
        \[
            \fullfunction{\cdot}{\mathbb{Z}\times A}{A}{(n, a)}{\begin{dcases}
                \overbrace{a + \cdots + a}^{n \text{ times}}, &\text{ if } n > 0\\
                0, &\text{ if } a = 0\\
                \underbrace{a + \dots + a}_{n\text{ times}}, &\text{ if } n< 0\\
            \end{dcases}}.
        \]
        \item Let \(F\) be a field, \(V\) a vector space over \(F\) and \(\alpha: V \to V\) a linear map. We can make \(V\) into an \(F[X]\)-module via
        \[
            \fullfunction{\cdot}{F[X]\times V}{V}{(f, v)}{(f(\alpha))(v)}.
        \]
        \begin{note}
            Different choices of \(\alpha\) make \(V\) into different \(F[X]\)-modules---sometimes write \(V = V_\alpha\) to make this clear.
        \end{note}
    \end{enumerate}
\end{example}