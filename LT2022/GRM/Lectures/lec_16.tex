\lecture{16}{24 Feb. 2022}{}
\begin{example}
    \leavevmode
    \begin{enumerate}
        \item By \cref{th:polyufd}, \(\mathbb{Z}[X]\) is a UFD.
        \item \(R[X_1, \ldots, X_n]\) be the polynomial ring in \(X_1, \ldots, X_n\) with coefficients in \(R\). We can define it inductively by \(R[X_1, \ldots, X_n] = R[R_1, \ldots, X_{n-1}][X_n]\). Applying \cref{th:polyufd} inductively, \(R[X_1, \ldots, X_n]\) is a UFD if \(R\) is a UFD.
    \end{enumerate}
\end{example}
\begin{theorem}{Eisenstein's Criterion}{}
    Let \(R\) be a UFD and \(f(X) = a_n X^n + \cdots + a_1 X + a_0 \in R[X]\) primitive. Suppose \(\exists p \in R\) irreducible such that
    \begin{enumerate}
        \item \(p \nmid a_n\);
        \item \(p \mid a_i\) for all \(0 \leq i \leq n - 1\);
        \item \(p^2 \nmid a_0\).
    \end{enumerate}
    Then \(f\) is irreducible in \(R[X]\).
\end{theorem}
\begin{proof}
    Suppose \(f = gh\) with \(g, h \in R[X]\) not units. Polynomial \(f\) being primitive implies that \(\deg g, \deg h > 0\). Let \(g = r_k X^k + \cdots + r_1 X + r_0\), and \(h = s_l X^l + \cdots + s_1 X + s_0\) with \(k + l = n\). Then \(p \nmid a_n = r_k s_l \implies p \nmid r_k\) and \(p \nmid s_l\), and \(p \mid a_0 = r_0 s_0\implies p \mid r_0\) or \(p \mid s_0\). WLOG, \(p \mid r_0\). Then \(\exists j \leq k\) such that \(p \mid r_0, p \mid r_1, \ldots, r\mid r_{j-1}, r\nmid r_j\), and
    \[
        a_j = r_0 s_j + r_1 s_{j-1} + \cdots + r_{j-1}s_1 + r_j s_0.
    \]
    Since \(j < n\), \(p\mid a_j\), we have \(p \mid r_j s_0 \implies p \mid s_0 \implies p^2 \mid r_0 s_0 = a_0\), \(\contra\).
\end{proof}
\begin{example}
    \begin{enumerate}
        \item \(f(X) = X^3 + 2X + 5 \in \mathbb{Z}[X]\).

        If \(f\) is not irreducible in \(\mathbb{Z}[X]\), then
        \[
            f(X) = (X+a)(X^2 + bX + c)
        \]
        with \(a,b,c \in \mathbb{Z}\). Thus, \(ac = 5\), but \(\pm 1, \pm 5\) re not roots of \(f\), contradiction.

        By Gauss's Lemma, \(f\) is irreducible in \(\mathbb{Q}[X]\). Thus, \(\nicefrac{\mathbb{Q}[X]}{(f)}\) is a field.
        \item Let \(p \in \mathbb{Z}\) prime. Eisenstein's Criterion tells us \(X^n - p\) is irreducible in \(\mathbb{Z}[X]\). Hence, it is irreducible in \(\mathbb{Q}[X]\) by Gauss's Lemma.
        \item Let \(f(X) = X^{p-1} + X^{p-2} + \cdots + X + 1 \in \mathbb{Z}[X]\) where \(p \in \mathbb{Z}\) is a prime number. Eisenstein does not apply directly. But note that \(f(X) = \frac{X^p - 1}{X - 1}\). Substituting \(Y = X - 1\) gives
        \[f(Y+1) = \frac{(Y+1)^p - 1}{(Y+1)-1} =Y^{p-1} + \binom{p}{1}Y^{p-2} + \cdots + \binom{p}{p-2}Y + \binom{p}{p-1}.\]
        Now \(p \mid \binom{p}{i}\) for all \(1\leq i \leq p-1\) and \(p^2 \nmid \binom{p}{p-1}=p\). Thus, \(f(Y+1)\) is irreducible in \(\mathbb{Z}[X]\), and \(f(X)\) is irreducible in \(\mathbb{Z}[X]\).
        
        This gives the cyclotomic extension of the rationals.
    \end{enumerate}
\end{example}
\subsection{Algebraic Integers}
Recall \(\mathbb{Z}[i] = \{a + bi \mid a, b \in \mathbb{Z}\} \leq \mathbb{C}\), the ring of Gaussian integers. The norm function \(N: \mathbb{Z}[i] \to \mathbb{Z}_{\geq 0}\) that maps \(a + bi \mapsto a^2 + b^2\) with \(N(z_1 z_2) = N(z_1)N(z_2)\) is a Euclidean function. Thus, \(\mathbb{Z}[i]\) is a ED, hence UFD and PID, and so primes are the same as irreducibles in \(\mathbb{Z}[i]\).

The units in \(\mathbb{Z}[i]\) are \(\pm 1, \pm i\) which are the only elements of norm \(1\).
\begin{example}
    \begin{enumerate}
        \item \(2 = (1+i)(1-i)\) and \(5 = (2 + i)(2-i)\) are not primes in \(\mathbb{Z}[i]\).
        \item \(N(3) = 9\) so if \(3 = ab\) in \(\mathbb{Z}[i]\), \(N(a)N(b) = 9\). But \(\mathbb{Z}[i]\) has no elements of norm \(3\). Thus, either \(a\) or \(b\) is a unit, so \(3\) is irreducible, and hence prime.
        
        Similarly, 7 is prime in \(\mathbb{Z}[i]\).
    \end{enumerate}
\end{example}
\begin{proposition}{}{ratprime}
    Let \(p \in \mathbb{Z}\) be a prime number. Then the following are equivalent.
    \begin{enumerate}
        \item \(p\) is not prime in \(\mathbb{Z}[i]\).
        \item \(p = a^2 + b^2\) for some integers \(a, b \in \mathbb{Z}\).
        \item \(p = 2\) or \(p \equiv 1 \bmod 4\).
    \end{enumerate}
\end{proposition}
\begin{proof}
    (1)\(\implies\)(2). Let \(p = xy\) where \(x, y \in \mathbb{Z}[i]\) not units. Then \(p^2 = N(p) = N(x)N(y)\), \(N(x),N(y) > 1\). Thus, \(N(x) = N(y) = p\). Writing \(x = a + bi\) gives \(p = N(x) = a^2 + b^2\).

    (2)\(\implies\)(3). The squares\({}\bmod 4\) are \(0\) and \(1\). Thus, if \(p = a^2 + b^2\), then \(p \not\equiv 3 \bmod 4\).
    
    (3)\(\implies\)(1). Already know that \(2\) is not prime in \(\mathbb{Z}[i]\). We know \((\nicefrac{\mathbb{Z}}{p\mathbb{Z}})^\times\) is cyclic with order \(p - 1\). So if \(p \equiv 1 \bmod 4\), then \((\nicefrac{\mathbb{Z}}{p\mathbb{Z}})^\times\) contains an element of order \(4\). That is, exists \(x \in \mathbb{Z}\) with \(x^4 \equiv 1 \bmod p\) but \(x^2 \not \equiv 1 \bmod p\). Then \(x^2 \equiv -1 \bmod p\). Now \(p \mid x^2 + 1 = (x + i)(x - i)\). But \(p \nmid x + i\) and \(p \nmid x - i\). Thus, \(p\) is not prime in \(\mathbb{Z}[i]\).
\end{proof}
\begin{remark}
    We also proved (3)\(\implies\)(2) in the process. It is an important result in number theory about sum of squares.
\end{remark}
\begin{theorem}{}{ziprime}
    The primes in \(\mathbb{Z}[i]\) (up to associates) are
    \begin{enumerate}
        \item \(a + bi\), where \(a, b \in \mathbb{Z}\) and \(a^2 + b^2 = p\) is a prime number with \(p = 2\) or \(p \equiv 1 \bmod 4\);
        \item prime numbers \(p \in \mathbb{Z}\) with \(p \equiv 3 \bmod 4\).
    \end{enumerate}
\end{theorem}
\begin{proof}
    We first prove that they are indeed prime.
    \begin{enumerate}
        \item \(N(a + bi) = p\). If \(a + bi = uv\), then either \(N(u) = 1\) or \(N(v) = 1\). Thus, \(a + bi\) is irreducible and hence prime.
        \item \cref{pr:ratprime}.
    \end{enumerate}
    Now let \(z \in \mathbb{Z}[i]\) be a prime (or irreducible). Then \(\overline{z} \in \mathbb{Z}[i]\) is irreducible, and \(N(z) = z \overline{z}\) is a factorization into irreducibles.

    Let \(p \in \mathbb{Z}\) be a prime such that \(p \mid N(z)\) because \(N(z) \neq 1\). If \(p \equiv 3 \bmod 4\), then \(p\) is prime in \(\mathbb{Z}[i]\). Thus, \(p \mid z\) or \(p \mid \overline{z}\), so \(p\) is an associate of \(z\) or \(\overline{z}\). Consider the units in \(\mathbb{Z}[i]\), we know \(p\) is an associate of \(z\).

    Otherwise, \(p = 2\) or \(p \equiv 1 \bmod 4\) and
    \[
        p = a^2 + b^2 = (a + bi)(a - bi)
    \]
    for some \(a, b \in \mathbb{Z}\). Then
    \[
        (a + bi)(a - bi) \mid z \overline{z}.
    \]
    Thus, \(z\) is an associate of \(a + bi\) or \(a - bi\) by uniqueness of factorization.
\end{proof}