\lecture{1}{20 Jan. 12:00}{Groups}
\section*{Introduction}
\subsection*{Groups}
Continuation from IA, focussing on
\begin{enumerate}
    \item Simple groups, p-groups, p-subgroups.
    \item Main results in this part of the course will be the Sylow Theorems.
\end{enumerate}
\subsection*{Rings}
Sets where you can add, subtract and multiply.

\begin{example}
    Examples of rings include,
    \begin{enumerate}
        \item \(\mathbb{Z}\) or \(\mathbb{C}[X]\).
        \item Rings of integers \(\mathbb{Z}[i], \mathbb{Z}[\sqrt{2} ]\)  (More in Part II Number Fields).
        \item Polynomial rings \(\mathbb{C}[x_1, \ldots , x_2 ]\) (More in Part II Algebraic Geometry).
    \end{enumerate}
\end{example}

A ring where you can divide is called a \textit{field}.
\begin{example}
    \(\mathbb{Q},\mathbb{R},\mathbb{C}\) or \(\mathbb{Z} / p\mathbb{Z}\) for p prime.
\end{example}

\subsection*{Modules}
An analogue of vector space where the scalars belong to a ring instead of a field.

We will classify modules over certain nice rings.

Allows us to prove Jordan normal form, and classify finite Abelian groups.

\section{Groups}
\subsection{Revision and Basic Theory}
We revisit basic properties and definition from Part IA Groups.
\begin{definition}[Group]
    A \textit{group} is a pair \((G, \cdot)\) where \(G\) is a set and \(\cdot: G \times G \to G\) is a binary operation satisfying:
    \begin{enumerate}
        \item (Associativity) \(a \cdot (b \cdot c) = (a \cdot b) \cdot c \).
        \item (Identity) \(\exists a \in G\) s.t. \(e\cdot g = g \cdot e = g~\forall g \in G\).
        \item (Inverses) \(\forall g \in G, \exists y^{-1}\in G \) s.t. \( g\cdot g^{-1} = g^{-1}\cdot g = e\).
    \end{enumerate}
\end{definition}
\begin{remark}
    Some things to note form definition of a group.
    \begin{enumerate}
        \item \textit{Closure} is included implicitly in the definition of a binary operation. In checking \(\cdot\) well-defined, we need to check closure, i.e. \(a,b \in G \implies a\cdot b \in G\).
        \item If using additive (or multiplicative) notation, often write 0 (or 1) for identity.
    \end{enumerate}
\end{remark}

\begin{definition}[Subgroup]
A subset \(H \subset G\) is a \textit{subgroup} (written \(H \leq G\)) if \(h\cdot h^{-1}\in H, \forall h,h^\prime \in H\), and \((H, \cdot)\) is a group.
Remark: A non-empty subset \(H\) of G is a subgroup if \(a, b \in H \implies a\cdot b^{-1}\in H\) 
\end{definition}

\begin{example}
    Here we list some common groups and their subgroups.
    \begin{enumerate}
        \item Additive \((\mathbb{Z},+) \leq (\mathbb{Q},+) \leq (\mathbb{R},+)\).
        \item Cyclic and dihedral group, \(C_n \leq  D_{2n}\).
        \item Abelian groups - those \((G, \cdot)\) such that \(a \cdot b = b \cdot a~\forall a, b \in G\) 
        \item Symmetric and Alternating groups, \(A_n \leq  S_n\).
        \item Quaternion group \(Q_8 = \{\pm 1, \pm i, \pm j, \pm k\}\).
        \item General and Special Linear Groups, \(SL_n(\mathbb{R}) \leq  SL_n(\mathbb{R})\).
    \end{enumerate}
\end{example}

\begin{definition}[Direct Product]
The \textit{(direct) product} of groups \(G\) and \(H\) is the set \(G\times H\) with operation given by
\[
    (g_1, h_1) \cdot (g_2, h_2) = (g_1 g_2, h_1 h_2).
\] 
\end{definition}

Let \(H \leq G\), the \textit{left cosets} of \(H\) in \(G\) are the sets \(gH = \{gh\mid h \in H\}\) for \(g \in G\). These partition \(G\), and each coset has the same cardinality as \(H\). So we can deduce.

\begin{theorem}[Lagrange's Theorem]
    Let \(G\) be a finite group and \(H \leq  G\), Then \(\left\vert G \right\vert = \left\vert H \right\vert \cdot [G:H]\) where \([G:H]\) is the number of left cosets of \(H\) in \(G\). \([G:H]\) is the \textit{index} of \(H\) in \(G\).
\end{theorem}

\begin{remark}
    Can also carry this out with right cosets. Lagrange's Theorem then implies that the number of left cosets is the same as the number of right cosets.
\end{remark}

\begin{definition}[Order]
    Let \(g \in G\). If \(\exists n \geq  1\) s.t. \(g^n = 1\) then the least such n is the \textit{order} of \(g\), otherwise we say that g has infinite order.
\end{definition}

\begin{remark}
    If \(g\) has order \(d\), then
    \begin{enumerate}
        \item \(g^{n} = 1 \implies d \mid n\).
        \item \(\{1,g, \ldots , g^{d-1}\} \leq G\) and so if \(G\) is finite, then \(d\mid \left\vert G \right\vert \) (by Lagrange's Theorem).
    \end{enumerate}
\end{remark}

\begin{definition}[Normal Subgroup]
    A subgroup \(H \leq  G\) is \textit{normal} if \(g^{-1}Hg = H~\forall g \in G\). We write \(H \trianglelefteq G\).
\end{definition}

\begin{proposition}
    If \(H \trianglelefteq G\) then the set \(G / H\) of left cosets of \(H\) in \(G\) is a group (called the \textit{quotient group}) with operation
\[
    g_1 H \cdot g_2 H = g_1 g_2 H.
\]
\end{proposition}
\begin{proof}
Check that \(\cdot\) is well-defined.

Suppose \(g_1 H = g_1 ^\prime H\) and \(g_2 H = g_2 ^\prime\). Then \(g_1 ^\prime =g_1 h_1\) and \(g_2 ^\prime  = g_2 h_2\) for some \(h_1, h_2 \in H\), we have
\begin{align*}
    g_1' g_2' &= g_2h_1g_2h_2\\
    &=g_1 g_2 (g_2^{-1}h_2 g_2)h_2
\end{align*}
so \(g_1 ^\prime g_2 ^\prime  H = g_1 g_2 H\).

Associativity is inherited from \(G\), the identity is \(H = e H\), and the inverse of \(g H\) is \(g^{-1} H\).
\end{proof}

\begin{definition}[Homomorphism]
    \(G, H\) groups. A function \(\phi: G \to H\) is a group homomorphism if \(\phi(g_1 g_2) = \phi(g_1)\phi(g_2) \forall g_1, g_2 \in G\). It has \textit{kernel}
\[
    \ker(\phi) = \{y \in G \mid \phi(y) = 1\} \trianglelefteq G,
\]
and \textit{image} \(\Ima(\phi)= \{\phi(y)\mid y\in G\} \leq H\).
\end{definition}

\begin{proof}
If \(a \in \ker(\phi)\) and \(g \in G\), then
\begin{align*}
    \phi(g^{-1}ag) &= \phi(g^{-1})\phi(a)\phi(g)\\
    &= \phi(g^{-1}) \phi(g)\\
    &= \phi(g^{-1}g)\\
    &= \phi(1)\\
    &= 1.
\end{align*}
So it is indeed a normal subgroup.
\end{proof}