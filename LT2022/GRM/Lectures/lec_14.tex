\lecture{14}{19 Feb. 2022}{}
\begin{example}
    \(\mathbb{Z}[X]\) is not a PID. Consider \(I = (2, X) \nsub \mathbb{Z}[X]\), then
    \begin{align*}
        I &= \{2f_1(X) + X f_2(X) \mid f_1, f_2 \in \mathbb{Z}[X]\}\\
        &=\{f\in \mathbb{Z}[X]\mid f(0)\text{ is even}\}.
    \end{align*}
    Suppose otherwise that \(I = (f)\) for some \(f \in \mathbb{Z}[X]\). So \(2 = fg\) for some \(g \in \mathbb{Z}[X]\). Thus, \(\deg f = \deg g = 0\), and \(f \in \mathbb{Z}\). So \(f = \pm 1\) of \(f = \pm 2\). Thus, \(I = \mathbb{Z}[X]\) or \(I = 2\mathbb{Z}[X]\). Contradiction for both cases.
\end{example}
\begin{definition}
    An integral domain is a \textit{unique factorization domain} (UFD) if
    \begin{enumerate}
        \item Every non-zero, non-unit element is a product of irreducibles.
        \item If \(p_1\cdots p_m = q_1 \cdots q_n\) where \(p_i,q_i\) are irreducible, then \(m = n\) and we can reorder so that \(p_i\) is an associate of \(q_i\) for all \(i = 1, \ldots, n\).
    \end{enumerate}
\end{definition}
We want to show that \(\text{PID}\implies \text{UFD}\).
\begin{proposition}
    \label{udfprime}
    Let \(R\) be an integral domain satisfying (1) in definition of UFD. Then \(R\) is UFD if and only if every irreducible is prime.
\end{proposition}
\begin{proof}
    "\(\implies\)" Suppose \(p \in R\) is irreducible, and \(p \mid ab\). Then \(ab = pc\) for some \(c \in R\). Writing \(a,b,c\) as products of irreducibles, it follows from (2) that \(p \mid a\) or \(p \mid b\). Thus, \(p\) is prime.

    "\(\impliedby\)" Suppose \(p_1\cdots p_m = q_1 \cdots q_n\) with \(p_i, q_i\) irreducible. Since \(p_1\) is prime, and \(p_1 \mid q_1 \cdots q_n\), we have \(p_1 \mid q_i\) for some \(i\). Upon reordering, we have \(p_1 \mid q_1\). That is, \(q_1 = p_1 u\) for some \(u \in R\). But \(q_1\) is irreducible, and \(p_1\) not a unit. Thus, \(p_1, q_1\) are associates. Cancelling \(p_1\) gives \(p_2 \cdots p_m = uq_2 \cdots q_n\). Result then follows by induction.
\end{proof}
\begin{lemma}
    \label{pidisno}
    Let \(R\) be a PID, and let \(I_1 \subseteq I_2 \subseteq I_3 \subseteq \cdots\) be a nested sequence of ideals, then \(\exists N \in \mathbb{N}\) such that \(I_n = I_{n+1}\) for all \(n \geq N\). (Rings satisfying the "ascending chain condition" is called \textit{Noetherian Rings})
\end{lemma}
\begin{proof}
    Let \(I = \bigcup_{i=1}^\infty I_i\), and this is an ideal in \(R\). (See example sheet 2)

    Since \(R\) is a PID, we have \(I = (a)\) for some \(a \in R\). Then \(a \in \bigcup_{i=1}^\infty I_i\), so \(a \in I_{N}\) for some \(N\).

    Then for any \(n \geq N\), we have
    \[
        (a) \subseteq I_N \subseteq I_n \subseteq I = (a),
    \]
    and so \(I_n = I\).
\end{proof}
\begin{theorem}
    If \(R\) is a principal ideal domain, then \(R\) is a unique factorization domain. (i.e. PID \(\implies\) UFD)
\end{theorem}
\begin{proof}
    We check (1) and (2) in the definition of UFD.
    \begin{enumerate}
        \item Let \(0 \neq x \in R\) not a unit. Suppose that \(x\) is not a product of irreducibles. Then \(x\) is not irreducible. So we can write \(x = x_{1}y_1\) where \(x_i, y_i\) are not units. Then either \(x_1\) or \(y_1\) is not a product of irreducible, say \(x_1\). We have \((x) \subseteq (x_1)\), and the inclusion is strict since \(y_1\) is not a unit.

        Now write \(x_1 = x_2 y_2\) where \(x_2, y_2\) are not units, and ad infinitum we get
        \[
            (x)\subsetneq (x_1) \subsetneq (x_2) \subsetneq \cdots,
        \]
        which is a contradiction to Lemma \eqref{pidisno}.
        \item By Proposition \eqref{udfprime}, it suffices to show irreducibles are primes. Conclude by Proposition \eqref{pidprime}.
    \end{enumerate}
\end{proof}
\begin{example}
    ED \(\implies\) PID \(\implies\) UFD \(\implies\) Integral Domains.
    \begin{enumerate}
        \item \(\nicefrac{\mathbb{Z}}{4\mathbb{Z}}\) is not an integral domain.
        \item \(\mathbb{Z}[\sqrt{-5}]\) is an integral domain that is not a UFD.
        \item \(\mathbb{Z}[X]\) is a UFD that is not a PID.
        \item \(\mathbb{Z}[\frac{1 + \sqrt{-19} }{2}]\) is a PID but not an ED. (It will be proved in Part II Number Fields)
        \item \(\mathbb{Z}[i]\) is an ED.
    \end{enumerate} 
    \begin{definition}
        \(R\) an integral domain, then
        \begin{enumerate}
            \item \(d \in R\) is a \textit{greatest common divisor} of \(a_1, \ldots,a_n \in R\) (written as \(d = \gcd(a_1, \ldots, a_n)\)) if \(d \mid a_i\) for all \(i\) and if \(d' \mid a_i\) for all \(i\), then \(d'\mid d\);
            \item \(m \in R\) is a \textit{least common multiple} of \(a_1, \ldots,a_n \in R\) (written as \(m = \mathrm{lcm}(a_1, \ldots,a_n)\)) if \(a_i \mid m\) for all \(i\) and if \(a_i \mid m'\) for all i, then \(m \mid m'\).
        \end{enumerate}
        Both \(\gcd\) and \(\lcm\) (when they exist) are unique up to associates.
    \end{definition}
\end{example}
\begin{proposition}
    In a UFD, both \(\gcd\) and \(\lcm\) exist.
\end{proposition}
\begin{proof}
    Write \(a_i = u_i \prod\limits_{j}p_j^{i,j}\) for all \(1 \leq i \leq n\), where \(u_i\) is a unit, the \(p_j\) are irreducible which are not associates of each other and \(n_{i,j} \in \mathbb{Z}_{\geq 0}\). We claim that \(d = \prod_j p_j ^{m_j}\) where \(m_j = \mathop{\min}_{1\leq i\leq n}n_{i,j}\) is the \(\gcd\) of \(a_i, \ldots, a_n\).

    Certainly, \(d \mid a_i\) for all \(i\). If \(d' \mid a_{i}\) for all \(i\), then writing \(d' = u \prod_j p_j^{t_j}\). We find that \(t_i \leq n_{i,j}\) for all \(i\), so \(t_j \leq m_j\). Therefore, \(d' \mid d\).

    The argument for \(\lcm\) is similar.
\end{proof}