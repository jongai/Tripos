\lecture{3}{25 Jan. 1:00}{}
\begin{proof}
    Induct on \(\abs{G}\). When \(\abs{G} =1\), the statement is obviously true.

    If \(\abs{G} > 1\), let \(G_{m - 1}\) be a normal subgroup of the largest possible order that is not \(\abs{G} \). By the correspondence theorem, \(\nicefrac{G}{G_{m - 1}}\) is simple.

    Apply inductively to \(G_{m-1}\).
\end{proof}

\subsection{Group Action}
\leavevmode
\begin{definition}{}{}
    For \(X\) a set, let \(\mathrm{Sym}(X)\) be the group of all bijections \(X \to X\) under composition. The identity is \(id = id_X\).

    A group \(G\) is a \textit{permutation group} of degree \(n\) if \(G \leq \mathrm{X}\) with \(\abs{Z} = n\).
\end{definition}
\begin{example}
    \leavevmode
    \begin{enumerate}
    \item \(S_n = \mathrm{Sym}(\{1,2, \ldots , n \})\) is a permutation group of degree \(n\), as is \(A_n \leq S_n\).
    \item \(D_{2n}\), the symmetries of regular n-gon, is a subgroup of \(\mathrm{Sym}(n)\).
    \end{enumerate}
\end{example}
\begin{definition}{}{}
    An \textit{action} of a group \(G\) on a set \(X\) is a function \(*:G\times X \to  X\) satisfying
    \begin{enumerate}
        \item \(e*x = x\) for all \(x \in X\),
        \item \((g_1g_2)*x = g_1*(g_2*x)\) for all \(g_1, g_2 \in G, x \in X\).
    \end{enumerate}
\end{definition}
\begin{proposition}{}{}
    An action of a group \(G\) on a set \(X\) is equivalent to specifying a group homomorphism \(\phi: G \to \mathrm{Sym}(X)\).
\end{proposition}
\begin{proof}
    For each \(g \in G\) let \(\phi_g: X \to X\), \(x \mapsto g* x\). We have
    \begin{align*}
        \phi_{g_1 g_2}(x) &= (g_1 g_2)*x\\
        &= g_1*(g_2*x)\\
        &= \phi_{g_1}(g_2 * x)\\
        &= \phi_{g_1}\circ\phi_{g_2}(x).
    \end{align*}
    Thus, \(\phi_{g_1 g_2} = \phi_{g_1} \phi_{g_2}\).
    
    In particular \(\phi_{g_1} \circ \phi_{g_1^{-1}} = \phi_{g_1^{-1}} \circ \phi_{g_1} = \phi_e = id_X\).

    Because \(\phi_g\) has an inverse, it is bijective. So \(\phi_g \in \mathrm{Sym}(X)\).
    Define
    \begin{equation*}
    \begin{aligned}
    \phi\colon G & \longrightarrow \mathrm{Sym}(X)      \\
    g          & \longmapsto \phi_g
    \end{aligned}
    \end{equation*}
    which is indeed a group homomorphism.

    Conversely, let \(\phi: G \to \mathrm{Sym}(X)\) be a gruop homomorphism.

    Define
    \begin{equation*}
    \begin{aligned}
      *\colon G\times X & \longrightarrow X      \\
                (g,x)&\longmapsto \phi(g)(x).
    \end{aligned}
    \end{equation*}
    Then it does satisfy the requirements for a group action,
    \begin{enumerate}
        \item \(e*x = \phi(e)(x) = id_X(x) = x\),
        \item
        \(\begin{aligned}[t]
        (g_1 g_2) * x &= \phi(g_1 g_2)(x)\\
        &= \phi(g_1)(\phi(g_2)(x))\\
        &=g_1 * (g_2 * x).
        \end{aligned}\)
    \end{enumerate}
\end{proof}
\begin{definition}{}{}
    We say \(\phi: G \to \mathrm{Sym}(X)\) is a \textit{permutation representation} of \(G\).
\end{definition}
\begin{definition}{}{}
    Let \(G\) act on a set \(X\).
    \begin{enumerate}
        \item The \textit{orbit} of \(x\in X\) is \(\mathrm{orb}_G(x) = \{g*x \mid g \in G\}\subseteq X\) 
        \item the \textit{stabilizer} of \(x \in X\) is \(G_x = \{g \in G \mid g * x = x\} \leq G\).
    \end{enumerate}
\end{definition}
Recalled from IA, we have the Orbit-Stabilizer Theorem.
There is a bijection \(\mathrm{orb}_G(x) \leftrightarrow \nicefrac{G}{G_x}\), the set of left cosets in G.

In particular, if \(G\) is finite, then
\[
    \abs{G} = \abs{\mathrm{orb}_G(x)} \abs{G_x}.
\]

\begin{example}
    Let \(G\) be the group of all symmetries of a cube, and \(X\) be the set of vertices. Let \(x \in X\) be any vertex \(\abs{\mathrm{orb}_G(x)} = 8, \abs{G_x} =8  \). So \(\abs{G} = 48\).
\end{example}
\begin{remark}
    \begin{enumerate}
        \item \(\ker \phi = \cap_{x\in X} G_x\) is called the \textit{kernel} of the group action.
        \item The orbits partition \(X\). We say that the action is \textit{transitive} if there is just one orbit.
        \item \(G_{g*x} = g G_x g^{-1}\), so if \(x,y \in X\) belong to the same orbit, then their stabilizers are conjugate.
    \end{enumerate}
\end{remark}
\begin{example}
    \leavevmode
    \begin{enumerate}
        \item Let \(G\) act on itself by left multiplication. That is, \(g*x = gx\). The kernel of this action is
    \[
        \{g \in G\mid g*x = x ~\forall x\in G\} = \textbf{1} .
    \]
    Thus, \(G\) injects into \(\mathrm{Sym}(G)\). This proves,
\end{enumerate}
\begin{theorem}{Cayley's Theorem}{}
    Any finite group \(G\) is isomorphic to a subgroup of \(S_n\) for some \(n\) (take \(n = \abs{G} \)).
\end{theorem}
\begin{enumerate}
    \setcounter{enumi}{1}
\item Let \(H \leq G\), \(G\) acts on \(\nicefrac{G}{H}\), the set of left cosets, by left multiplication. That is \(g*xH = gxH\).

This action is transitive (since \((x_{2}x_1^{-1})x_{1}H = x_2 H \)) with
\begin{align*}
    G_{xH} &= \{g\in G \mid gxH = xH\}\\
    &= \{g\in G \mid x^{-1}gx\in H\}\\
    &= xHx^{-1}.
\end{align*}
Thus, \(\ker(\phi) = \cap_{x\in G} xHx^{-1}\). This is the largest normal subgroup of \(G\) that is contained in \(H\)
\end{enumerate}
\end{example}
\begin{theorem}{}{}
    Let \(G\) be a non-Abelian simple group, and \(H\leq G\) a subgroup of index \(n>1\). Then \(n\geq 5\) and \(G\) is isomorphic to a subgroup of \(A_n\).
\end{theorem}
\begin{proof}
    Let \(G\) act on \(X = \nicefrac{G}{H}\) by left coset multiplication, and let \(\phi:G \to \mathrm{Sym}(X)\) be associated permutation representation.

    As \(G\) is simple, \(\ker(\phi)=\mathbf{1} \text{ or } G\). Since \(G\) acts transitively on \(X\) and \(\abs{X} >1\), \(\ker(\phi) = \textbf{1} \) and \(G \cong \ima(\phi)\leq S_n\).

    Since \(G \leq S_n\) and \(A_n\) is a normal subgroup of \(S_n\). The Second Isomorphism Theorem gives \( G\cap A_n\cong \nicefrac{GA_n}{A_n} \leq \nicefrac{S_n}{A_n} \cong C_2 \). Because \(G\) is simple, we have \(G\cap A_{n} = 1 \text{ or } G\). If the intersection is trivial, we have an injection into \(C_2\) by First Isomorphism Theorem, but \(G\) is non-Abelian. So we must have
    \[
        G\cap A_n = G \implies G \leq A_n.
    \]
    Finally, if \(n\leq 4\), it is easy to check that \(A_n\) does not have non-Abelian simple subgroups. So we must have \(n > 5\).
\end{proof}