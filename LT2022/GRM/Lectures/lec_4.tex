\lecture{4}{27 Jan. 12:00}{}
\begin{enumerate}
    \setcounter{enumi}{2}
    \item If \(G\) is a group. Let \(G\) act on itself by conjugation. That is \(g * x = gxg^{-1}\). We have the following definitions.
    \begin{align*}
        \mathrm{orb}_G(x) &= \{gxg^{-1}\mid g\in G\} = \mathrm{ccl}_G(x) &\textit{(conjugacy class)}\\
        G_x&=\{g\in G\mid gx = xg\} = C_G(x) \leq G &\textit{(centralizer)}\\
        \ker(\phi)&=\{g\in G\mid gx = xg~\forall x\in G\} = Z(G) \leq G. &\textit{(center)}
    \end{align*}
    \begin{note}
        The map \(\phi(g): G\to G\)  satisfies \(h\mapsto ghg^{-1}\) is a group homomorphism, and also a bijection. That is, it is an isomorphism from \(G\) to itself.
    \end{note}
    \begin{definition}
        \(\mathrm{Aut}(G) = \{\text{isomorphisms } f: G\to G\}\).
    \end{definition}
    Then \(\mathrm{Aut}(G) \leq \mathrm{Sym}(G)\), and \(\phi: G \to  \mathrm{Sym}(G)\) has image in \(\mathrm{Aut}(G)\).
    \item Let \(X\) be set of all subgroups of \(G\), then \(G\) acts on \(X\) by conjugation. That is, \(g*H = gHg^{-1}\).
   
    The stabilizer of \(H\) is \(\{g\in G\mid gHg^{-1}=H = N_G(H)\}\), called the \textit{normalizer} of \(H\) in \(G\). This is the largest subgroup of \(G\) containing \(H\) as a normal subgroup.

    In particular \(H \trianglelefteq G \iff N_G(H) = G\).
\end{enumerate}

\subsection{Alternating Groups}
In Part IA, we showed that the elements in \(S_n\) are conjugate if and only if they have the same cycle type.

\begin{example}
    In \(S_5\), we have the following table.
    \begin{center}
    \begin{tabular}{c|c|c}
        Cycle type & Number of Elements & Sign\\
        \hline
        \textbf{1}& 1 & +\\
        (* *)& 10& -\\
        (* *)(* *)& 15 & +\\
        (* * *)& 20 & +\\
        (* *)(* * *)& 20& -\\
        (* * * *)&30&-\\
        (* * * * *)&24&+\\
        \hline
        Total&120&
    \end{tabular}
\end{center}
\end{example}
Let \(g \in A_n\). Then \(C_{A_n}(g) = C_{S_n}(g)\cap A_n\). If there is an odd permutation commuting with \(g\),
\[
    \left\vert C_{A_n}(g) \right\vert = \frac{1}{2}\left\vert C_{S_n}(g) \right\vert \text{ and } \left\vert \mathrm{ccl}_{A_n}(g) \right\vert = \left\vert \mathrm{ccl}_{S_n}(g) \right\vert .
\]
Otherwise,
\[
    \left\vert C_{A_n}(g) \right\vert = \left\vert C_{S_n}(g) \right\vert \text{ and } \left\vert \mathrm{ccl}_{A_n}(g) \right\vert = \frac{1}{2}\left\vert \mathrm{ccl}_{S_n}(g) \right\vert .
\]
\begin{example}
    When \(n = 5\), \((1~2)(3~4)\) commutes with \((1~2)\), and \((1~2~3)\) commutes with \((4~5)\). But if \(h \in C_{S_5}(g)\), \(g=(1~2~3~4~5)\), then
    \begin{align*}
        (1~2~3~4~5) &= h (1~2~3~4~5) h^{-1}\\
        &=(h(1)~h(2)~h(3)~h(4)~h(5)),
    \end{align*}
    so \(h \in \langle g\rangle\leq A_5\), and it does split. Thus, \(A_5\) has conjugacy classes of sizes \(1,15,20,12,12\).
    
    To show the simplicity of \(A_5\). If \(H \trianglelefteq A_5\), then \(H\) is a union of conjugacy classes,
    \[
        \implies \left\vert H \right\vert = 1 + 15a + 20b + 12c
    \]
    with \(a,b \in \{0,1\}\) and \(c \in \{0,1,2\}\), and by Lagrange's Theorem \(\left\vert H \right\vert \mid 60\). So by simple arithmetic, \(\left\vert H \right\vert = 1\) or \(60\). That is \(A_5\) is simple.
\end{example}
\begin{lemma}
    \(A_n\) is generated by 3-cycles.
    \label{3gen}
\end{lemma}
\begin{proof}
    Each \(\sigma \in A_n\) is a product of an even number of transpositions. Thus suffices to write the product of any two transpositions as a product of 3-cycles.
    
    We have
    \begin{itemize}
        \item \((a~b)(b~c) = (a~b~c)\),
        \item \((a~b)(c~d) = (a~c~b)(a~c~d)\).
    \end{itemize}
\end{proof}
\begin{lemma}
    If \(n \geq 5\), then all 3-cycles in \(A_n\) are conjugate.
    \label{3con}
\end{lemma}
\begin{proof}
    We claim that every 3-cycle is conjugate to \(1~2~3\). Indeed, if \((a~b~c) = \sigma(1~2~3)\sigma^{-1}\) for some \(\sigma\in S_n\).
    If \(\sigma \notin A_n\), then replace \(\sigma\) by \(\sigma(4~5)\), and \(\sigma(4~5)\) is an element of \(A_n\). Note, here we use the fact that \(n\geq 5\). \(A_4\) is not simple in particular.
\end{proof}
\begin{theorem}
    \(A_n\) is simple for all \(n\geq 5\).
\end{theorem}
\begin{proof}
    Let \(1\neq N \trianglelefteq A_n\). Suffices to show that \(N\) contains a 3-cycle, since Lemma \eqref{3gen} and \eqref{3con} shows that we will have \(N=A_n\).

    Take \(1\neq \sigma\in N\) and write \(\sigma\) as a product of disjoint cycles. We consider three cases,
    \begin{enumerate}
        \item \(\sigma\) contains a cycle of length \(r \geq 4\). Without loss of generality, \(\sigma = (1~2~3~\cdots~r)\tau\). Let \(\delta = (1~2~3)\), and we have
        \begin{align*}
            \sigma^{-1}\delta^{-1}\sigma\delta &= (r~\cdots~2~1)(1~3~2)(1~2~\cdots~r)(1~2~3)\\
            &=(2~3~r).
        \end{align*}
        This implies that \(N\) contains a 3-cycle.
        \item \(\sigma\)contains two 3-cycles. Without loss of generality, let \(\sigma=(1~2~3)(4~5~6)\tau\). Let \(\delta = (1~2~4)\), we have
        \begin{align*}
            \sigma^{-1}\delta^{-1}\sigma\delta&=(1~3~2)(4~6~5)(1~4~2)(1~2~3)(4~5~6)(1~2~4)\\
            &=(1~2~4~3~6).
        \end{align*}
        Thus, we are done by case 1.
        \item \(\sigma\) contains two 2-cycles. Without loss of generality, let \(\sigma = (1~2)(3~4)\tau\). Let \(\delta = (1~2~3)\), we have
        \begin{align*}
            \sigma^{-1}\delta^{-1}\sigma\delta&=(1~2)(3~4)(1~3~2)(1~2)(3~4)(1~2~3)\\
            &=(1~4)(2~3)\equiv \pi.
        \end{align*}
        Let \(\epsilon = (2~3~5)\) (here we also used \(n \geq 5\)), we have
        \begin{align*}
            \pi^{-1}\epsilon^{-1}\pi\epsilon&=(1~4)(2~3)(2~5~3)(1~4)(2~3)(2~3~5)\\
            &=(2~5~3).
        \end{align*}
        Thus, \(N\) contains a 3-cycle.
    \end{enumerate}
    It remains to consider \(\sigma\) with cycle type (* *), (* *)(* * *) which are not elements of \(A_n\), and (* * *) which is a 3-cycle itself.
\end{proof}