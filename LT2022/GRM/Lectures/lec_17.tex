\lecture{17}{26 Feb. 2022}{}
\begin{remark}
    In \cref{th:ziprime}, if \(p = a^2 + b^2\), then \(a + bi\) and \(a - bi\) are not associates unless \(p = 2\). (\((1 + i) = (1-i)i\))
\end{remark}
\begin{corollary}{}{sumofsquare}
    An integer \(n \geq 1\) is the sum of 2 squares if and only if every prime factor \(p\) of \(n\) with \(p \equiv 3 \bmod 4\) divides \(n\) to an even power.
\end{corollary}
\begin{proof}
    \(n = a^2 + b^2\) if and only if \(n = N(x)\) for some \(x \in \mathbb{Z}[i]\) if and only if \(n\) is a product of norms of primes in \(\mathbb{Z}[i]\).

    \cref{th:ziprime} implies that the norms of primes in \(\mathbb{Z}[i]\) are the primes \(p \in \mathbb{Z}\) with \(p \not \equiv 3 \bmod 4\)
\end{proof}