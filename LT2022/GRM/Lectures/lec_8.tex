\lecture{8}{5 Feb. 2022}{}
\section{Rings}
\subsection{Definitions and Examples}
\leavevmode
\begin{definition}{}{}
    A \textit{ring} is a triple \((R, + , \cdot)\) consisting of a set \(R\) and two binary operations \(+: R \times R \to R\) and \(\cdot: R \times R \to R\) satisfying the following axioms:
    \begin{enumerate}
        \item \((R, +)\) is an Abelian group with identity \(0 = 0_R\).
        \item Multiplication is associative and has an identity. i.e.
        \[x\cdot(y\cdot z) = (x\cdot y)\cdot z~\forall x,y,z\in R\]
        and
        \[\exists 1=1_{R} \in R \text{ s.t. } x\cdot 1 = 1 \cdot x = x~\forall x\in R.\]
        We say \(R\) is a \textit{commutative} ring if \(x \cdot y = y \cdot x~\forall x,y \in R\). In fact, in this course, we will only consider commutative rings.
        \item Distributivity: \(x \cdot (y + z) = x \cdot y + x \cdot z\) and \((y + z)\cdot x = y \cdot x + z \cdot x\) for all \(x, y, z \in R\).
    \end{enumerate}
\end{definition}
\begin{remark}
    \leavevmode
    \begin{enumerate}
        \item As in case of groups, we need to check closure.
        \item For \(x\in R\), we write \(-R\) as the inverse of \(x\) under \(+\), and abbreviate \(x + (-y)\) as \(x - y\).
        \item \(0 \cdot x = (0 + 0)\cdot x = 0 \cdot x + 0 \cdot x \implies 0 \cdot x = 0\quad\forall x \in R\).
        \item \(\begin{aligned}[t]
            0 = 0\cdot x &= (1-1)\cdot x\\
            &= 1\cdot x + (-1)\cdot x\\
            &= x + (-1)\cdot x\\
            \implies& (-1)\cdot x = -x\quad \forall x \in R.
        \end{aligned}\) 
    \end{enumerate}
\end{remark}
\begin{definition}{}{}
    A subset \(S \subseteq R\) is a \textit{subring} (written \(S \leq R\)) if it is a ring under \(+\) and \(\cdot\), with the same identity elements \(0\) and \(1\).
\end{definition}
\begin{example}
    \leavevmode
    \begin{enumerate}
        \item \(\mathbb{Z} \leq \mathbb{Q}\leq \mathbb{R}\leq \mathbb{C}\).
        \item \(\mathbb{Z}[i] = \{a + bi\mid a,b \in \mathbb{Z}\}\leq \mathbb{C}\) (ring of Gaussian integers).
        \item \(\mathbb{Q}[\sqrt{2}] = \{a + b \sqrt{2} \mid a, b \in \mathbb{Q}\} \leq \mathbb{R}\).
        \item \(\nicefrac{\mathbb{Z}}{n\mathbb{Z}}=\{\text{integers mod }n\}\).
        \item If \(R, S\) are rings, the ring \(R \times S\) is a ring via the operations
        \begin{itemize}
            \item \((r_1,s_1) + (r_2, s_2) = (r_1 + r_2, s_1 + s_2)\);
            \item \((r_1,s_1)\cdot (r_2,s_2) = (r_1 \cdot r_2, s_1 \cdot s_2)\);
            \item \(0_{R\times S} = (0_R, 0_S)\) and \(1_{R\times S} = (1_R, 1_{S})\).
        \end{itemize}
        (Note: \(R \times \{0\}\) is not a subring).
        \item If \(R\) is a ring, a \textit{polynomial}  \(f\) over \(R\) is an expression \(f = a_0 + a_{1}X + a_2 X^2 + \cdot + a_n X^n, a_i \in R\). "\(X\)" is just a formal symbol.

        The degree of a polynomial is largest \(n \in \mathbb{N}\) such that \(a_n\) is nonzero. We write \(R[X]\) for the set of all polynomials over \(R\).

        If \(g = b_0 + b_1 X + \cdots + b_m X^m\) is another polynomial, set
        \begin{align*}
            f + g &= \sum\limits_{i}(a_i + b_i)X^i,\\
            f \cdot g &= \sum_i(\sum\limits_{j=0}^{i} a_{j}b_{i - j})X^i.
        \end{align*}
        Then \(R[X]\) is a ring with identities \(0_R\) and \(1_R\) which are constant polynomials.

        A \textit{monic polynomial}  is one such that the leading coefficient \(a_n = 1_R\).

        We can identify \(R\) with subring of \(R[X]\) of constant polynomials (i.e. \(\sum\limits_{i}^{} a_i X^i, a_i = 0~\forall i > 0\)).
    \end{enumerate}
\end{example}
\begin{definition}{}{}
    An element \(r \in R\) is a \textit{unit}  if it has an inverse under multiplication, i.e. \(\exists s \in R\) s.t. \(s \cdot r = 1\).
\end{definition}
The units in \(R\) form a group \((R^{\times}, \cdot)\) under multiplication.
\begin{example}
    \leavevmode
    \begin{itemize}
    \item \(\mathbb{Z}^\times = \{\pm 1\}\).
    \item \(\mathbb{Q}^\times = \mathbb{Q}\setminus \{0\}\) 
    \end{itemize}
\end{example}
\begin{definition}{}{}
    A \textit{field} is a ring with \(0 \neq 1\), such that every non-zero element is a unit.
\end{definition}
\begin{example}
    \(\mathbb{Q}\), \(\nicefrac{\mathbb{Z}}{p\mathbb{Z}}\) with \(p\) prime.
\end{example}
\begin{remark}
    If \(R\) is a ring with \(0 = 1\), then \(x = 1\cdot x = 0 \cdot x = 0\) for all \(x \in R\). And \(R = \{0\}\) is the \textit{trivial ring}.
\end{remark}
\begin{proposition}{}{}
    Let \(f, g \in R[X]\). Suppose the leading coefficient of \(g\) is a unit. Then there exist \(q, r \in R[X]\) such that
    \[
        f(x) = q(x)g(x) + r(x) \text{ where \(\mathrm{deg}(r) < \mathrm{deg}(g)\)}.
    \]
    \label{polydiv}
\end{proposition}
\begin{proof}
    Induction on \(n = \deg(f)\). Write
    \begin{align*}
        f(X) &= a_n X^n + a_{n - 1}X^{n - 1} + \cdots + a_0, &a_n &\neq 0\\
        g(X) &= b_m X^m + b_{m - 1}X^{m - 1} + \cdots + b_0. &b_m &\in R^\times
    \end{align*}
    If \(n < m\), then put \(q = 0, r = f\), and we are done.

    Otherwise, we have \(n \geq m\), and we set
    \[
        f_1(X) = f(X) - a_n b_m^{-1}g(X)X^{n - m}
    \]
    because \(b_m\) is a unit. The coefficient of \(X^n\) is \(a_n - a_n b_m^{-1} b_m = 0\). Thus, \(\deg(f_1) < n\). By inductive hypothesis \(\exists q_1, r \in R[X]\) such that
    \[
        f_1(X) = q_1(X)g(X) + r(X), \text{ where } \deg(r) < \deg(g).
    \]
    Therefore,
    \[
        f(X) = (q_1(X) + a_n b_m^{-1}X^{n-m})g(X) + r(X).
    \]
\end{proof}
\begin{remark}
    We often work with polynomials over a field, then we only need the assumption that \(g \neq 0\).
\end{remark}
\begin{example}
    \leavevmode
    \begin{enumerate}
        \item If \(R\) is a ring and \(X\) a set, then the set of all functions \(X \to R\) is a ring under point-wise operations. That is,
        \begin{align*}
            (f + g)(x) &= f(x) + g(x);\\
            (f \cdot g)(x) &= f(x) \cdot g(x).
        \end{align*}
        More interesting examples will appear as subrings. For example, the continuous functions from \(\mathbb{R}\to \mathbb{R}\) and the polynomial functions \(\mathbb{R} \to \mathbb{R}\) which is \(\mathbb{R}[X]\).
        \item Power series ring. \(R\llbracket X\rrbracket = \{a_0 + a_{1}X + a_2 X^2 + \cdots \mid a_i \in \mathbb{R}\}\) with the same operation as the polynomial. (you should not think this as infinite sum of elements, but a formal object instead)
        \item Laurent polynomials.
        \[R[X, X^{-1}]=\{\sum\limits_{i\in \mathbb{Z}} a_i X^i\mid a_i \in R, a_i \text{ is non-zero for finitely many \(i\)}\}.\]
    \end{enumerate}
\end{example}