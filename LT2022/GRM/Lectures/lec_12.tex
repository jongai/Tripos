\lecture{12}{15 Feb. 2022}{}
\begin{remark}
    If \(\mathrm{char}(R) = n\), then \(\nicefrac{\mathbb{Z}}{n\mathbb{Z}}\leq R\). So if \(R\) is an integral domain, \(\nicefrac{\mathbb{Z}}{n\mathbb{Z}}\) is an integral domain. So \(n\mathbb{Z} \nsub \mathbb{Z}\) is a prime ideal, and we have \(n = 0\) or \(n = p\) a prime.

    In particular, a field has characteristic \(0\) (and so contains \(\mathbb{Q}\)) or has characteristic \(p\) (and contains \(\nicefrac{\mathbb{Z}}{p\mathbb{Z}} = \mathbb{F}_p\)).
\end{remark}
\subsection{Factorization in Integral Domains}
In this section, \(R\) is always an integral domain.
\begin{definition}
    \leavevmode
    \begin{enumerate}
        \item \(a \in R\) is a \textit{unit} if \(\exists b\in R\) with \(ab = 1\). (equivalently, \((a) = R\)) And we write the units in \(R\) as \(R^\times \).
        \item \(a \in R\) divides \(b \in R\) (written \(a \mid b\)) if \(\exists c \in R\) such that \(b = ac\). (equivalently, \((a) \subseteq (b)\))
        \item \(a,b \in R\) are \textit{associates}  if \(a = bc\) for some unit \(c \in R\). (equivalently, \((b) = (a)\))
        \item \(r \in R\) is \textit{irreducible} if \(r\neq 0\), \(r\) is not a unit, and \(r = ab \implies a\) or \(b\) is a unit.
        \item \(r \in R\) is \textit{prime}  if \(r \neq 0\) and \(r\) is not a unit, and \(r\mid ab \implies r \mid a\) or \(r\mid b\).
    \end{enumerate}
\end{definition}
\begin{note}
    These properties depend on ambient ring \(R\). For example,
    \begin{enumerate}
    \item \(2\) is prime and irreducible in \(\mathbb{Z}\), but not in \(\mathbb{Q}\).
    \item \(2X\) is irreducible in \(\mathbb{Q}[X]\), but not in \(\mathbb{Z}[X]\).
    \end{enumerate}
\end{note}
\begin{lemma}
    \((r) \nsub R\) is a prime ideal if and only if \(r = 0\) or \(r\) is prime.
\end{lemma}
\begin{proof}
    \(\implies\) direction. Suppose \((r)\) is prime and \(r \neq 0\). Since prime ideals are proper, \((r) \neq R\), so \(r \notin R^\times \). If \(r \mid ab\), then \(ab \in (r)\), so \(a \in (r)\) or \(b \in (r)\), and we have \(r\mid a\) or \(r \mid b\). That is, \(r\) is prime.

    \(\impliedby\) direction. \(\{0\}\nsub R\) is a prime ideal since \(R\) is an integral domain. Let \(r \in R\) be a prime. \((r) \neq R\) since \(r \notin R^\times \). If \(ab \in (r)\), then \(r \mid ab\) so \(r \mid a\) or \(r \mid b\). That is \(a \in (r)\) or \(b \in (r)\); that is, \((r)\) is a prime ideal.
\end{proof}
\begin{lemma}
    If \(r\) is a prime element, then it is irreducible.
\end{lemma}
\begin{proof}
    Since \(r\) is prime, \(r\neq 0\) and \(r \notin R^\times \). Suppose \(r = ab\). Then \(r \mid ab\) so \(r \mid a\) or \(r \mid b\). Assume \(r \mid a\), so \(a = rc\) for some \(c \in R\). Then \(r = ab = rcb\implies r(1-bc)=0\implies bc = 1\) since \(R\) is an integral domain and \(r\neq 0\). That is, \(b\) is a unit.
\end{proof}
The converse does not hold in general.
\begin{example}
    Let \(R=\mathbb{Z}[\sqrt{-5}] = \{a + b\sqrt{-5} \mid a,b \in \mathbb{Z}\}\leq \mathbb{C}\) which is isomorphic to \(\nicefrac{\mathbb{Z}[X]}{(X^2 + 5)}\). \(R\) is a subring of a field, so an integral domain.

    Define a function \(N: R \to \mathbb{Z}\), \(a + b\sqrt{-5} \mapsto a^2 + 5b^2\).

    Note that \(N(z_1 z_2) = N(z_1) N(z_2)\). We claim that \(R^\times = \{\pm 1\}\).
    \begin{proof}
        If \(r \in R^\times \), i.e. \(rs = 1\) for some \(s \in R\). Then
        \[
            N(r)N(s) = N(1) = 1 \implies N(r) = 1.
        \]
        But the only integer solutions to the equation \(a^2 + 5b^2 = 1\) are \((\pm 1,0)\).
    \end{proof}
    Next we note that \(2 \in R\) is irreducible.
    \begin{proof}
        Suppose \(2 = rs\) with \(r,s \in R\). Then \(4 = N(2) = N(r)N(2)\). Since the equation \(a^2 + 5b^2 = 2\) has no integer solutions, \(R\) has no elements of norm 2. Thus, \(N(r) = 1\) and \(N(s) = 4\) or vice versa. But, \(N(r) = 1 \implies r\) is a unit.
    \end{proof}
    Similarly, \(3,1 + \sqrt{-5}, 1 - \sqrt{-5} \) are irreducible. (as these are the elements of norm 3)
    
    But we have \((1+\sqrt{-5} )(1-\sqrt{-5}) = 2 \cdot 3 = 6\), so \(2 \mid (1 + \sqrt{-5})(1 - \sqrt{-5})\). But we know \(2 \nmid 1 + \sqrt{-5} \) and \(2 \nmid 1 - \sqrt{-5} \). Check by taking norms, we have \(4 \nmid 6\). Thus, 2 is not prime in \(R\).
\end{example}
\begin{remark}
    \leavevmode
    \begin{enumerate}
        \item In general, irreducible elements don't have to be prime.
        \item \(2\cdot 3 = (1 + \sqrt{-5})(1 - \sqrt{-5})\) gives two different factorization into irreducibles. Since \(R^\times = \{\pm 1\}\), the irreducibles are not associates, so they are really different factorizations.
    \end{enumerate}
\end{remark}
\begin{definition}
    An integral domain \(R\) is a \textit{principal ideal domain} (PID) if any ideal \(I \nsub R\) is principal, i.e. \(I = (r)\) for some \(r \in R\).
\end{definition}
\begin{example}
    \(\mathbb{Z}\) is a PID as proved in previous lecture.
\end{example}
\begin{proposition}
    Let \(R\) be a PID. Then every irreducible element of \(R\) is a prime.
\end{proposition}
\begin{proof}
    Let \(r \in R\) be irreducible. If \(r \mid ab\) and \(r \nmid a\), Because \(R\) is a PID, \((a,r) = (d)\) for some \(d \in R\). In particular, \(r = cd\) for some \(c \in R\). Since \(r\) is irreducible, either \(c\) or \(d\) is a unit.

    If \(c\) a unit, then \((a,r) = (r)\). So \(r \mid a\), contradiction.

    If \(d\) a unit, then \((a,r) = R\). So exists \(s,t \in R\) such that \(sa + t r= 1\), then
    \[
        b = sab + r rb,
    \]
    and since \(r \mid ab\), we must have \(r \mid b\). Thus, \(r\) is a prime.
\end{proof}