\lecture{9}{8 Feb. 2022}{}
\subsection{Homomorphisms, Ideals and Quotients}
\leavevmode
\begin{definition}{}{}
    Let \(R\) and \(S\) be rings. A function \(\phi: R \to S\) is a \textit{ring homomorphism} if
    \begin{enumerate}
        \item \(\phi(r_1 + r_2) = \phi(r_1) + \phi(r_2)\quad \forall r_1, r_2 \in R\),
        \item \(\phi(r_1\cdot r_1) = \phi(r_1)\cdot \phi(r_2) \quad \forall r_1, r_2 \in R\),
        \item \(\phi(1_R) = 1_S\).
    \end{enumerate}
    A ring homomorphism that is also a bijection is called an \textit{isomorphism}.

    The \textit{kernel} of \(\phi\) is \(\ker(\phi) = \{r\in R\mid \phi(r) = 0_S\}\).
\end{definition}
\begin{lemma}{}{}
    A ring homomorphism \(\phi:R \to S\) is injective if and only if \(\ker(\phi) = \{0_R\}\).
\end{lemma}
\begin{proof}
    \(\phi: (R, +) \to (S, +)\) is also a group homomorphism. And the result follows from the corresponding result from groups.
\end{proof}
\begin{definition}{}{}
    A subset \(I \subseteq R\) is an \textit{ideal}, written \(I \trianglelefteq R\), if
    \begin{enumerate}
        \item \(I\) is a subgroup of \((R, +)\);
        \item if \(r \in R\) and \(x \in I\), then \(rx \in I\).
    \end{enumerate}
    We say \(I\) is \textit{proper} if \(I \neq R\).
\end{definition}
\begin{lemma}{}{}
    If \(\phi: R\to S\) is a ring homomorphism, then \(\ker(\phi)\) is an ideal of \(R\).
    \label{ringker}
\end{lemma}
\begin{proof}
    Again, \(\phi: (R, +) \to (S, +)\) is a group homomorphism between the additive groups. So \(\ker(\phi)\) is a subgroup of \((R, +)\).

    If \(r \in R\) and \(x \in \ker(\phi)\), then
    \[
        \phi(rx) = \phi(r)\phi(x) = \phi(r)\cdot 0_S = 0_S \implies rx \in \ker(\phi).
    \]
\end{proof}
\begin{remark}
    If \(I\) contains a unit, then \(1_R \in I\) because ideal is closed by multiplication with any element in \(R\); hence, \(I=R\). Thus, if \(I\) is a proper ideal, \(1_R \notin I\), so \(I\) is not a subring of \(R\).
\end{remark}
\begin{lemma}{}{}
    The ideals in \(\mathbb{Z}\) are \(n\mathbb{Z} = \{\ldots,-2n, -n, 0, n, 2n, \ldots\}\), for \(n = 0, 1, 2, \ldots\).
\end{lemma}
\begin{proof}
    Certainly, \(n\mathbb{Z} \nsub \mathbb{Z}\).

    Let \(I \nsub \mathbb{Z}\) be a non-zero ideal, and let \(n\) be the smallest positive integer in \(I\). Then \(n\mathbb{Z} \subseteq I\). If \(m \in I\), then write \(m = qn + r\) where \(q, r\in \mathbb{Z}\), and \(0\leq r < n\) by division algorithm. Then \(r = m - qn \in I\) contradicts the minimality of \(n\) unless \(r = 0\). Then we have \(m \in n\mathbb{Z}\); i.e. \(I = n\mathbb{Z}\).
\end{proof}
\begin{definition}{}{}
    For \(a \in R\), write \((a) = \{ra\mid r\in R\} \nsub R\). This is the \textit{ideal generated by \(a\)}.

    Generally, if \(a_1, \ldots, a_n \in R\), we write \textit{the ideal generated by \(a_1, \ldots, a_n\)} as
    \[
        (a_1, \ldots, a_n) = \{r_{1}a_1 + r_2 a_2 + \ldots + r_n a_n \mid r_i \in R\} \nsub R.
    \]
\end{definition}
\begin{definition}{}{}
    Let \(I \nsub R\), we say \(I\) is \textit{principal} if \(I = (u)\) for some \(u \in R\).
\end{definition}
\begin{theorem}{}{}
    If \(I \nsub R\), then the set \(\nicefrac{R}{I}\) of cosets of \(I\) in \((R, +)\) forms a ring (called the \textit{quotient ring}) with the operations
    \begin{align*}
        (r_1 + I) + (r_2 + I) &= r_1 + r_2 + I,\\
        (r_1 + I) \cdot (r_2 + I) &= r_1 \cdot r_2 + I
    \end{align*}
    and \(0_{\nicefrac{R}{I}} = O_R + I\), \(1_{\nicefrac{R}{I}} = 1_R + I\).

    Moreover, the map \(R \to \nicefrac{R}{I}, r \mapsto r + I\) is a ring homomorphism (called the \textit{quotient map}) with kernel \(I\).
\end{theorem}
\begin{proof}
    We already know that \((\nicefrac{R}{I}, +)\) is a group. We want to show that the multiplication is well-defined. If \(r_1 + I = r_1' + I\), \(r_2 + I = r_2' + I\), then for some \(a_1, a_2 \in I\), \(r_1' = r_1 + a_1\), \(r_2' = r_2 + a_2\). Then we have
    \begin{align*}
        r_1'r_2' &= (r_1 + a_1)(r_2 + a_2)\\
        &=r_{1r_2 + r_{1}}a_2 + r_2 a_1 + a_1 a_2.
    \end{align*}
    Thus, \(r_{1}r_2 + I = r_1'r_2' + I\).

    The remaining properties for \(\nicefrac{R}{I}\) follows from those properties for \(R\). And the quotient map is clearly a ring homomorphism from the definitions of the quotient ring.
\end{proof}
\begin{example}
    \leavevmode
    \begin{enumerate}
        \item \(n\mathbb{Z}\nsub \mathbb{Z}\) with quotient ring \(\nicefrac{\mathbb{Z}}{n\mathbb{Z}}\). To be precise, \(\nicefrac{\mathbb{Z}}{n\mathbb{Z}}\) has elements \(0 + n\mathbb{Z}, 1 + n\mathbb{Z}, \ldots\).
        Addition and multiplication are carried out \(\bmod n\).
        \item Consider \((X) \nsub \mathbb{C}[X]\), the polynomials with constant term \(0\).

        If \(f(X) = a_n X^n + \cdots + a_1 X + a_0, a_i \in \mathbb{C}\), then \(f(X) + (X) = a_0 + (X)\). There is a bijection from \(\nicefrac{\mathbb{C}[X]}{(X)} \to \mathbb{C}, f(X) + (X) \mapsto f(0)\).

        These maps are ring homomorphisms. Thurs, \(\nicefrac{\|C[X]}{(X)}\cong \mathbb{C}\).
        \item Consider \((X^2 + 1) \nsub \mathbb{R}[X]\),
        \[
            \nicefrac{\mathbb{R}[X]}{(X^2 + 1)} = \{f(X) + (X^2 + 1)\mid f(X) \in \mathbb{R}[X]\}.
        \]
        By Proposition \eqref{polydiv}, \(f(X) = q(X)(X^2 + 1) + r(X)\) with \(\deg r < 2\), i.e. \(r(X) = a + bX, a,b \in \mathbb{R}\). Thus,
        \[\nicefrac{\mathbb{R}}{(X^2 + 1)} = \{a + bX + (X^{2} + 1)\mid a, b \in \mathbb{R}\}.\]
        If \(a + bX + (X^2 + 1) = a' + b'X + (X^2 + 1)\), then
        \[a - a' + (b - b')X = g(X)(X^2 + 1).\]
        Comparing degrees, we see \(g(x) = 0\) and \(a = a'\) and \(b = b'\). Consider the bijection
        \begin{equation*}
        \begin{aligned}
          \phi \colon \nicefrac{\mathbb{R}[X]}{X^2 + 1} & \longrightarrow \mathbb{C} \\
                    a + bX + (X^2 + 1) & \longmapsto a + b i.
        \end{aligned}
        \end{equation*}
        We show that \(\phi\) is a ring homomorphism. It preserves addition and maps \(1 + (X^2 + 1)\) to 1. We show that it preserves multiplication.
        \begin{align*}
            &\phi((a + bX + (X^2 + 1))(c + dX + (X^2 + 1)))\\
            =&\phi((a + bX)(c+dX) + (X^2 + 1))\\
            =&\phi(ac + (ad + bc)X + bd(X^2 + 1) - bd + (X^2 + 1))\\
            =&ac - bd + (ad + bc)i\\
            =&(a+bi)(c + di)\\
            =&\phi(a + bX + (X^2 + 1))\phi(c + dX + (X^2 + 1)).
        \end{align*}
        Thus, \(\nicefrac{\mathbb{R}[X]}{(X^2 + 1)} \cong \mathbb{C}\).
    \end{enumerate}
\end{example}