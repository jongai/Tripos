\lecture{19}{3 Mar. 2022}{}
\begin{example}
    Now we consider some general constructions.
    \begin{enumerate}
        \item For any ring \(R\), \(R^n\) is an \(R\)-module via \(r \cdot (r_1, \dots, r_n) = (r r_1, \dots, r r_n)\).

        In particular, taking \(n = 1\), \(R\) is an \(R\)-module.
        \item If \(I \nsub R\), then \(I\) is an \(R\)-module. (restrict the usual multiplication on \(R\)) And \(\nicefrac{R}{I}\) is an \(R\)-module via \(r \cdot(s + I) = rs + I\).
        \item Let \(\phi: R \to S\) be a ring homomorphism. Then an \(S\)-module \(M\) may be regarded as an \(R\)-module via \(r \cdot m = \phi(r)m\).

        In particular, if \(R \leq S\) then any \(S\)-module may be viewed as an \(R\)-module
    \end{enumerate}
\end{example}
\begin{definition}{}{}
    Let \(M\) be an \(R\)-module. \(N \subseteq M\) is an \(R\)-submodule (written \(N \leq M\)) if it is a subgroup of \((M, +)\) and \(r \cdot n \in N\) for all \(r \in R, n \in N\).
\end{definition}
\begin{example}
    \begin{enumerate}
        \item A subset of \(R\) is an \(R\)-submodule precisely when it is an ideal.
        \item When \(R = F\) is a field, a module is a vector space, and a submodule is a vector subspace.
    \end{enumerate}
\end{example}
\begin{definition}{}{}
    If \(N \leq M\) is an \(R\)-submodule, the quotient \(\nicefrac{M}{N}\) is the quotient of groups under \(+\) with \(r \cdot (m + N) = r\cdot m + N\). This is well-defined, and makes \(\nicefrac{M}{N}\) and \(R\)-module.
\end{definition}
\begin{definition}{}{}
    Let \(M,N\) be \(R\)-modules. A function \(f: M \to N\) is an \textit{\(R\)-module homomorphism} if it is a homomorphism of Abelian groups and
    \[
        f(r \cdot m) = r \cdot f(m). \qquad \forall r\in R, m \in M
    \]
\end{definition}
\begin{example}
    If \(R = F\) is a field, an \(F\)-module homomorphism is just a linear map.
\end{example}
\begin{theorem}{First Isomorphism Theorem}{}
    Let \(f: M \to N\) be an \(R\)-module homomorphism. Then
    \begin{align*}
        \ker(f) &\coloneqq  \set{m\in M | f(m) = 0} \leq M\\
        \ima(f) &\coloneqq \set{f(m) \in N | m \in M} \leq N.
    \end{align*}
    And
    \[
        \frac{M}{\ker(f)}\cong \ima(f).
    \]
\end{theorem}
\begin{proof}
    Similar to before.
\end{proof}
\begin{theorem}{Second Isomorphism Theorem}{}
    Let \(A, B \leq M\) be \(R\)-submodules, then
    \begin{align*}
        A + B &\coloneqq \set{a + b | a\in A, b\in B} \leq M\\
        A \cap B &\in M.
    \end{align*}
    And
    \[
        \frac{A}{A \cap B} \cong \frac{A + B}{B}.
    \]
\end{theorem}
\begin{proof}
    Apply First Isomorphism Theorem to the composite \(A \xhookrightarrow{} M \to \nicefrac{M}{B}\).
\end{proof}
For Third Isomorphism Theorem, we note that there is a bijection
\[
    \{\text{submodules of }\nicefrac{M}{N}\} \longleftrightarrow\{\text{submodules of }M\text{ containing }N\}.
\]
\begin{theorem}{Third Isomorphism Theorem}{}
    If \(N \leq L \leq M\) are \(R\)-submodules, then
    \[
        \frac{\nicefrac{M}{N}}{\nicefrac{L}{N}}\cong \frac{M}{L}.
    \]
\end{theorem}
In particular, these apply to vector spaces. (compare with results from Linear Algebra)

Let \(M\) be a module over a ring \(r\). If \(m \in M\), write \(Rm = \set{rm \in M | r \in R}\), which is called the \textit{submodule generated by \(m\)}.

If \(A, B \leq M\), then \(A + B = \set{a + b | a \in A, b \in B}\).
\begin{definition}{}{}
    A module \(M\) is \textit{finitely generated} if \(\exists m_1,\dots, m_n \in M\) such that
    \[M = Rm_1 + Rm_2 + \dots + Rm_n.\]
\end{definition}
\begin{lemma}{}{}
    \(M\) is finitely generated if and only if there exists a surjective \(R\)-module homomorphism \(f: R^n \to M\) for some \(n \in \mathbb{N}\).
\end{lemma}
\begin{proof}
    We first prove \(\implies\) direction. If \(M = Rm_1 + \dots + Rm_n\). Define
    \[
        \fullfunction{f}{R^n}{M}{(r_1,\dots,r_n)}{\sum r_i m_i}
    \]
    which is a surjective \(R\)-module homomorphism.

    For the converse direction, let \(e_1 = (0,\dots,0,\underbrace{1}_{i\text{th place}},0,\dots,0) \in R^n\). Given \(f: R^n \to M\) surjective. Set \(m_i = f(e_i)\). Then any \(m \in M\) is of the form \(f(r_1, \dots, r_n) = f(\sum r_i e_i) = \sum r_i f(e_i) = \sum r_i m_i\). Thus,
    \[
        M = Rm_1 + \dots + Rm_n.
    \]
\end{proof}
\begin{corollary}{}{}
    Let \(N \leq M\) be an \(R\)-submodule. If \(M\) is finitely generated, then \(\nicefrac{M}{N}\) is also finitely generated.
\end{corollary}
\begin{proof}
    Let \(f: R^n \to M\) be a surjective \(R\)-module homomorphism. Then \(R^n \to M \to \nicefrac{M}{N}\) is a surjective \(R\)-module homomorphism.
\end{proof}
\begin{example}
    A submodule of finitely generated module need not be finitely generated.

    Let \(R\) be a non-Noetherian ring, and \(I \nsub R\) a non-finitely generated ideal. Then \(R\) is a finitely generated \(R\)-module, and \(I\) is a submodule which is not finitely generated.
\end{example}
\begin{remark}
    A submodule of a finitely generated module over a Noetherian ring is finitely generated. (Ex. Sheet 4)
\end{remark}
\begin{definition}{}{}
    Let \(M\) be an \(R\)-module.
    \begin{enumerate}
        \item An element \(m \in M\) is \textit{torsion} if \(0 \neq r \in R\) with \(r \cdot m = 0\).
        \item \(M\) is a \textit{torsion module} if every \(m \in M\) is torsion.
        \item \(M\) is \textit{torsion free} if \(0 \neq m \in M\) is not torsion.
    \end{enumerate}
\end{definition}
\begin{example}
    The torsion elements in a \(\mathbb{Z}\)-module (Abelian group) are the elements of finite order.

    Any \(F\)-module (vector space) will be torsion-free.
\end{example}