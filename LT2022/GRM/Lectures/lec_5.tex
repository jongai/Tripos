\lecture{5}{29 Jan. 12:00}{}
\subsection{p-groups and p-subgroups}
\leavevmode
\begin{definition}{}{}
    Let \(p\) be a prime. A finite group \(G\) is a p-group if \(\abs{G} = p^n, n\geq 1\).
\end{definition}
\begin{theorem}{}{}
    If \(G\) is a p-group, then \(Z(G) \neq \textbf{1} \).
\end{theorem}
\begin{proof}
    For \(g \in G\), we have by orbit-stabilizer theorem,
    \[\abs{\mathrm{ccl}_G(g)} \abs{C_G(g)} = \abs{G} = p^n.\] 
    So each conjugacy class has size a power of \(p\). Since \(G\) is a disjoint union of conjugacy classes, \(\abs{G}= \# (\text{conjugacy classes of size 1})\mod p\). It is easy to see that the conjugacy classes of size 1 are precisely the elements in the center of a group. That is, \(\abs{Z(G)} \equiv 0 \mod p  \), and hence \(Z(G) \neq \textbf{1}\).
\end{proof}
\begin{corollary}{}{}
    The only simple p-group is \(C_p\).
\end{corollary}
\begin{proof}
    Let \(G\) be a simple p-group. Since \(Z(G) \trianglelefteq G\), we have \(Z(G) \neq \textbf{1} \), so \(Z(G) = G\). That is \(G\) is Abelian. We know that the only Abelian simple groups are \(C_p\), so \(G = C_p\).
\end{proof}
\begin{corollary}{}{}
    Let \(G\) be a p-group of order \(p^n\). Then \(G\) has a subgroup of order \(p^r\) for all \(0 < r \leq n\).
\end{corollary}
\begin{proof}
    By \cref{le:compseries}, \(G\) has a composition series
    \[
        \textbf{1} \cong G_0 \trianglelefteq G_1 \trianglelefteq \ldots \trianglelefteq G_m \cong G
    \]
    with each quotient \(\nicefrac{G_{i+1}}{G_{i}}\) simple.
    
    Because \(G\) is a p-group, each of the quotients is a p-group. So \(\nicefrac{G_{i+1}}{G_i}\cong C_p\).

    Thus, \(\abs{G_i} =p^i\) for \(0\leq i \leq m = n\).
\end{proof}
\begin{lemma}{}{cyclicab}
    For \(G\) a group, if \(\nicefrac{G}{Z(G)}\) is cyclic, then \(G\) is Abelian. (so in fact \(\nicefrac{G}{Z(G)} = \textbf{1} \))
\end{lemma}
\begin{proof}
    Let \(gZ(G)\) be a generator for \(\nicefrac{G}{Z(G)}\), then each coset is of the form \(g^{r}Z(G)\) for some \(r \in \mathbb{Z}\).

    Thus, \(G = \{g^{r}z\mid r\in \mathbb{Z}, z\in Z(G)\}\). We check that two general elements in the group commute.
    \[
        g^{r_1}z_1 g^{r_2}z_2 = g^{r_1 + r_2}z_{1}z_2= g^{r_1 + r_2}z_{2}z_1=g^{r_2}z_2 g^{r_1}z_1.
    \]
    So \(G\) is Abelian.
\end{proof}
\begin{corollary}{}{}
    If \(\abs{G} =p^2\) then \(G\) is Abelian.
\end{corollary}
\begin{proof}
    We have three choices for the size of the center of the group. Noting that the center cannot be trivial for a p-group. So \(\abs{Z(G)} = p\) or \(\abs{Z(G)} = p^2\).

    If \(\abs{Z(G)} = p\), \(\abs{\nicefrac{G}{Z(G)}} = p\), apply \cref{le:cyclicab}, and we have a contradiction.

    If \(\abs{Z(G)} = p^2\), then \(Z(G) = G\) so \(G\) is Abelian.
\end{proof}
Note that this is not true for \(\abs{G} = p^3\).
\begin{theorem}{Sylow Theorems}{}
    Let \(G\) be a finite group of order \(p^a m\) where \(p\) is a prime with \(p \nmid m\). Then
    \begin{enumerate}
        \item The set \(\mathrm{Syl}_p(G) = \{P \leq G\mid \abs{P} = p^a\}\) of Sylow p-subgroups is non-empty.
        \item All elements of \(\mathrm{Syl}_p(G)\) are conjugates.
        \item If \(n_p\coloneqq \abs{\mathrm{Syl}_p(G)} \) satisfies \(n_p \equiv 1 \mod p\) and \(n_p \mid \abs{G} \) (and so \(n_p\mid m\)).
    \end{enumerate}
\end{theorem}
\begin{corollary}{}{}
    If \(n_p = 1\), then the unique Sylow p-subgroup is normal.
\end{corollary}
It is useful to show that the group of a certain order cannot be simple.
\begin{proof}
    Let \(g \in G\), and \(P \in \mathrm{Syl}_p(G)\). Then \(gPg^{-1} = P\) because \(n_p = 1\). Thus, \(P \trianglelefteq G\).
\end{proof}
\begin{example}
    Let \(\abs{G} = 1000 = 2^3\cdot 5^3\). Then \(n_5 \equiv 1 \mod 5\) and \(n_5 \mid 8\). So \(n_5 = 1\). Thus, the unique Sylow 5-subgroup is normal and of order \(125\). Hence, \(G\) is not simple.
\end{example}
\begin{example}
    Let \(\abs{G} = 132 = 2^2\cdot 3 \cdot 11\). We have \(n_{11}\equiv 1 \mod 11\) and \(n_{11}\mid 12\), so \(n_{11}=1\) or \(12\).

    Suppose that \(G\) is simple. Then \(n_{11} \neq 1\) (otherwise the Sylow 11-subgroup is normal) and hence \(n_{11} = 12\).

    Now we consider \(n_3 \equiv 1 \mod 3\) and \(n_3 \mid 44\). So \(n_3 = 1, 4, 22\). Similarly, \(n_3 \neq 1\).

    Suppose \(n_3 = 4\). Then letting \(G\) act on \(\mathrm{Syl}_{3}(G)\) by conjugation gives a group homomorphism \(\phi: G \to S_4\). So \(\ker (\phi) \trianglelefteq G \implies \ker(\phi)=1\) or \(G\). But by second Sylow Theorem, the action is transitive, so the kernel must be trivial, but \(132 > 24\), so \(\phi\) cannot possibly be an injection.

    Thus, \(n_{3} = 22\) and \(n_{11} = 12\). So \(G\) would have \(22\cdot(3-1)\) elements of order 3 and \(G\) has \(12 \cdot (11-1) = 120\) elements of order 11. But \(44 + 120 > 132 = \abs{G} \), contradiction.
\end{example}
\begin{proof}[Proof of Sylow Theorems]
    We have \(\abs{G} = p^a m\) where \(p\) is prime and \(p\nmid m\).
    \begin{enumerate}
        \item Let \(\Omega\) be set of all subsets of \(G\) of order \(p^a\). So
        \[
            \abs{\Omega} =\binom{p^a m}{p^a} = \frac{p^a m}{p^a} \frac{p^a m - 1}{p^a - 1}\cdots \frac{p^a m - p^a + 1}{1}
        \]
        for \(0 \leq k < p^a\), the number \(p^a m - k\), the numbers \(p^a m - k\) and \(p^a - k\) are divisible by the same powers of \(p\). So \(\abs{\Omega} \) is coprime to \(p\). Let \(G\) act on \(\abs{\Omega} \) by left multiplication. That is, for \(g \in G\) and \(X \in \Omega\),
        \[g * X = \{gx \mid x \in X\} \in \Omega.\]

        For any \(X \in \Omega\), we have
        \[
            \abs{G_X} \abs{\mathrm{orb}_G(X)} =\abs{G} =p^a m.
        \]
        Because \(\abs{\Omega} \) is coprime to \(p\). We can find some \(X\) such that \(\abs{\mathrm{orb}_G(X)} \) is coprime to \(p\). Thus, \(p^a \mid \abs{G_X} \).

        On the other hand, if \(g \in G\) and \(x \in X\), then \(g \in (gx^{-1})*X\), and we have
        \[
            G = \bigcup_{g\in G}g *X = \bigcup_{y\in \mathrm{orb}_G(x)}y.
        \]
        So \(\abs{G} \leq  \abs{\mathrm{orb}_G(X)} \abs{X}\implies \abs{G_x} = \frac{\abs{G}}{\abs{\mathrm{orb}_G(X)}} \leq \abs{X} =p^a \).

        Combining the two facts, we have \(\abs{G_X} =p^a\), i.e., \(G_x \in \mathrm{Syl}_p(G)\).
        \item We prove a stronger result.
        \begin{lemma}{}{sylowp}
            If \(P \in \mathrm{Syl}_p(G)\) and \(Q \leq G\) is a p-subgroup, then \(Q \leq gPg^{-1}\) for some \(g \in G\).
        \end{lemma}
        The lemma implies Second Sylow Theorem by taking \(Q\) a Sylow subgroup.

        Let \(Q\) act at the set of left cosets \(\nicefrac{G}{P}\) by left multiplication. That is, \(q * gP = (qg)P\). By orbit-stabilizer Theorem, each orbit has size dividing \(\abs{Q}\), so each orbit has size 1 or a power of \(p\).

        Since \(\abs{\nicefrac{G}{P}} = m\) is coprime to \(p\), there must exist an orbit of size 1. That is, there exists \(g \in G\) such that \(qgP = gP\) for all \(q\). So \(g^{-1}qg \in P\) for all \(q \in Q\). So \(Q \leq gPg^{-1}\).
        \item Let \(G\) act on \(\mathrm{Syl}_p(G)\) by conjugation. By Second Sylow Theorem, the action is transitive. Thus, orbit-stabilizer implies \(n_p = \abs{\mathrm{Syl}_p(G)}\mid \abs{G} \).

        Now let \(P \in \mathrm{Syl}_p(G)\). Then \(P\) act on \(\mathrm{Syl}_p(G)\) by conjugation. The orbits have size dividing \(\abs{P} = p^a\), the size is either 1 or a power of \(p\). So \(P\) is in an orbit of size 1.

        To show \(n_p \equiv 1 \mod p\), suffices to show that \(\{P\}\) is the only orbit of size one. If \(\{Q\}\) is another orbit of size 1, then \(P\) normalizes \(Q\). That is \(P \leq N_G(Q)\). Note that \(P,Q\) are both Sylow p-subgroups of \(N_G(Q)\).

        Thus, by Second Sylow Theorem, \(P\) and \(Q\) are conjugate in \(N_G(Q)\), hence equal since \(Q \trianglelefteq N_G(Q)\). Hence, \(P = Q\) and \(\{P\}\) is the unique orbit of size 1.
    \end{enumerate}
\end{proof}