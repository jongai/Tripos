\lecture{2}{24 Jan. 11:00}{More Examples}
\begin{enumerate}
    \setcounter{enumi}{5}
    \item Let \(S^1 = \{z \in \mathbb{C}\mid \left\vert z \right\vert =1\}\).
    
    The \textit{torus} \(S^1 \times S^1\), with the subspace topology from \(\mathbb{C}^2\) which is the product topology.
    \begin{lemma}
        The torus is a topological surface.
    \end{lemma}
    \begin{proof}
        We consider the map
        \begin{align*}
            \mathbb{R}^2 &\to S^1 \times S^1 \subseteq \mathbb{C}\times \mathbb{C},\\
            (s,t) &\mapsto (e^{2\pi is},e^{2\pi it}).
        \end{align*}
        Note that this induces map
        \[
            \begin{tikzcd}
                \mathbb{R}^2 \arrow[d,"q"] \arrow[r, "e"] & S^1\times S^1 \\ \nicefrac{\mathbb{R}^2}{\mathbb{Z}^2} \arrow[ru, "\hat{e}"]& 
            \end{tikzcd}.
        \]
        That is, on the equivalence relation on \(\mathbb{R}^2\) given by translating by \(\mathbb{Z}^2\), e is constant on equivalence classes, so it induces a map of sets \(\nicefrac{\mathbb{R}}{\mathbb{Z}^2}\). We can think of it as a quotient space equipped with the quotient topology.

        \(\nicefrac{\mathbb{R}}{\mathbb{Z}^2}\) is compact. A continuous map from a compact space to a Hausdorff space that is a bijection is a homeomorphism.

        Note we already know that \(S^1 \times S^1\) is compact and Hausdorff. (closed and bounded in \(\mathbb{R}^4\)).

        As for \(S^2 \to \mathbb{RP}^2\), pick \([p] = q(p), p \in \mathbb{R}\) and a small closed disk \(\overline{D}(p)\in \mathbb{R}^2\) such that for all \((n,m)\in \mathbb{Z}^2\), we have \(\overline{D}(p)\cap (\overline{D}(p) + (n,m))=\varnothing \). Then \(e\) and \(q\) restricted to the small closed disk is injective. They are bijective continuous maps from compact spaces to Hausdorff spaces, so they are homeomorphisms. Restricting it further to a smaller open disk, and we have a neighborhood of \([p]\) that is homeomorphic to a disk. Since \([p]\) is arbitrary, and \(S^1 \times S^1\) is a topological surface.
    \end{proof}
    \item Let \(P\) be a planar Euclidean polygon. Assume the edges are \textit{oriented} and paired, and for simplicity assume the Euclidean length for \(e, \hat{e}\) are equal if they are paired.

    If \(\{e, \hat{e}\}\) are paired edges, there is a unique isometry from \(e\) to \(\hat{e} \) respecting their orientations, say \(f_{e\hat{e}}: e \to \hat{e} \).

    There maps generate an equivalence relation on \(P\) where we identify \(x \in P\) with \(f_{e \hat{e} }(x)\) whenever \(x \in e\).
    \begin{lemma}
        \(\nicefrac{P}{\sim}\) (with the quotient topology) is a topological surface.
    \end{lemma}
    \begin{example}
        The torus as \(\nicefrac{[0,1]^2}{\sim}\). We consider three different kinds of points.

        If \(p\) is in the interior. We can find a small enough neighborhood that is injective, and again by topological inverse function theorem, that small enough disk is homeomorphic to an open disk.

        If \(p\) is on the edge. Say \(p = (0,y) \sim (1,y)\) and \(\delta >0\) is small enough such that a half disk of radius \(\delta\) does not touch vertices. Define a map from the union of the half-disks to \(B(0,\delta) \subseteq \mathbb{R}^2\) by \((x,y) \mapsto x, y - y_0\) and \((x,y)\mapsto (x-1,y-y_0)\) on each part of the half-disk. Recall if \(X = A \cup B\) is a union of closed subspaces, and we have continuous maps \(f: A\to Y, g: B\to Y\), and \(\left.f\right|_{A\cap B} = \left.g\right|_{A\cap B}\), they define a continuous map from \(X\) to \(Y\).
    \end{example}
\end{enumerate}