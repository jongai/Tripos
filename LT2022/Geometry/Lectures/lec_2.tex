\lecture{2}{24 Jan. 11:00}{More Examples}
\begin{enumerate}
    \setcounter{enumi}{5}
    \item Let \(S^1 = \{z \in \mathbb{C}\mid \left\vert z \right\vert =1\}\).
    
    The \textit{torus} \(S^1 \times S^1\)
    \item Let \(P\) be a planar Euclidean polygon. Assume the edges are \textit{oriented} and paired, and for simplicity assume the Euclidean length for \(e, \hat{e}\) are equal if they are paired.

    If \(\{e, \hat{e}\}\) are paired edges, there is a unique isometry from \(e\) to \(\hat{e} \) respecting their orientations, say \(f_{e\hat{e}}: e \to \hat{e} \).

    There maps generate an equivalence relation on \(P\) where we identify \(x \in P\) with \(f_{e \hat{e} }(x)\) whenever \(x \in e\).
    \begin{lemma}
        \(\nicefrac{P}{\sim}\) (with the quotient topology) is a topological surface.
    \end{lemma}
    \begin{example}
        The torus as \(\nicefrac{[0,1]^2}{\sim}\). We consider three different kinds of points.

        If \(p\) is in the interior. We can find a small enough neighborhood that is injective, and again by topological inverse function theorem, that small enough disk is homeomorphic to an open disk.

        If \(p\) is on the edge. Say \(p = (0,y) \sim (1,y)\) and \(\delta >0\) is small enough such that a half disk of radius \(\delta\) does not touch vertices. Define a map from the union of the half-disks to \(B(0,\delta) \subseteq \mathbb{R}^2\) by \((x,y) \mapsto x, y - y_0\) and \((x,y)\mapsto (x-1,y-y_0)\) on each part of the half-disk. If \(X = A \cup B\) is a union then they define a continuous map from \(X\) to \(Y\).
    \end{example}
\end{enumerate}