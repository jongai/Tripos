\lecture{3}{26 Jan. 11:00}{}
\begin{enumerate}
    \setcounter{enumi}{6}
    \item Given topological surfaces \(\Sigma_1\), \(\Sigma_2\), I can remove an open disc from each and glue the resulting circles.

    Explicitly, I take \(\Sigma_{1}\setminus D_1\coprod \Sigma_2 \setminus D_2\) and impose a quotient relation. 
    \[
        \theta \in \partial D_1 \sim \theta \in \partial D_2
    \]
    where \(\theta\) parametrizes \(S^1 = \partial D_i\).

    The result \(\Sigma_1 \# \Sigma_2\) is called the \textit{connect sum} of \(\Sigma_1\) and \(\Sigma_2\). (In principle this depends on any choices, suppressed from the notation).

    \lemma{The connect sum \(\Sigma_1\#\Sigma_2\)} is a topological surface.
\end{enumerate}