\lecture{4}{28 Jan. 2022}{}
\leavevmode
\begin{definition}
    A \textit{subdivision} of a compact topological surface \(\Sigma\) comprises
    \begin{enumerate}
        \item a finite set \(V \in \Sigma\) of vertices;
        \item a finite collection \(E = \{e_i: [0,1] \to \Sigma\}_{i\in E}\) of \textit{edges} such that
        \begin{itemize}
            \item for all \(i\), \(e_i\) is a continuous injection on its interior and \(e_i^{-1}V = \{0,1\}\),
            \item \(e_i\) and \(e_j\) have disjoint image except perhaps at their endpoints in \(V\);
        \end{itemize}
        \item such that each connected component of \(\Sigma \setminus (\cup e_i[0,1]\cup V)\) is homeomorphic to an open disc called a \textit{face}. (So the closure of a face has boundary \(\overline{F}\setminus F\) lying in \(E\cup V\)).
    \end{enumerate}

    A subdivision is a \textit{triangulation} if each \textit{closed} face (closure of a face) contains exactly 3 edges, and two closed faces are disjoint or meet in exactly one edge (or possible just one vertex).
\end{definition}
\begin{example}
    \leavevmode
    \begin{itemize}
        \item A cube displays a subdivision of \(S^2\).
        \item A tetrahedron displays a triangulation of \(S^2\).
        \item We can describe subdivisions using planar polygons. For example, the normal depiction of \(T^2\) has 1 vertex, 2 edges and 1 face.
        \item By our definition, we can have degenerate subdivisions like a subdivision of \(S^2\) with 1 vertex, 0 edge and 1 face.
    \end{itemize}
\end{example}
\begin{definition}
    The \textit{Euler characteristic} of a subdivision is the number \(\# V - \# E + \# F\).
\end{definition}
\begin{theorem}
    \leavevmode
    \begin{enumerate}
        \item Every compact topological surface admits subdivision, and indeed triangulation.
        \item The Euler characteristic denoted \(\chi (\Sigma)\) does not depend on the choice of subdivision and defines a topological invariant of the surface. (depends only on the homeomorphic type of \(\Sigma\))
    \end{enumerate}
\end{theorem}
\begin{example}
    \leavevmode
    \begin{enumerate}
        \item \(\chi(S^2) = 2\);
        \item \(\chi(T^2) = 0\);
        \item If \(\Sigma_1\) and \(\Sigma_{2}\) are compact topological surfaces, we can form \(\Sigma_1 \# \Sigma_2\) by removing an open disc \(D_i \subseteq \Sigma_i\) which is a face of a triangulation, and gluing the boundary circles \(\partial D_i\) by a homeomorphism taking edges to edges.
        
        The resulting surface \(\Sigma_1 \# \Sigma_2\) inherits a subdivision, and we have
        \[
            \chi (\Sigma_1 \# \Sigma_2) = \chi(\Sigma_1) + \chi(\Sigma_{2}) - 2.
        \]

        In particular, if we take a surface with \(g\) holes, which is \(\Sigma_g = \#^{g}_{i=1}T^2\), then \(\chi(\Sigma_g) = 2 - 2g\); \(g\) is called the \textit{genus} of \(\Sigma\).
    \end{enumerate}
\end{example}
\begin{remark}
    \leavevmode
    \begin{enumerate}
        \item Part 1 is hard to prove.
        \item You should believe Part 2 since I can turn a subdivision into a triangulation, and I can relate triangulations by local moves. It is easy to check both subdividing and switch diagonal preserves \(\chi\).

        But it is hard to rigorize this result, and you learn essentially nothing from the combinatorial proof. A much cleaner approach is developed in Part II algebraic topology.
    \end{enumerate}
\end{remark}
Recalled if \(\Sigma\) is a topological surface, every \(p \in \Sigma\) lies in an open neighborhood \(p \in u \subseteq \Sigma\) with \(u\) homeomorphic to an open disc (or equivalently to \(\mathbb{R}^2\)).
\begin{definition}
    A pair \((u, \phi)\) where \(u\) is an open set in \(\Sigma\) and \(\phi : u \to V \) an open set in \(\mathbb{R}^2\) which is a homeomorphism is called a \textit{chart} for \(\Sigma\). (If \(p\in u\) we might say "a chart for \(\Sigma\) at \(p\)")

    A collection \(\{(u_i, \phi_1)_{i\in I}\}\) of charts such that \(\cup_{i\in I} = \Sigma\) is called an \textit{atlas} for \(\Sigma\).

    The inverse \(\sigma = \phi^{-1}:v \to u \in \Sigma\) is called a \textit{local parametrization} for \(\Sigma\).
\end{definition}
\begin{example}
    \leavevmode
    \begin{enumerate}
        \item If \(Z \subseteq \mathbb{R}^2\) is a closed set, \(\mathbb{R}^2 \setminus Z\) is a topological surface with an atlas with one chart that is \((\mathbb{R}^2,\mathbf{id})\).
        \item For \(S^2\), we have an atlas with 2 charts, the 2 stereographic projections.
    \end{enumerate}
\end{example}
\begin{definition}
    Let \((u_i, \phi_i)\) be charts containing \(p\), the map \[\left.\phi_2 \circ \phi_1^{-1} \right|_{\phi_1 (u_1\cap u_2)}\] is called the \textit{transition map} between the charts. This is a homeomorphism of open sets in \(\mathbb{R}^2\)
\end{definition}
Recall that if \(V \subseteq \mathbb{R}^2\) and \(V' \subseteq \mathbb{R}^m\) open subsets, then a map \(f: V \to V'\) is called \textit{smooth} if it is infinitely differentiable; that is, it has partial derivatives of all orders of all variables.

If \(n = m\), a homeomorphism \(f: V \to V'\) is called a \textit{diffeomorphism} if it is smooth and has smooth inverse.

\begin{definition}
    An \textit{abstract smooth surface} \(\Sigma\) is a topological surface with an atlas of charts \(\{(u_i, \phi_i)\}_{i\in I}\) such that all transition maps
    \[
        \phi_i \circ \phi_j^{-1}: \phi_j(u_i\cap u_j) \to \phi_i(u_i \cap u_j)
    \]
    are diffeomorphisms of open sets in \(\mathbb{R}^2\).
\end{definition}
\begin{note}
    It would not make sense to ask for the \(\phi_i\) themselves to be smooth, as \(\Sigma\) is just a topological space.
\end{note}
\begin{example}[Example Sheet 2]
    The atlas of 2 charts with stereographic projections gives \(S^2\) the structure of an abstract smooth surface.
\end{example}
\begin{example}
    The torus \(T^2 = \nicefrac{\mathbb{R}^2}{\mathbb{Z}^2}\) is also an abstract smooth surface. Recall we obtained charts from the inverse of the projection restricted to small discs in \(\mathbb{R}^2\), the ones that are disjoint from translation by \(\mathbb{Z}^2 \setminus \{(0,0)\}\).

    The transition maps are the translations, so \(T^2\) inherits the structure of an abstract smooth surface.
\end{example}