\lecture{1}{21 Jan. 11:00}{Introduction}
\section{Surfaces}
\subsection{Topological Surfaces}
We start with some definitions.
\begin{definition}
    A \textit{topological surface} is a topological space \(\Sigma\) such that
    \begin{enumerate}
        \item \textbf{T1:} \(\forall p \in \Sigma\) there is an open neighborhood \(p \in U \subseteq \Sigma\) such that \(U\) is homeomorphic to \(\mathbb{R}^2\) , or a disc \(D^2 \subseteq \mathbb{R}^2\) with its usual Euclidean topology.
        \item \textbf{T2:} \(\Sigma\) is Hausdorff and second countable.
    \end{enumerate}
\end{definition}
\begin{remark}
    We have the following remarks.
    \begin{enumerate}
        \item \(\mathbb{R}\cong D(0,1)\), so homeomorphic to a disc is enough as stated in the definition.
        \item A space \(X\) is \textit{Hausdorff} if for \(p \neq q \in X\), there exists disjoint open sets \(p \in U\) and \(q \in V\) in \(X\).
        \item A space \(X\) is \textit{second countable} if it has a countable base i.e. \(\exists \{u_i\}_{i \in \mathbb{N}}\) open sets s.t. every open set is a union of some \(u\).
        \item \textbf{T1} is the point and \textbf{T2} is for technical honesty.
        \item If \(X\) is Hausdorff/ second countable, so are subspaces of \(X\). In particular, Euclidean space has these properties. (For second countable, consider open balls with rational center and rational radius).
    \end{enumerate}
\end{remark}
\begin{eg}
    Here we present some examples of topological surfaces.
    \begin{enumerate}
        \item \(\mathbb{R}^2\), the plane.
        \item Any open subset of \(\mathbb{R}^2\), i.e. \(\mathbb{R}^2 \setminus Z\) where \(Z\) is closed:
        \begin{itemize}
            \item \(Z = \{0\}\),
            \item \(Z = \{(0,0)\}\cup \{(0,\frac{1}{n}\mid n = 1,2,3, \ldots )\}\).
        \end{itemize}
        \item Graphs:

        Let \(f: \mathbb{R}^2 \to R\) be a continuous function. The graph \(\Gamma_f = \{(x,y,f(x, y))\mid (x,y) \in \mathbb{R}^2\} \subseteq \mathbb{R}^2\) (subspace topology).

        Recall that if \(X, Y\) are spaces, the product topology on \(X \times Y\) has basic open sets \(U \times V\) with \(U\) open and \(V\) open.

        It has the feature that \(f: Z \to  X \times Y\) is continuous if and open if the two projective maps are continuous.
        \todo{Commu- tative diagram}

        Application: \(\Gamma_f \subseteq X \times Y \), if \(f: X \to Y\) is continuous, if homeomorphic to \(X\).

        So \(\Gamma_f \cong \mathbb{R}^2\) for any \(f: \mathbb{R}^2 \to \mathbb{R}\) that is continuous, so \(\Gamma_f\) is a topological surface.

        \begin{note}
            As a topological surface, \(\Gamma_f\) is independent of \(f\), but later on as a geometric object, it will reflect features of \(f\). 
        \end{note}
        \item The sphere (subspace topology):
        \[
            S^2 = \{(x,y,z) \in \mathbb{R}^3 \mid x^2 + y^2 + z^2 = 1\}.
        \]
        \todo{Stereo- graphic projection graph}
        Stereographic projection
        \begin{align*}
            \pi_+: S^2 \setminus \{(0,0,1)\} &\to \mathbb{R}^2\\
            (x,y,z) &\mapsto (\frac{x}{1-z}, \frac{y}{1 - z})
        \end{align*}
        \begin{note}
            The map is continuous and has an inverse,
            \todo{Explicit formula for inverse}
            \(\pi_+\) is a continuous bijection with continuous inverse, and hence a homeomorphism.

            Stereographic projection from the South Pole is also a homeomorphism from \(S^2 \setminus \{(0,0,1)\} \to \mathbb{R}^2\). 
        \end{note}
        So \(S^2\) is a topological surface:

        \(\forall p \in S^2\), either \(p\) lies in the domain of \(\pi_+\) or of \(\pi_-\) (or both) and so it lies in an open set homeomorphic to \(\mathbb{R}^2\). (And Hausdorff and second countable from \(\mathbb{R}^2\)).
        \begin{remark}
            \(S^2\) has a global property as it is compact as a topological space, since it is a closed bounded set in \(\mathbb{R}^3\) .
        \end{remark}
        \item The real projective place:
        
        The group \(\mathbb{Z} / 2\) acts on \(S^2\) by homeomorphism via the \textit{antipodal map} \(a: S^2 \to S^2\).
        \[
            a(x,y,z) = (-x,-y,-t).
        \]
        i.e. There exists a homomorphism \(\mathbb{Z} / 2\mathbb{Z} \to  \mathrm{Homeo} (S^2)\), such that it maps the non-identity element to the antipodal map.

        \begin{definition}
            The \textit{real projective plane} is the quotient space of \(S^2\) given by identifying every point with its antipodal image:
            \[
                \mathbb{RP}^2 = \nicefrac{S^2}{\mathbb{Z}/ 2\mathbb{Z}}.
            \]
        \end{definition}

        \begin{lemma}
            As a set, \(\mathbb{RP}^2\) is naturally in bijection with the set of straight lines in \(\mathbb{R}^3\) through the origin.
        \end{lemma}
        \begin{proof}
            \todo{Graph of the sphere} Any straight line that goes through the origin meets the sphere exactly twice, and any such pair determines a straight line.
        \end{proof}
        \begin{lemma}
            \(\mathbb{RP}^2\) is a topological surface.
        \end{lemma}
        \begin{proof}
            We check that it is Hausdorff:

            Recall if \(X\) is a space and \(q: X \to Y\) is a quotient map, \(V \subseteq Y\) is open \(\iff q^{-1}V \subseteq X\) open. 

            \todo{More balls}
            If \([p],[q] \in \mathbb{RP}^2\), then \(\pm p, \pm q \in S^2\) are distinct antipodal pairs. Take small open discs around \(p,q\) and their antipodal images, as in the picture.
        \end{proof}
    \end{enumerate}
\end{eg}