\lecture{1}{21 Jan. 10:00}{Introduction}
\subsubsection*{Textbook}
\begin{enumerate}
    \item \textit{Complex Analysis} -Ahlfors
    \item \textit{Real and Complex Analysis} -Rudin
\end{enumerate}
\section{Analytic Functions}
\begin{notation}
    We have the following notations.
    \begin{itemize}
        \item \(\mathbb{C}\): complex plane.
        \item \(\bar{z}\): complex conjugate of \(z \in \mathbb{C}\).
        \item \(\left\vert z \right\vert \): modulus of \(z \in \mathbb{C}\).

        Note that \(d(z, w) = \left\vert z - w \right\vert \) defines a metric on \(\mathbb{C}\) (the usual or standard metric).
        \item \(D(a, r) = \{z\in \mathbb{C}\mid \left\vert z - a \right\vert <r\}\) is the open disk with centre \(a\) and radius \(r\).
    \end{itemize}
\end{notation}
\begin{definition}
    A subset \(U \in \mathbb{C}\) is \textit{open} if it is open with respect to the above metric, i.e. if for every \(a \in U\) there exists \(r > 0\) 
\end{definition}
All topological notions we will use (limits, continuity of function, compactness)

This course is about complex valued functions of a single complex variables. i.e. function
\[
    f:A\to \mathbb{C},\quad \text{where}~ A\subset \mathbb{C}.
\]
\begin{remark}
    Identifying \(\mathbb{C}\) with \(\mathbb{R}^2\) in the usual way, we.
\end{remark}

Almost exclusively we'll focus on differentiable function \(f\). Let's start by recalling continuity.

\begin{definition}
    The function \(f\) is \textit{continuous} at a point \(w \in A\) if \(\forall \epsilon > 0, \exists \delta > 0\) such that
    \[
        z \in A, \left\vert z - w \right\vert <\delta \implies \left\vert f(z)-f(w) < \epsilon \right\vert.
    \]
    This is the same as saying that \(\lim\limits_{z \to w} f(z)=f(w)\).
\end{definition}
\begin{remark}
    \(f\) is continuous if and only if \(u,v\) is continuous with respect to the metric on \(\mathbb{R}^2\).
\end{remark}

\subsection{Complex Differentiation}
We extend the definition of derivatives in this section.
\begin{definition}
    \begin{enumerate}
        We have some definitions of differentiation in \(\mathbb{C}\) .
        \item \(f\) is \textit{differentiable} at \(w \in U\) if the limit
        \[
            f'(w)=\lim\limits_{z \to w}\frac{f(z)-f(w)}{z-w}.
        \]
        exists as a complex numbers. \(f'(w)\) is called the derivative of \(f\) at \(w\).
        \item \(f\) is \textit{holomorphic at \(w \in U\)} if there is \(\epsilon>0\) such that \(D(w, \epsilon) \subset U\) and \(f\) is differentiable at every point in \(D(w, \epsilon)\). (Locally differentiable everywhere).
        \item \(f\)  is \textit{holomorphic in \(U\)} if \(f\) is holomorphic at every point in \(U\), or equivalently, \(f\) is differentiable at every point in \(U\).
    \end{enumerate}
\end{definition}
\begin{remark}
    Sometimes we use "analytic" to mean holomorphic. But precisely "analytic" means that the function has a convergent Taylor series at the point. It means the function is very nice behaving. However, in \(\mathbb{C}\), holomorphic if and only if analytic which we will prove later on.
\end{remark}
Usual rules of differentiation of real functions of a real variable hold for complex functions. Derivatives of sums, products, quotients of functions are obtained in the same way (as can easily be checked).

The chain rule for composite functions also holds:
\[
    f:U\to\mathbb{C},~g:V\to \mathbb{C},~f(U)\subset V,~h=g\circ f:U \to \mathbb{C}.
\]
If \(f\) is differentiable at \(w \in U\) \(g\) is differentiable at \(f(w)\), then \(h\) is differentiable at \(w\) and \(h'(w)=g'(f(w))f'(w)\).

\begin{problem}
    Write \(f(z) = u(x,y)+iv(x,y),z=x+iy\). Is differentiability of \(f\) at a point \(w = c + id \in U\) is the same as differentiability of \(u\) and \(v\) at \((c,d)\)?
\end{problem}

Recall from Analysis \& Topology that \(u: U \to \mathbb{R}\) is differentiable at \((c,d)\in U\) if there is a "good affine approximation of \(u\) at \((c,d)\)." i.e. if there is a linear transformation
\[
    \lim\limits_{(x,y) \to (c,d)} \frac{u(x,y) - (u(c,d) + L(x-c,y-d))}{\sqrt{(x-c)^2 + (y-d)^2}} = 0.
\]

The answer for the above question though is no. The theorem below characterizes differentiability of \(f\) in terms of \(u\) and \(v\).

\begin{theorem}[Cauchy-Riemann equations]
    The function \(f = u + iv:U\to \mathbb{C}\) is differentiable at \(w = c + id \in U \iff  u,v: U \to \mathbb{R}\) are differentiable at \((c,d) \in U\) and \(u, v\) satisfy the Cauchy-Riemann equations at \((c,d)\), i.e.
    \begin{align*}
        \frac{\partial u}{\partial x} &= \frac{\partial v}{\partial y}\\
        \frac{\partial u}{\partial y} &= - \frac{\partial v}{\partial x}\quad\text{at}\quad(c,d).
    \end{align*}
    If \(f\) is differentiable at \(w = c + id\), then \(f'(w) = \frac{\partial u}{\partial x} (c,d) + i \frac{\partial v}{\partial x} (c,d)\) (and three other expressions following from the Cauchy-Riemann equations).
\end{theorem}

\begin{proof}
    \(f\) is differentiable at \(w\) with derivative \(f'(w) = p + iq\)
    \begin{gather*}
        \iff \lim_{z \to w} \frac{f(z) - f(w)}{z - w} = p + iq\\
        \iff \lim_{z \to w} \frac{f(z) - f(w) - (z - w)(p + iq)}{\left\vert z-w \right\vert} = 0\\
    \end{gather*}
    Writing \(f = u + iv\) and separating real and imaginary parts, we have
    \begin{align*}
        \iff &\lim\limits_{(x, y) \to (c, d)} \frac{u(x,y) - u(c,d) - p(x - c) + q(y - d)}{\sqrt{(x - c)^2 + (y - d)^2} } = 0,\\
        \text{and}~&\lim\limits_{(x, y)) \to (c,d)} \frac{v(x,y) - v(c,d) - q(y - d) - p(x - c)}{\sqrt{(x - c)^2 + (y - d)^2} } = 0.
    \end{align*}

    \(\iff \) \(u\) is differentiable at \((c,d)\) with \(Du(c,d)(x,y) = px - qy\), and \(v\) is differentiable at \((c,d)\) with \(Dv(c,d)(x,y) = qx + py\).

    \(\iff \) \(u,v\) are differentiable at \((c,d)\) and \(u_x (c,d) = p = v_y(c,d)\), and \(u_y(c,d) = -q = -v_x(c,d)\) i.e. Cauchy-Riemann equations hold at \((c,d)\).

    We also get from the above that if \(f\) is differentiable at \(w\), then \(f'(w) = p + iq = u_x(c,d) + iv_x(c,d)\).
\end{proof}
\begin{remark}
    \(u, v\) satisfying the Cauchy-Riemann equations at a point does not guarantee differentiability of \(f\). (See ex. sheet 1).
\end{remark}

\begin{remark}
    If we just want to show:

    Differentiability of \(f\) at \(w = c + id\) implies the partial derivatives \(u_x, u_y, v_x, v_y\) exist and satisfy the Cauchy-Riemann equations, the proof can be much simpler.
\end{remark}