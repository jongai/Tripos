\lecture{2}{24 Jan. 10:00}{}
\leavevmode
\begin{example}
    Here we give a non-example of Cauchy-Riemann equations. When \(f(z) = \overline{z} = x - iy\). For this, \(u(x,y) = x, v(x,y) = -y\), so \(u_x = 1, v_y = 0, v_x = 0, v_y = -1\). So Cauchy-Riemann equations do not hold anywhere, and \(f\) is not differentiable at any point.
\end{example}
\begin{corollary}
    Let \(f = u + iv: U \to \mathbb{C}\). If \(u, v\) have continuous partial derivatives at a point \((c,d) \in U\) and satisfy the Cauchy-Riemann equations at \((c,d)\), then \(f\) is differentiable at \(w = c + id\).

    In particular, if \(u, v\) are \(C^1\) functions on \(U\) satisfying the Cauchy-Riemann equations in \(U\), then \(f\) is holomorphic in \(U\). 
\end{corollary}
\begin{proof}
    Continuity of partial derivatives of \(u\) implies that \(u\) is differentiable, and similarly for \(v\) from Analysis \& Topology. So the corollary follows from Theorem 1.1.
\end{proof}
\begin{note}
    Quite remarkably, we can relax the requirement of continuity of partial derivatives of \(u,v\) in \(U\) to just continuity of \(u, v\) in \(U\). Thus, if \(f = u + iv\) is defined on an open set \(U\) and is continuous in \(U\), and if \(u, v\) satisfy the Cauchy-Riemann equations in \(U\), then \(f\) is holomorphic in \(U\). This is called the \textbf{Looman-Menchoff} theorem. It is quite non-trivial to prove.
\end{note}
\begin{remark}
    Complex differentiability is much more restrictive than real differentiability of real and imaginary parts (because of the additional requirement that C-R equations must hold). This leads to some surprising theorems compared to the real case. For instance:
    \begin{enumerate}
        \item if \(f: \mathbb{C} \to \mathbb{C}\) is holomorphic and bounded, then \(f\) is constant! (Liouville's Theorem)
        \item if \(f: U\to \mathbb{C}\) is holomorphic, then \( f\) is automatically infinitely differentiable on \(U\).
    \end{enumerate}
\end{remark}
\begin{note}
    (2) implies partial derivatives of \(u, v\) of all orders exists. So we can differentiate C-R equations to get
    \begin{align*}
        (u_x)_x = (v_y)_x &\implies u_{xx} = v_{yx}\\
        (u_y)_y = (-v_x)_y &\implies u_{yy} = -v_{xy}.
    \end{align*}
    Since \(v_{yx} = v_{xy}\) this gives, for \((x,y) \in U\),
    \[
        \Delta u = u_{xx} + u_{yy} = 1.
    \]
    Similarly, \(\Delta u = 0\) in \(U\).

    That is, real and imaginary parts of a holomorphic function are harmonic.
\end{note}
For the next corollary, we need the following,
\begin{definition}
    \leavevmode
    \begin{itemize}
        \item A \textit{curve} is a continuous map \(\gamma: [a,b] \to \mathbb{C}\), where \([a,b]\subseteq \mathbb{R}\) is a closed interval. We say that \(\gamma\) is a \textit{\(C^1\) curve} if \(\gamma'\) exists and is continuous on \([a,]\). (If \(\gamma(t) = x(t) + y(t)\) then \(\gamma'(t) = x'(t) + iy'(t)\); at the end points, \(\gamma'\) is the one-sided derivative.)
        \item An open set \(U \subseteq \mathbb{C}\) is \textit{path connected} if for any two points \(z,w \in U\), there is a curve \(\gamma: [0,1] \to U\) such that \(\gamma(0) = z\) and \(\gamma(1) = w\).
        \item A \textit{domain} is a non-empty, open, path connected subset of \(C\).
    \end{itemize}
\end{definition}
\begin{corollary}
    If \(U\subseteq C\) is a domain, \(f: U\to \mathbb{C}\) is holomorphic with \(f'(z) = 0\) for every \(z \in U\), then \(f\) is constant.
\end{corollary}
\begin{proof}
    Write \(f = u + iv\). By the Cauchy-Riemann equations, \(f'=0 \implies Du = 0, Dv=0\) in \(U\). Since \(U\) is a domain, this means (by a theorem from Analysis and Topology) that \(u,v\) are constants; that is, \(f\) is constant.
\end{proof}
So far we've only seen very few explicit holomorphic functions (namely, polynomials on \(\mathbb{C}\) and rational functions on their domains). We'd like to generate more. We do this by looking at power series.

\subsection{Power Series}
Recall (from IA Analysis)
\begin{theorem}[Radius of Convergence]
    If \((c_n)\) is a sequence of complex numbers then there is a unique number \(R \in [0, \infty]\) such that the power series
    \[
        \sum\limits_{i=1}^{\infty} c_n(z-a)^n, z,a\in\mathbb{C}
    \]
    converges absolutely if \(\left\vert z - a \right\vert <R\) and diverges if \(\left\vert z - a \right\vert > R\). If \(0<r<R\), then the series converges uniformly (with respect to the variable \(z\)) on the compact disk \(\overline{D(a, r)} = \{z\in\mathbb{C}\mid \left\vert z - a \right\vert \leq r\}\).
\end{theorem}
\(R\) is called the radius of convergence of the power series. Note that there is no claim about convergence when \(\left\vert z - a \right\vert = R, R>0\). There are various expressions for \(R\). For example,
\begin{itemize}
    \item \(R = \sup \{r \geq  0 \mid \left\vert c_n \right\vert r^n \to 0 \text{ as } n \to \infty\}\),
    \item \(\frac{1}{R} = \limsup_{n\to \infty}\left\vert c_n \right\vert^{\frac{1}{n}}\).
\end{itemize}
\begin{theorem}
    Let \(\sum\limits_{n=0}^{\infty} c_n(z-a)^n\) be a power series with radius of convergence equal to \(R > 0\). Fix \(a \in \mathbb{C}\), and define \(f: D(a, R) \to \mathbb{C}\) by \(f(z) = \sum\limits_{n=0}^{\infty} (z-a)^n\). Then:
    \begin{enumerate}
        \item \(f\) is holomorphic on \(D(a,R)\);
        \item the derived series also has radius of convergence equal to \(R\), and \(f'(x)\); 
        \item \(f\) has derivatives of all orders on \(D(a,R))\), and \(c_n = \frac{f^{(n)}(a)}{n!}\);
        \item if \(f\) vanishes on \(D(a,\epsilon)\) for some \(\epsilon >0\), then \(f = 0\) on \(D(a,R)\).
    \end{enumerate}
\end{theorem}
\begin{remark}
    \begin{enumerate}
    \item This theorem provides a way to generate a large class of holomorphic
    functions on a disk.
    \item Later we will show that every holomorphic function is locally given by a power series (Taylor series theorem). Once we have that, part (iii) above gives that holomorphic functions are automatically infinitely differentiable in their domain. (regardless of what the domain looks like)
    \end{enumerate}
\end{remark}
\begin{proof}
    \todo{finish}
    \begin{enumerate}
        \item[1,2.]
        \item[3.] Repeated apply (2). The formula \(c_n = \frac{f^(n)(a)}{n!}\) follows by differentiating the series \(n\) times and setting \(z = a\).
        \item[4.]
    \end{enumerate}
\end{proof}