\lecture{1}{20 Jan. 10:00}{Introduction}
\section{Basic Concepts}
Books
\begin{enumerate}
    \item Classical Mechanics - Douglas (more examples)
    \item Classical Mechanics - Tom Kibblet (more chatty)
    \item Lecture Notes - David Tong
\end{enumerate}
\subsection{Newtonian Mechanics}
A \textit{particle} is an object of insignificant size. For now, its only attribute is its position.

For large objects, we take the center of mass to define the position and treat them like a particle.

To describe the position, we pick a \textit{reference frame}: a choice of origin and 3 coordinate axes. With respect to this frame, a particle sweep out a \textit{trajectory} $\mathbf{x} (t).$ (sometimes, we may write $\mathbf{r} (t)$).

\begin{figure}[H]
    \centering
    \incfig{va}
    \caption{position and acceleration of a particle}
    \label{fig:va}
\end{figure}

Given two vector functions $\mathbf{f}(t)$ and $\mathbf{g} (t)$, $\frac{d}{dt}(\mathbf{f} \cdot \mathbf{g} ) = \frac{d \mathbf{g} }{t}\cdot \mathbf{g}  + \mathbf{f}  \cdot \frac{d \mathbf{g} }{dt}$ and $\frac{d}{dt}(\mathbf{f} \times \mathbf{g} ) = \frac{d \mathbf{g} }{t}\times \mathbf{g}  + \mathbf{f}  \times \frac{d \mathbf{g} }{dt}$.

\subsection{Newtonian Laws of Motion}
The framework of Newtonian mechanics rely on these axioms, known as \textit{Newton's Laws}:
\begin{definition}[Newton's Laws]
    The following are true (for inertial frames):
\begin{itemize}
    \item \textbf{N1}: Left alone, a particle moves with constant velocity.
    \item \textbf{N2}: The rate of change of momentum is proportional to the force.
    \item \textbf{N3}: Every action has an equal and opposite reaction.
\end{itemize}
\end{definition}
\subsection{Inertial Frames and The First Law}
For many reference frames, \textbf{N1} isn't true! It only holds for frames that are not themselves accelerating. Such frames are called \textit{inertial frames}:
\begin{definition}[Inertial Frame]
    In an inertial frame, $\ddot{\mathbf{x} }=0$ when left alone.
\end{definition}

A better framing of the 1st law is (N1' inertial frames exist).

For most purposes, this room approximates an inertial frame.

\subsection{Galilean Relativity}
Inertial frames are not unique. Given an inertial frame $S$, in which a particle has coordinates $\mathbf{x} $, we can construct another inertial frame $S^\prime $ in which the coordinates of the particle are given by $\mathbf{x} ^\prime $.
\begin{enumerate}
    \item Translations: $\mathbf{x} ^\prime =\mathbf{x}  + \mathbf{a} $, where $a$ is a constant.
    \item Rotations: $\mathbf{x} ^\prime = R\mathbf{x} $, where $R$ is a $3\times 3$ matrix with $R^{T}R=I$.
    \item (Galilean) Boost: $\mathbf{x}^\prime =\mathbf{x}  + \mathbf{v} t$.
\end{enumerate}

For each of these, if there is no force on a particle.
\(\ddot{\mathbf{x}} = 0 \implies \ddot{\textbf{x}}^\prime =0 \) 

The \textit{Galilean principle of relativity} tells us that the laws of physics are the same
\begin{enumerate}
    \item At every point in space.
    \item No matter which direction you face.
    \item No matter what constant velocity you move at.
    \item At all moments in time.
\end{enumerate}
The above are experimentally tested facts.

There is no such thing as "absolutely stationary", but notice that acceleration is absolute. You don't have to accelerate relative to something.