\RequirePackage[l2tabu, orthodox]{nag}
\documentclass[parskip=full-]{scrartcl}
% Core Math Packages
\usepackage{amssymb}
\usepackage{amsthm}
\usepackage{mathtools}

% Core Typesetting Packages
\usepackage{microtype}
\usepackage[utf8]{inputenc}
\usepackage[T1]{fontenc}
\usepackage[sc]{mathpazo}
\usepackage[english]{isodate}

\setkomafont{title}{\scshape\rmfamily}
\setkomafont{section}{\LARGE\rmfamily}
\setkomafont{subsection}{\large\rmfamily}

% Useful Packages
\usepackage{tcolorbox}
\usepackage{nicefrac}
\usepackage{braket}
\usepackage[all,error]{onlyamsmath}
\usepackage{subcaption}
\usepackage{bm}

\usepackage[hyphens]{url}
\usepackage{graphicx}
\usepackage{float}
\usepackage{booktabs}
\usepackage[inline]{enumitem}
\usepackage[colorlinks=true,linkcolor=blue,urlcolor=red]{hyperref}
\usepackage[noabbrev]{cleveref}
\usepackage{tikz-cd}
% useful macro for class
\newcommand{\probability}[2]{\mathbb{\MakeUppercase{P}}_{#1} \left(#2\right)}
\newcommand{\variance}[2]{\mathrm{Var}_{#1} \left[ #2 \right]}
\newcommand{\expectation}[2]{\mathbb{\MakeUppercase{E}}_{#1} \left[#2\right]}
\newcommand{\at}[3]{\left.#1\right\vert_{#2}^{#3}}

\newcommand{\identity}{\mathrm{id}}
\newcommand{\sinc}{\mathop{\mathrm{sinc}}}
\newcommand{\rect}{\mathop{\mathrm{rect}}}
\newcommand{\tri}{\mathop{\mathrm{tri}}}
\newcommand{\Real}{\mathop{\mathrm{Re}}}

\newcommand{\Homomorphism}{\mathrm{Hom}}
\newcommand{\Morphism}{\mathrm{Mor}}
\newcommand{\Object}{\mathrm{Ob}}
\DeclareMathOperator{\lcm}{lcm}

\pdfminorversion=7

% for the big braces
\usepackage{bigdelim}

\usepackage{multicol}
\usepackage{xcolor}

% Fancy script capitals
\usepackage{mathrsfs}
\usepackage{cancel}
% Bold math
\usepackage{bm}

% Add \contra symbol to denote contradiction
\usepackage{stmaryrd} % for \lightning
\newcommand\contra{\scalebox{1.5}{$\lightning$}}

% \let\phi\varphi

% Command for short corrections
% Usage: 1+1=\correct{3}{2}

\definecolor{correct}{HTML}{009900}
\newcommand\correct[2]{\ensuremath{\:}{\color{red}{#1}}\ensuremath{\to }{\color{correct}{#2}}\ensuremath{\:}}
\newcommand\green[1]{{\color{correct}{#1}}}

% horizontal rule
\newcommand\hr{
	\noindent\rule[0.5ex]{\linewidth}{0.5pt}
}

% hide parts
\newcommand\hide[1]{}

% si unix
\usepackage{siunitx}

% Environments
\makeatother
% For box around Definition, Theorem, etc.
\usepackage{tcolorbox}
\numberwithin{equation}{section}
\theoremstyle{definition}
\tcbuselibrary{theorems, breakable}
\tcbset{parbox=false}
\newtcbtheorem[number within=section, crefname={Definition}{Definitions}]{definition}{Definition}{}{def}
\newtheorem*{example}{Example}
\newtheorem*{notation}{Notation}
\newtheorem*{prev}{As previously seen}
\newtheorem*{remark}{Remark}
\newtheorem*{note}{Note}
\newtheorem*{problem}{Problem}
\newtheorem*{exercise}{Exercise}
\newtheorem*{answer}{Answer}
\newtheorem*{observe}{Observe}
\newtheorem*{property}{Property}
\newtheorem*{intuition}{Intuition}
\newtheorem*{summary}{Summary}
\newtcbtheorem[number within=section, crefname={Theorem}{Theorems}]{theorem}{Theorem}{}{th}
\newtcbtheorem[number within=section, crefname={Lemma}{Lemmas}]{lemma}{Lemma}{}{le}
\newtcbtheorem[number within=section, crefname={Corollary}{Corollarys}]{corollary}{Corollary}{}{co}
\newtcbtheorem[number within=section, crefname={Proposition}{Propositions}]{proposition}{Proposition}{}{pr}
% \makeatletter
% \def\th@plain{%
% 	\thm@notefont{}% same as heading font
% 	\itshape % body font
% }
% \def\th@definition{%
% 	\thm@notefont{}% same as heading font
% 	\normalfont % body font
% }
% \makeatother

% Fix some spacing
% http://tex.stackexchange.com/questions/22119/how-can-i-change-the-spacing-before-theorems-with-amsthm
\makeatletter
\def\thm@space@setup{%
	\thm@preskip=\parskip \thm@postskip=0pt
}

\usepackage{xifthen}
\def\testdateparts#1{\dateparts#1\relax}
\def\dateparts#1 #2 #3 #4 #5\relax{
	\marginpar{\small\textsf{\mbox{#1 #2 #3 #5}}}
}

\def\@lecture{}%
\newcommand{\lecture}[3]{
	\ifthenelse{\isempty{#3}}{%
		\def\@lecture{Lecture #1}%
	}{%
		\def\@lecture{Lecture #1: #3}%
	}%
	\subsection*{\@lecture}
	\marginpar{\small\textsf{\mbox{#2}}}
}

% These are the fancy headers
\usepackage[headsepline]{scrlayer-scrpage}
\usepackage{lastpage}
\clearpairofpagestyles

\ohead{\@lecture}
\ihead{\rightmark}
\cfoot{\pagemark}
\renewcommand*\pagemark{{\usekomafont{pagenumber}Page\nobreakspace\thepage\ of \pageref*{LastPage}}}
\addtokomafont{pageheadfoot}{\upshape}

\makeatother

% TodoNotes and inline notes in fancy boxes
\usepackage{todonotes}
\setuptodonotes{inline}

% Make boxes breakable
\tcbuselibrary{breakable}

% Figure support as explained in my blog post.
\usepackage{import}
\usepackage{xifthen}
\usepackage{pdfpages}
\usepackage{transparent}
\newcommand{\incfig}[1]{%
	\def\svgwidth{\columnwidth}
	\import{./figures/}{#1.pdf_tex}
}

% Fix some stuff
% %http://tex.stackexchange.com/questions/76273/multiple-pdfs-with-page-group-included-in-a-single-page-warning
\pdfsuppresswarningpagegroup=1

\renewcommand{\qed}{\hfill$\blacksquare$}

\newcommand*{\MaxNumOfChapters}{10}% Adjust these two settings for your needs.
\newcommand*{\MaxNumOfSections}{6}%

\usepackage{pgffor}%

\newcommand{\lec}[2]{%
	\foreach \c in {#1,...,#2}{%
			\IfFileExists{Lectures/lec_\c.tex} {%
				\input{Lectures/lec_\c.tex}%
			}{}%
		}%
}

% Custom Commands

% Operators
\DeclareMathOperator{\ima}{im}
\DeclareMathOperator{\aut}{Aut}
\DeclareMathOperator{\pr}{\mathbb{P}}
\DeclareMathOperator{\ex}{\mathbb{E}}
\DeclareMathOperator{\var}{Var}
\DeclareMathOperator{\cov}{Cov}

\DeclarePairedDelimiter{\paren}{\lparen}{\rparen}

% abs and norm
\DeclarePairedDelimiter\abs{\lvert}{\rvert}%
\DeclarePairedDelimiter\norm{\lVert}{\rVert}%
% Swap the definition of \abs* and \norm*, so that \abs
% and \norm resizes the size of the brackets, and the
% starred version does not.
\makeatletter
\let\oldabs\abs
\def\abs{\@ifstar{\oldabs}{\oldabs*}}
%
\let\oldnorm\norm
\def\norm{\@ifstar{\oldnorm}{\oldnorm*}}
%
\let\oldceil\ceil
\def\ceil{\@ifstar{\oldceil}{\oldceil*}}
%
\let\oldfloor\floor
\def\floor{\@ifstar{\oldfloor}{\oldfloor*}}
%
\let\oldparen\paren
\def\paren{\@ifstar{\oldparen}{\oldparen*}}
\makeatother

% Matrix Functions
\DeclareMathOperator{\tr}{Tr}

\newcommand{\nsub}{\DOTSB\mathrel{\trianglelefteq}}
\newcommand{\Z}[1]{\nicefrac{\mathbb{Z}}{#1\mathbb{Z}}}

\newcommand{\fullfunction}[5]{%
    \begin{array}[t]{@{}r@{}r@{}c@{}l@{}}
    #1 \colon & #2 & {}\longrightarrow{} & #3 \\
                & #4 & {}\longmapsto{}     & #5
    \end{array}
}

% Commands for derivatives
\newcommand{\dif}{\mathop{}\!\mathrm{d}}
\newcommand{\Dif}{\mathop{}\!\mathrm{D}}

\makeatletter
\newcommand{\spx}[1]{%
  \if\relax\detokenize{#1}\relax
    \expandafter\@gobble
  \else
    \expandafter\@firstofone
  \fi
  {^{#1}}%
}
\makeatother

\newcommand\pd[3][]{\frac{\partial\spx{#1}#2}{\partial#3\spx{#1}}}
\newcommand{\md}[6]{\frac{\partial\spx{#2}#1}{\partial#3\spx{#4}\partial#5\spx{#6}}}
\newcommand{\od}[3][]{\frac{\dif\spx{#1}#2}{\dif#3\spx{#1}}}

% Evaluate at
\newcommand{\genericdel}[4]{%
  \ifcase#3\relax
  \ifx#1.\else#1\fi#4\ifx#2.\else#2\fi\or
  \bigl#1#4\bigr#2\or
  \Bigl#1#4\Bigr#2\or
  \biggl#1#4\biggr#2\or
  \Biggl#1#4\Biggr#2\else
  \left#1#4\right#2\fi
}

\newcommand{\eval}[2][-1]{\genericdel.|{#1}{#2}}