\lecture{1}{29 Apr. 2022}{}
\section*{Motivation}
\subsection{The Brachistochrone Problem}
\begin{problem}
Particle slides on a wire under influence of gravity between two fixed points \(A\), \(B\). Which shape of the wire gives the shortest travel time, starting from rest?
\end{problem}

The travel time is \(T = \int_{A}^{B} \frac{\dif \ell}{v(x, y)}\), and by energy conservation, and by energy conservation
\[
    \frac{1}{2} mv^2 + mg y = 0 \implies v = \sqrt{-2gy}.
\]
So
\[
    T[y] = \frac{1}{\sqrt{2g}}\int_{0}^{x_2} \frac{\sqrt{1 + (y')^2}}{\sqrt{-y}} \dif x
\]
subject to \(y(0) = 0\), \(y(x_2) = y_2\).
\subsection{Geodesics}
\begin{problem}
    What is the shortest path \(\gamma\) between two points \(A\), \(B\) on a surface.
\end{problem}
Take \(\Sigma = \mathbb{R}^2\). The distance along \(\gamma\) is
\[
    D[y] = \int_{A}^{B}\dif \ell = \int_{x_1}^{x_2} \sqrt{1 + (y')^2} \dif x,
\]
and we want to minimize \(D\) by varying \(\gamma\).
\subsection{Introduction}
In general, we want to minimize (maximize)
\[
    F[y] = \int_{x_1}^{x_2} f(x, y, y') \dif x
\]
among all functions s.t. \(y(x_1) = y_1\), \(y(x_2) = y_2\). The expression is a \textit{functional}. (A function on a space of functions). Functions map numbers to numbers, and functionals map functions to numbers.

Area under the graph is when \(f = y\) and the length of a curve is when \(f = \sqrt{1 + (y')^2}\).

Calculus of variations finds extrema of functionals on space of functions.
\begin{notation}
    \begin{itemize}
        \item \(C(\mathbb{R})\) is space of continuous functions on \(\mathbb{R}\).
        \item \(C^k(\mathbb{R})\) is the space of functions with continuous \(k\)th derivatives.
        \item \(C^k_{(\alpha,\beta)}\) is the space of functions with continuous \(k\)th derivatives and \(f(\alpha) = f(\beta)\).
    \end{itemize}
    We need to specify the function space beforehand. It is a branch of functional analysis---Part III analysis on the space of functions, while Analysis I is analysis on the number line. Variational Principles follows principles in Nature, where the laws follow from extremizing functionals.
\end{notation}

\begin{example}[Fermat's Principle]
    Light between two pints travel along paths which require least time.
\end{example}

\begin{example}[Principle of Least Action]
    Let \(T\) be the kinetic energy \(\frac{m \abs{\dot{\mathbf{x}}}^2}{2}\) and potential energy \(V = V(\mathbf{x})\).
    \[
        S[\gamma] = \int_{t_1}^{t_2} (T - V) \dif t
    \]
    is minimized along paths of motion.

    Leibniz commented on this, saying that ``we live in the best of all worlds''.

    Feynman's take on this: ``This is wrong. In quantum theorem the motion takes place along all possible path with different possibilities''. [See Part III QFT]
\end{example}

In this course, we discuss
\begin{enumerate}
    \item necessary condition for extrema of the Euler Lagrange Equations;
    \item lots of examples (geometry, physics, problems with constraints);
    \item second variation (some sufficient condition of extrema).
\end{enumerate}
The following books will be useful
\begin{enumerate}
    \item Gelfand - Fomin ``Calculus of Variations'';
    \item DAMTP notes (e.g. P. Townsend);
    \item Lectures are self-contained.
\end{enumerate}